

%	options include 12pt or 11pt or 10pt
%	classes include article, report, book, letter, thesis

\title{}



\author{Brendan Whitaker}


\documentclass[12pt,oneside,reqno]{amsart}

%-------------------------------------
%--------PREAMBLE---------------------

%    Include referenced packages here.
\usepackage{}
\usepackage[margin=1in]{geometry}
\usepackage{graphicx}
\usepackage[margin=1in]{geometry}
\usepackage{amsmath}
\usepackage{amssymb}
\usepackage{amsthm}
\usepackage{bbm}
\usepackage{cancel}
\usepackage{verbatim}
\usepackage{amsrefs}
\usepackage{enumitem}
\usepackage{hyperref}
\usepackage{tikz-cd}
%\usepackage[pdf]{pstricks}
\usepackage{braket}
\usetikzlibrary{cd}
\hypersetup{
     colorlinks   = true,
     citecolor    = red
}
%\usepackage{adjustbox}
\usepackage[ruled,linesnumbered]{algorithm2e}
\usepackage{adjustbox}
\usepackage{changepage}


\let\oldemptyset\emptyset
\let\emptyset\varnothing

\theoremstyle{plain}
\newtheorem{Thm}{Theorem}
\newtheorem{Prob}[Thm]{Problem}
%\theoremstyle{definition}
\newtheorem{Remark}[Thm]{Remark}
\newtheorem{Tech}[Thm]{Technical Remark}
\newtheorem*{Claim}{Claim}
%----------------------------------------
%CHAPTER STUFF
\newtheorem{theorem}{Theorem}%[chapter]
%\numberwithin{section}%{chapter}
%\numberwithin{equation}%{chapter}
%CHAPTER STUFF
%----------------------------------------
\newtheorem{lem}[theorem]{Lemma}
%\newtheorem{Q}[theorem]{Question}
\newtheorem{Prop}[theorem]{Proposition}
\newtheorem{Cor}[theorem]{Corollary}

\theoremstyle{definition}
\newtheorem{e}{Exercise}
\newtheorem{Def}[theorem]{Definition}
\newtheorem{Ex}[theorem]{Example}
\newtheorem{xca}[theorem]{Exercise}



\theoremstyle{remark}
\newtheorem{rem}[theorem]{Remark}


\newcommand{\Mod}[1]{\ (\mathrm{mod}\ #1)}
\newcommand{\norm}{\trianglelefteq}
\newcommand{\propnorm}{\triangleleft}
\newcommand{\semi}{\rtimes}
\newcommand{\sub}{\subseteq}
\newcommand{\fa}{\forall}
\newcommand{\R}{\mathbb{R}}
\newcommand{\z}{\mathbb{Z}}
\newcommand{\n}{\mathbb{N}}
\newcommand{\Q}{\mathbb{Q}}
\renewcommand{\c}{\mathbb{C}}
\newcommand{\bb}{\vspace{3mm}}

\newcommand{\bee}{\begin{equation}\begin{aligned}}
\newcommand{\eee}{\end{aligned}\end{equation}}
\newcommand{\nequiv}{\not\equiv}
\newcommand{\lc}[2]{#1_1 + \cdots + #1_{#2}}
\newcommand{\lcc}[3]{#1_1 #2_1 + \cdots + #1_{#3} #2_{#3}}
\newcommand{\ten}{\otimes} %tensor product
\newcommand{\fracc}{\frac}
\newcommand{\tens}{\otimes}
\newcommand{\lpar}{\left(}
\newcommand{\rpar}{\right)}
\newcommand{\floor}{\lfloor}

\renewcommand{\rm}{\normalshape}%text inside math
\renewcommand{\Re}{\operatorname{Re}}%real part
\renewcommand{\Im}{\operatorname{Im}}%imaginary part
\renewcommand{\bar}{\overline}%bar (wide version often looks better)
\renewcommand{\phi}{\varphi}

\makeatletter
\newenvironment{restoretext}%
    {\@parboxrestore%
     \begin{adjustwidth}{}{\leftmargin}%
    }{\end{adjustwidth}
     }
\makeatother

%---------END-OF-PREAMBLE---------
%---------------------------------







\begin{document}

\title{CSE 6331 Homework 9}

\date{SP18}

\author[Brendan Whitaker]{Brendan Whitaker}

\maketitle



\begin{enumerate}[label=\arabic*.]
\item \textit{Consider a flow network in which vertices, as well as edges, have capacities.  
In addition to the 
original edge capacity constraint, 
there is now a new vertex capacity constraint: 
the total positive flow 
entering any vertex $u$ cannot exceed its capacity $c(u)$. Show that determining the maximum flow in a 
network with edge and vertex capacities can be reduced to an ordinary maximum flow problem. }

\bb

\begin{proof}
We simply replace each vertex $u$ with a pair of vertices $v_1,v_2$ and an edge $(v_1,v_2) \in E$ between them such that $c(u) = c(v_1,v_2)$. All incoming edges to $u$ now enter $v_1$, and all outgoing edges from $u$ now exit $v_2$.  
\end{proof}

\item \textit{Suppose that during an execution of Relabel-to-Front, Discharge$(u)$ is called \textbf{twice} for some particular node $u$. Prove that if an edge $(u,v)$ is inadmissible at the end/exit of the first Discharge$(u)$, then it is still inadmissible at the beginning/entry of the second Discharge$(u)$. }

\bb

\begin{proof}
Let $(u,v)$ be inadmissible at the exit of the first Discharge$(u)$. Then either $c_f(u,v) = 0$ or $h(u) \leq h(v)$. 

\bb


\textbf{Case 1:} Suppose $c_f(u,v) = 0$. Then after we exit the first Discharge$(u)$, we could execute a push from $v$ to $u$ before we come back to the second Discharge$(u)$, or we could not. Suppose we push from $v$ to $u$ before starting the second Discharge$(u)$. Then we know that in order to push from $v$ to $u$, we had to have $h(v) = h(u) + 1$. So since the height of $u$ will not change since we are not discharging $u$, we know that when we enter Discharge$(u)$ for the second time $h(v) \geq h(u) + 1$, since $h(v)$ never decreases. But then we know that at the start of the second Discharge$(u)$, we have $h(u) \leq h(v) - 1 < h(v) + 1$, hence we must have that $(u,v)$ is inadmissible. 

\textbf{Case 2:} Suppose $c_f(u,v) > 0$. Then we must have that $h(u) \leq h(v)$ at the exit of the first Discharge$(u)$ since we said $(u,v)$ is inadmissible. This is because since $(u,v)$ is a residual edge, we know $h(u) \leq h(v) + 1$ , and we can't have $h(u) = h(v) + 1$ since this would violate the assumption that $(u,v)$ is inadmissible. Now since $h(u)$ does not change if we are not in Discharge$(u)$, and $h(v)$ can only increase, we know that when we come back to the start of the second Discharge$(u)$, $h(u) \leq h(v)$ still, so $(u,v)$ is still inadmissible. 
\end{proof}



\item \textit{Let $G = (V,E)$ be a flow network with source $s$, sink $t$, and integer capacities. Suppose we are given a maximum flow $f$ in $G$, and suppose the capacity of a single edge $(u^*,v^*) \in E$ is \textbf{increased} by 1. Give an $O(V + E)$-time algorithm to update the maximum flow. }

\bb
\begin{restoretext}
\begin{algorithm}[H]\label{alg1}
\KwData{The set of vertices $V$ and edges $E$ of the given flow network $G$. A maximum flow $f$ in $G$. }
\KwResult{The updated maximum flow $f$. }
\Begin{
\tcc{We compute the residual network $E_f$ of $G$. }
\For{$(u,v) \in E$} {
	\If{$c(u,v) - f(u,v) > 0$}{
		\textbf{add} $(u,v)$ \textbf{to} $E_f$\;
	}
}
\tcc{We check if $(u^*,v^*) \in E_f$. }
\If{$(u^*,v^*) \in E_f$}{
	\textbf{return} $f$\;
}
\tcc{Else, we do DFS on $s$ in residual network to find if there is a path to $t$ (augmenting path). We return a linked list of the augmenting path if it exists, and we return NULL otherwise. }
$augPath \longleftarrow DFS(V,E_f,s,t)$\;
\If{$augPath = NULL$} {
	\textbf{return} $f$\;
}

$child \longleftarrow augPath.root()$\;
\tcc{We increase the flow by 1 for each edge in the linked list (augmenting path). }
\While{$augPath.next(child) \neq NULL$}{
	$u \longleftarrow child$\;
	$v \longleftarrow augPath.next(child)$\;
	$f(u,v) \longleftarrow f(u,v) + 1$\;
}
\Return $f$\;


}
\caption{Update-Flow$(G)$}
\end{algorithm}
\end{restoretext}
\bb


\item \textit{Let $G = (V,E)$ be a flow network with source $s$, sink $t$, and integer capacities. Suppose we are given a maximum flow $f$ in $G$, and suppose the capacity of a single edge $(u,v) \in E$ is decreased by 1. Give an $O(V + E)$-time algorithm to update the maximum flow. }


\bb
\begin{restoretext}
\begin{algorithm}[H]\label{alg2}
\KwData{The set of vertices $V$ and edges $E$ of the given flow network $G$. A maximum flow $f$ in $G$. }
\KwResult{The updated maximum flow $f$. }
\Begin{
\tcc{We compute the residual network $E_f$ of $G$. }
\For{$(u,v) \in E$} {
	\If{$c(u,v) - f(u,v) > 0$}{
		\textbf{add} $(u,v)$ \textbf{to} $E_f$\;
	}
}
\tcc{We check if $(u^*,v^*) \in E_f$. }
\If{$(u^*,v^*) \in E_f$}{
	\textbf{return} $f$\;
}
\tcc{Else, we do DFS on $s$ in $G$ with the max flow $f$ to find if there is a path to $u^*$. We return a linked list of the path if it exists, and we return NULL otherwise. We do the same for DFS from $v^*$ to $t$. }
$pathU \longleftarrow DFS(G,f,s,u^*)$\;
$pathV \longleftarrow DFS(G,f,v^*,t)$\;
\If{$pathU = NULL$ \textbf{or} $pathV = NULL$} {
	\Return $f$\;
}

$child \longleftarrow pathU.root()$\;
\tcc{Else, we decrease the flow by 1 for each edge in the linked lists. }
\While{$pathU.next(child) \neq v^*$}{
	$u \longleftarrow child$\;
	$v \longleftarrow pathU.next(child)$\;
	$f(u,v) \longleftarrow f(u,v) - 1$\;
}
\While{$pathV.next(child) \neq NULL$}{
	$u \longleftarrow child$\;
	$v \longleftarrow pathV.next(child)$\;
	$f(u,v) \longleftarrow f(u,v) - 1$\;
}
\Return $f$\;


}
\caption{Update-Flow2$(G)$}
\end{algorithm}
\end{restoretext}
\bb






\end{enumerate}
























































































































\end{document}


