

%	options include 12pt or 11pt or 10pt
%	classes include article, report, book, letter, thesis

\title{Math 5590H Bonus}



\author{Brendan Whitaker}

\date{AU17}
\documentclass[10pt,oneside,reqno]{amsart}

\usepackage{graphicx}
\usepackage[margin=1in]{geometry}
\usepackage{amsmath}
\usepackage{amssymb}
\usepackage{amsthm}
\usepackage{bbm}
\usepackage{cancel}
\usepackage{verbatim}
\usepackage{amsrefs}
\usepackage{enumitem}
\usepackage{etoolbox}% http://ctan.org/pkg/etoolbox
\patchcmd{\thmhead}{(#3)}{#3}{}{}
\usepackage{braket}


\theoremstyle{plain}
\newtheorem{theorem}{Theorem}
\newtheorem{Thm}{Theorem}
\newtheorem{Cor}[Thm]{Corollary}
\newtheorem{Prop}[Thm]{Proposition}
\newtheorem{Lem}[Thm]{Lemma}
\newtheorem{Prob}[Thm]{Problem}
\newtheorem{Def}[Thm]{Definition}
\newtheorem{Q}[Thm]{Question}
\newtheorem*{e}{Exercise}
\newtheorem{ee}{Exercise}
\theoremstyle{definition}
\newtheorem{Remark}[Thm]{Remark}
\newtheorem{Tech}[Thm]{Technical Remark}
\newtheorem*{Claim}{Claim}
\newtheorem{Ex}[Thm]{Example}




\newcommand{\Mod}[1]{\ (\mathrm{mod}\ #1)}
\newcommand{\norm}{\trianglelefteq}
\newcommand{\propnorm}{\triangleleft}
\newcommand{\semi}{\rtimes}
\newcommand{\sub}{\subseteq}
\newcommand{\fa}{\forall}
\newcommand{\R}{\mathbb{R}}
\newcommand{\z}{\mathbb{Z}}



\begin{document}

\title{Math 5590H Final Theorems}

\date{AU17}

\author[Brendan Whitaker]{Brendan Whitaker}

\maketitle

\begin{theorem}
$Inn(G) \cong G/Z(G)$. 
\end{theorem}

\begin{theorem}
$F[x]$ is an ED. 
\end{theorem}

\begin{theorem}
$F[x]/(f(x))$ is a field if and only if $f(x)$ is irreducible. 
\end{theorem}

\begin{theorem}
Ways to prove a group is abelian: 
\begin{enumerate}
\item Show that the commutator $xyx^{-1}y^{-1}$ of any two elements is trivial. 

\item Show the group is a direct product of abelian groups. 

\end{enumerate}
\end{theorem}

\begin{theorem}
If $P \cap Q = 1$, and $|PQ| = |G|$, then $PQ = G$. 
\end{theorem}

\begin{theorem}
If $P \norm G$, $P \cap Q = 1$, and $PQ = G$, then $P \semi Q = G$. 
\end{theorem}

\begin{theorem}
If $P,Q$ are sylow $p,q$-subgroups of a group $G$ with only two distinct prime factors, and $n_p = 1$, then $P \semi Q = G$. 
\end{theorem}

\begin{theorem}
If $\mathbb{Z}_n = \mathbb{Z}_m \times \mathbb{Z}_k$ and $m,k$ are relatively prime, then we must have $\mathbb{Z}_n = \mathbb{Z}_{mk}$, and if $(m,k) = 1$, then $\z_{mk} \cong \z_m \times \z_k$. 
\end{theorem}

\begin{theorem}
If $N \norm G$ and both $N$ and $G/N$ are solvable, then $G$ is solvable. 
\end{theorem}

\begin{theorem}
All $p$-groups are nilpotent. 
\end{theorem}

\begin{theorem}
Any subring must be an additive subgroup. 
\end{theorem}

\begin{theorem}
Any cyclic group of a cyclic group $(\z)$ is cyclic. 
\end{theorem}

\begin{theorem}
A homomorphism is injective if and only if its kernel is $(0)$. 
\end{theorem}

\begin{theorem}
The ideal $(1) = R$, the whole ring, and the ideal $(0) = \{0\}$ is just the ideal containing only the element $0$. 
\end{theorem}

\begin{theorem}
Ways to show an ideal $M$ is maximal: 
\begin{enumerate}
\item Show that if an ideal $I$ contains $M$ then $I = M$ or $I = R$, the whole ring. 

\item Show that $R/M$ is a field. 
\end{enumerate}
\end{theorem}

\begin{theorem}
An ideal $P$ is prime if and only if the quotient ring $R/P$ is an integral domain. 
\end{theorem}

\begin{Cor}
See page 685 for information on Noetherian rings, prime ideals, radicals, etc. 
\end{Cor}

\begin{theorem}
If $x$ is nilpotent then $\phi(x)$ is nilpotent (Exercise 7.3.32). 
\end{theorem}

\begin{theorem}
If $\phi$ is surjective, then the preimage of a maximal ideal is maximal. 
\end{theorem}

\begin{theorem}
Any nonzero ring homomorphism from a field into a ring is injective (Corollary 7.4.10). 
\end{theorem}

\begin{theorem}
If $\phi$ is surjective, the image of an ideal is an ideal. 
\end{theorem}
\begin{proof}
Let $\phi:R \to S$ be a surjective hom. Then let I be an ideal in $R$. COnsider $\phi(i)$. We want to show that $\phi(I)s \sub \phi(I)$ $\fa s \in S$. So since $\phi$ is sujective, $\exists r \in R$ s.t. $\phi(r) = s$. And note $Is \sub I$ since I is ideal. So we have $\phi(Is) = \phi(I)\phi(s) \sub \phi(I)$ which tells us that the image is indeed an ideal by definition. 
\end{proof}

\begin{theorem}
Any ideal in a commutative, unital ring is a subring. 
\end{theorem}

\begin{theorem}
$\z[i],\z$ are EDs. 
\end{theorem}

\begin{theorem}
Maximal ideals are always prime. 
\end{theorem}

\begin{theorem}
In a PID, every nonzero prime ideal is maximal. 
\end{theorem}

\begin{theorem}
In UFDs, irreducible if and only if prime. 
\end{theorem}

\begin{theorem}
Primes in the Gaussian integers. Note that the conjugate of any prime is also prime here. A Gaussian integer is prime if and only if: one of $a,b$ is zero and its absolute value is a prime of the form $4k + 3$, or both are nonzero and $a^2 + b^2$ is a prime number. Refer to Proposition 18 on page 291. 
\end{theorem}

\begin{theorem}
Ideals can be principal but not maximal/prime in a PID. Consider $4\z$. It is not prime in $\z$ but it is principal. 
\end{theorem}

\begin{theorem}
Every ideal is the kernel of some ring hom. 
\end{theorem}

\begin{theorem}
Prime iff the quotient ring is an Integral domain. 
\end{theorem}

\begin{theorem}
Maximal if and only if the quotient ring is a field. 
\end{theorem}


















\end{document}



