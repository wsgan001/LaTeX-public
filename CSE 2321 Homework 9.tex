

%	options include 12pt or 11pt or 10pt
%	classes include article, report, book, letter, thesis

\title{CSE 2321 Homework 9}



\author{Brendan Whitaker}

\date{AU17}
\documentclass[10pt,oneside,reqno]{amsart}

\usepackage{graphicx}
\usepackage[margin=1in]{geometry}
\usepackage{amsmath}
\usepackage{amssymb}
\usepackage{amsthm}
\usepackage{bbm}
\usepackage{cancel}
\usepackage{verbatim}
\usepackage{amsrefs}
\usepackage{enumitem}


\theoremstyle{plain}
\newtheorem{Thm}{Theorem}
\newtheorem{Cor}[Thm]{Corollary}
\newtheorem{Prop}[Thm]{Proposition}
\newtheorem{Lem}[Thm]{Lemma}
\newtheorem{Prob}[Thm]{Problem}
\newtheorem{Def}[Thm]{Definition}
\newtheorem{Q}[Thm]{Question}
\theoremstyle{definition}
\newtheorem{Remark}[Thm]{Remark}
\newtheorem{Tech}[Thm]{Technical Remark}
\newtheorem*{Claim}{Claim}
\newtheorem{Ex}[Thm]{Example}



\newcommand{\Mod}[1]{\ (\mathrm{mod}\ #1)}



\begin{document}

\title{CSE 2321 Homework 9}

\date{AU17}

\author[Brendan Whitaker]{Brendan Whitaker}

\maketitle

\begin{enumerate}[label=\arabic*.]

\item 

\begin{enumerate}

\item 
We find the bounds of the summation for the first while loop. \\\\
\begin{tabular}{c|c}
iterations & $i = n$\\
\hline
1 & $6n$\\
2 & $6^2n$\\
\vdots & \vdots\\
$k$ & $6^kn$
\end{tabular}.\\\\
So we have $5n^3 = 6^kn \Rightarrow 6^k = 5n^2 \Rightarrow k = \log_6(5n^2)  = \boxed{2\log_6(n)} + \log_6(5)$. So $1 \leq k \leq \lfloor 2\log_6(n) \rfloor$. Now we find the bounds for the second while loop\\\\
\begin{tabular}{c|c}
iterations & $j = 6n^2$\\
\hline
1 & $6n^2/4$\\
2 & $6n^2/4^2$\\
\vdots & \vdots\\
$l$ & $6n^2/4^l$
\end{tabular}.\\\\
So we have $3 = 6n^2/4^l \Rightarrow 4^l = 2n^2 \Rightarrow l = \log_4(2n^2) = \boxed{2\log_4(n)} + \log_4(2)$. Thus $1 \leq l \leq \lfloor 2\log_4(n)\rfloor$. Hence the running time is given by



\begin{equation}
\begin{aligned}
t &= \sum_{k = 1}^{\lfloor 2\log_6(n)\rfloor} \sum_{l = 1}^{\lfloor 2\log_4(n)\rfloor} c = \sum_{k = 1}^{\lfloor 2\log_6(n)\rfloor} 2\log_4(n)c = \frac{4}{\log_2(4)\log_2(6)}(\log_2(n))^2c \\
&= \boxed{\Theta((\log_2(n))^2).} 
\end{aligned}
\end{equation}

\item We find the bounds for the while loop\\
\begin{tabular}{c|c}
iterations & $j = 2i$\\
\hline
1 & $3 \cdot 2i$\\
2 & $3^2 \cdot 2i$\\
\vdots & \vdots\\
$k$ & $3^k \cdot 2i$
\end{tabular}.\\\\
So $3^k \cdot 2i = i^4 \Rightarrow k = \\log_3(i^3/2) = 3\log_3(i) + \log_3(2)$. So $1 \leq k \leq \lfloor 3\\log_3(i)\rfloor$. Thus the running time is given by

\begin{equation}
\begin{aligned}
t &= \sum_{i = 1}^{3n^2} \sum_{k = 1}^{\lfloor 3\log_3(i)\rfloor}c = \sum_{i = 1}^{3n^2} 3\log_3(i)c = \sum_{i = 1}^{n^2 - 1} 3\log_3(i)c + \sum_{i = n^2}^{3n^2} 3\log_3(i)c. 
\end{aligned}
\end{equation}
We find an upper bound, plugging in $3n^2$ for $i$
\begin{equation}
\begin{aligned}
t &\leq \sum_{i = 1}^{3n^2} 3\log_3(3n^2)c = 9n^2( 2(\log_3(n)) + 1)c = \boxed{18cn^2\log_3(n)} + 9cn^2. 
\end{aligned}
\end{equation}
Now we find a lower bound, splitting the sum and plugging in $n^2$ for $i$
\begin{equation}
\begin{aligned}
t &\geq \sum_{i = n^2}^{3n^2} 3\log_3(n^2)c = \boxed{12cn^2\log_3(n). }
\end{aligned}
\end{equation}
So $t = c_1n^2\log_3(n)$, where $12c \leq c_1 \leq 18c$, so $t = \boxed{\Theta(n^2\log_3(n)).}$

\item We find the bounds of the summation for the first while loop. \\\\
\begin{tabular}{c|c}
iterations & $i = n$\\
\hline
1 & $n + 5$\\
2 & $n + 2 \cdot 5$\\
\vdots & \vdots\\
$k$ & $n + 5k$
\end{tabular}.\\\\
So $2n^3 = n + 5k \Rightarrow 5k = 2n^3 - n \Rightarrow k = \boxed{\lfloor \frac{2}{5}n^3\rfloor} - \frac{n}{5}$. So $1 \leq k \leq \lfloor \frac{2}{5}n^3\rfloor$. 
Now we find the bounds of the summation for the second while loop\\\\
\begin{tabular}{c|c}
iterations & $j = i^2$\\
\hline
1 & $i^2/4$\\
2 & $i^2/4^2$\\
\vdots & \vdots\\
$l$ & $i^2/4^l$
\end{tabular}.\\\\
So we have $i^2/4^l = i \Rightarrow 4^l = i \Rightarrow l = \boxed{\lfloor \log_4(i)\rfloor.}$ Thus $1 \leq l \leq \lfloor \log_4(i) \rfloor$. Then
\begin{equation}
\begin{aligned}
t &= \sum_{k = 1}^{\lfloor \frac{2}{5}n^3\rfloor} \sum_{l = 1}^{\lfloor \log_4(i) \rfloor}c = \sum_{k = 1}^{\lfloor \frac{2}{5}n^3\rfloor} \log_4(i)c = \sum_{k = 1}^{\lfloor \frac{1}{5}n^3\rfloor - 1}\log_4(i)c + \sum_{k = \lfloor \frac{1}{5}n^3\rfloor}^{\lfloor \frac{2}{5}n^3\rfloor}\log_4(i)c. 
\end{aligned}
\end{equation}
We find an upper bound, plugging in $\frac{2}{5}n^3$ for $k$. Then $i = n + 2n^3$, so since we are taking an upper bound, we let $i = 3n^3$. Thus we have
\begin{equation}
\begin{aligned}
t &\leq \sum_{k = 1}^{\lfloor \frac{2}{5}n^3\rfloor} \log_4(3n^3)c = \frac{2}{5}cn^3 \log_4(3n^3) = \frac{2}{5}cn^3 (3\log_4(n) + \log_4(3))\\
&= \boxed{\frac{6}{5}cn^3\log_4(n) }+ \frac{2}{5}cn^3\log_4(3). 
\end{aligned}
\end{equation}
Now we find a lower bound, splitting the summation and plugging in $\frac{1}{5}n^3$ for $k$. Then $i = n + n^3$. So since we are finding a lower bound, we take $i = n^3$. Then we have
\begin{equation}
\begin{aligned}
t &\geq \sum_{k = \lfloor \frac{1}{5}n^3\rfloor}^{\lfloor \frac{2}{5}n^3\rfloor}\log_4(n^3)c = \frac{1}{5}n^3\log_4(n^3)c = \boxed{\frac{3}{5}cn^3\log_4(n). }
\end{aligned}
\end{equation}
Hence $t = c_2n^3\log_4(n)$, where $\frac{3}{5}c \leq c_2 \leq \frac{6}{5}c$. So $t = \boxed{\Theta (n^3\log_4(n)).}$\\









\end{enumerate}

\item Note
\begin{equation}
\begin{aligned}
f_a(n) &= \Theta(n^4\log(n)),\\
f_b(n) &= \Theta (4^n),\\
f_c(n) &= \Theta (n^{0.8}),\\
f_d(n) &= \Theta (1),\\
f_e(n) &= \Theta(n^{0.7}),\\
f_f(n) &= \Theta(n^5),\\
f_g(n) &=\Theta( n^2),\\
f_h(n) &= \Theta(n^{1.5}),\\
f_i(n) &= \Theta(n^{2.5}),\\
f_j(n) &= \Theta(n^2 \log(n)),\\
f_k(n) &= \Theta(n \log(n)),\\
f_l(n) &= \Theta(n^4),\\
f_m(n) &= \Theta(1),\\
f_n(n) &= \Theta(n^n),\\
\end{aligned}
\end{equation}
Let $\nu \in \{a,b,...,n\}^{14}$ s.t. 
\begin{equation}
\begin{aligned}
f_{\nu_i} = O(f_{\nu_{i + 1}})
\end{aligned}
\end{equation}
for $1 \leq i \leq 14$. Then 
\begin{equation}
\begin{aligned}
\nu = (d,m,e,c,k,h,g,j,i,l,a,f,b,n).
\end{aligned}
\end{equation}
Also note $f_m(n) = \Theta(f_d(n))$. 






\end{enumerate}











\end{document}


