

%	options include 12pt or 11pt or 10pt
%	classes include article, report, book, letter, thesis

\title{CSE 2331 Homework 3}



\author{Brendan Whitaker}

\date{AU17}
\documentclass[10pt,oneside,reqno]{amsart}

\usepackage{graphicx}
\usepackage[margin=1in]{geometry}
\usepackage{amsmath}
\usepackage{amssymb}
\usepackage{amsthm}
\usepackage{bbm}
\usepackage{cancel}
\usepackage{verbatim}
\usepackage{amsrefs}
\usepackage{enumitem}


\theoremstyle{plain}
\newtheorem{Thm}{Theorem}
\newtheorem{Cor}[Thm]{Corollary}
\newtheorem{Prop}[Thm]{Proposition}
\newtheorem{Lem}[Thm]{Lemma}
\newtheorem{Prob}[Thm]{Problem}
\newtheorem{Def}[Thm]{Definition}
\newtheorem{Q}[Thm]{Question}
\theoremstyle{definition}
\newtheorem{Remark}[Thm]{Remark}
\newtheorem{Tech}[Thm]{Technical Remark}
\newtheorem*{Claim}{Claim}
\newtheorem{Ex}[Thm]{Example}



\newcommand{\Mod}[1]{\ (\mathrm{mod}\ #1)}



\begin{document}

\title{CSE 2331 Homework 5}

\date{AU17}

\author[Brendan Whitaker]{Brendan Whitaker}

\maketitle

\begin{enumerate}[label=\arabic*.]

\item 
The inner while loop executes $log_2(n - 3) - 1$ times, and so takes $clog_2(n)$ time. 
\begin{enumerate}
\item In the worst case, $k\geq n/5$. So we have:
\begin{equation}
\begin{aligned}
T(n) &= \sum_{i = 1}^{n^4}clog_2(n) = cn^4log_2(n) \in \Theta(n^4log_2(n)). 
\end{aligned}
\end{equation}
\item $Prob(k < n/5) = \lfloor n/5 \rfloor / n \approx 1/5$. Let $X$ be the number of times $k<n/5$. \\
The running time is $clog_2(n)(X + 1) + c_1$, since lines 4-7 happen before our if statement. \\
The expected running time is $E(clog_2(n)(X + 1) + c_1) = clog_2(n)E(X) + clog_2(n) + c_1$, by linearity of expectation. \\
We use the formula $E(X) = \sum_{l = 1}^\infty Prob(X \geq l)$. \\
$Prob(X \geq l) = \left(\frac{1}{5}\right)^{l}$ if $l \leq n^4$. \\
$Prob(X \geq l) = 0$ if $l > n^4$. 
\begin{equation}
\begin{aligned}
E(X) &= \sum_{l = 1}^\infty Prob(X \geq l) = \sum_{l = 1}^{n^4} Prob(X \geq l) = \sum_{l = 1}^{n^4} \left(\frac{1}{5}\right)^{l}\\
&\leq \sum_{l = 1}^{\infty} \left(\frac{1}{5}\right)^{l} = \frac{1}{1 - \frac{1}{5}} - 1 = \frac{1}{4} \in O(1),\\
E(X) &\geq \frac{1}{5} \in \Omega(1). 
\end{aligned}
\end{equation}
Thus $E(X) \in \Theta(1)$. Hence $ET(n) = clog_2(n)(c_2 + 1 )+ c_1 \in \Theta(log_2(n))$. \\
\end{enumerate}

\item Steps 3-7 take $clog_2(n)$ time as computed in the above problem. 
\begin{enumerate}
\item In the worst case, $k < n/5$, so we have:
\begin{equation}
\begin{aligned}
T(n) &= \sum_{i = 1}^{\sqrt{n}}clog_2(n) = n^{1/2}clog_2(n) \in \Theta(n^{1/2}log_2(n)). 
\end{aligned}
\end{equation}
\item $Prob(k < n/5) = \lfloor n/5 \rfloor / n \approx 1/5$. \\
Let $X$ be the number of times $k<n/5$. \\
The running time is given by $clog_2(n)(X + 1) + c_1$ since the steps whose time is given by $clog_2(n)$ execute before the if-return-statement. \\
So the expected running time is given by: $ET(n) = E(clog_2(n)(X + 1) + c_1) = clog_2(n)(E(X) + 1) + c_1$ by linearity of expectation. \\
$E(X) = \sum_{l = 1}^{\infty}Prob(X \geq l)$. \\
$Prob(X \geq l) = \left(\frac{1}{5}\right)^{l}$ if $l \leq \lfloor n^{1/2} \rfloor$. \\
$Prob(X \geq l) = 0$ if $l > \lfloor n^{1/2} \rfloor$. \\
So we have:
\begin{equation}
\begin{aligned}
E(X) &= \sum_{l = 1}^{\infty}Prob(X \geq l) = \sum_{l = 1}^{\lfloor n^{1/2} \rfloor}Prob(X \geq l) = \sum_{l = 1}^{\lfloor n^{1/2} \rfloor}\left(\frac{1}{5}\right)^{l}\\
&\leq \sum_{l = 1}^{\infty}\left(\frac{1}{5}\right)^{l} = \frac{1}{4} \in O(1),\\
E(X) &= \sum_{l = 1}^{\lfloor n^{1/2} \rfloor}\left(\frac{1}{5}\right)^{l} \geq \frac{1}{5} \in \Omega(1). 
\end{aligned}
\end{equation}
Thus $E(X) \in \Theta(1)$, so let $E(X) = c_2$. Then we have:
\begin{equation}
\begin{aligned}
ET(n) = clog_2(n)(c_2 + 1) + c_1 \in \Theta(log_2(n)). 
\end{aligned}
\end{equation}
\end{enumerate}
\item Steps 1-11 take $cn^{2.5}$ time. 
\begin{enumerate}
\item In the worst case, $k = n - 1$. So we have:
\begin{equation}
\begin{aligned}
T(n) &= cn^{2.5} + T(n - 1)\\
&= cn^{2.5} + c(n - 1)^{2.5} +  T(n - 2)\\
&= cn^{2.5} + c(n - 1)^{2.5} + c(n - 2)^{2.5} + \cdots + T(0)\\
&\leq cn^{2.5} + cn^{2.5} + cn^{2.5} + \cdots + T(0)\\
&= ncn^{2.5} = cn^{3.5} \in O(n^{3.5}),\\
T(n) &\geq cn^{2.5} + c(n - 1)^{2.5} + c(n - 2)^{2.5} + \cdots + c(n - \frac{n}{2})^{2.5}\\
&\geq c\left(\frac{n}{2}\right)^{2.5} + c\left(\frac{n}{2}\right)^{2.5} + \cdots + c\left(\frac{n}{2}\right)^{2.5}\\
&= \frac{n}{2}c\left(\frac{n}{2}\right)^{2.5} \in \Omega(n^{3.5}). 
\end{aligned}
\end{equation}
So $T(n) \in \Theta(n^{3.5})$. 
\item $Prob(k = l) = \frac{1}{n - 1}$ for  $1 \leq l \leq n - 1$. \\
$Prob(k \leq n/2) = \frac{1}{2}$. \\
$Prob(k > n/2) = \frac{1}{2}$. \\
$ET(k \leq n/2) \leq ET(k = n/2) = cn^{2.5} + ET(n/2)$. \\
$ET(k > n/2) \leq ET(k = n -1) = cn^{2.5} + ET(n - 1)$. \\
\begin{equation}
\begin{aligned}
ET(n) &= Prob(k \leq n/2)Time(k \leq n/2) + Prob(k > n/2)Time(k > n/2)\\
&= \frac{1}{2}ET(k \leq n/2)  + \frac{1}{2}ET(k > n/2)\\
&\leq \frac{1}{2}(cn^{2.5} + ET(n/2))  + \frac{1}{2}(cn^{2.5} + ET(n - 1))\\
&= cn^{2.5} + \frac{1}{2}ET(n/2) + \frac{1}{2}ET(n - 1)\\
&\leq cn^{2.5} + \frac{1}{2}ET(n/2) + \frac{1}{2}ET(n)\\
ET(n) - \frac{1}{2}ET(n) &\leq cn^{2.5} + \frac{1}{2}ET(n/2) \\
\frac{1}{2}ET(n) &\leq cn^{2.5} + \frac{1}{2}ET(n/2)\\
ET(n) &\leq 2cn^{2.5} + ET(n/2)\\
&= c_2n^{2.5} + ET(n/2)\\
&= c_2n^{2.5} + \frac{1}{2}c_2n^{2.5} + \frac{1}{2^2}c_2n^{2.5} + \cdots + \frac{1}{2^{log_2(n)}}ET(1)\\
&\leq c_2n^{2.5}\left(1 + \frac{1}{2} + \frac{1}{2^2} + \cdots \right)\\
&= 2c_2n^{2.5} \in O(n^{2.5}). \\
\end{aligned}
\end{equation}
And as a lower bound we take $k = 1$, then the function takes $cn^{2.5}$ time, so $ET(n) \Omega(n^{2.5})$. Hence $ET(n) \in \Theta(n^{2.5})$. \\
\end{enumerate}

\item Steps 1-7 take $cn$ time. 
\begin{enumerate}
\item In the worst case, $k = 1$, so step 8 takes $c_1$ time, and we have:
\begin{equation}
\begin{aligned}
T(n) &= cn + + c_1 + T(n - 1)\\
&\approx cn + T(n - 1)\\
&= cn + c(n - 1) + c(n - 2) + \cdots + T(0)\\
&\leq cn + cn + cn + \cdots + T(0)\\
&=n(cn) \in O(n^2),\\
T(n) &\geq cn + c(n - 1) + c(n - 2) + \cdots + c(n - \frac{n}{2})\\
&\geq c\frac{n}{2} + c\frac{n}{2}  + \cdots + c\frac{n}{2} \\
&= \frac{n}{2}c\frac{n}{2} \in \Omega(n^2). 
\end{aligned}
\end{equation}
So $T(n) \in \Theta(n^2)$. 
\item In the best case, $k = n/2$, so we have:
\begin{equation}
\begin{aligned}
ET(n) &\geq cn + ET(n/2) + ET(n/2)\\
&= cn + 2ET(n/2) \in \Omega(nlog_2(n)). 
\end{aligned}
\end{equation}
We find an upper bound for the expected running time. \\
$ET(n) = Prob(k < n/4)ET(k < n/4) + Prob(k \geq n/4)ET(k \geq n/4)$. \\
$Prob(k < n/4) = Prob(k \geq n/4) = \frac{1}{2}$. \\
$ET(k < n/4) \leq ET(n = 1) = cn + ET(n - 1)$. \\
$ET(k \geq n/4) \leq ET(n = n/4) = cn + ET(n/4) + ET(3n/4)$. \\
Thus:
\begin{equation}
\begin{aligned}
ET(n) &\leq \frac{1}{2}(cn + ET(n - 1)) + \frac{1}{2}(cn + ET(n/4) + ET(3n/4))\\
&\leq cn + \frac{1}{2}ET(n) + \frac{1}{2}(ET(n/4) + ET(3n/4))\\
ET(n) - \frac{1}{2}ET(n) &\leq cn + \frac{1}{2}(ET(n/4) + ET(3n/4))\\
\frac{1}{2}ET(n) &\leq cn + \frac{1}{2}(ET(n/4) + ET(3n/4))\\
ET(n) &\leq c_2n + ET(n/4) + ET(3n/4).\\
\end{aligned}
\end{equation}
Now we use a recursion tree. Each node has two children for each of the two recursive calls. The sum over each level is $cn$, and the height of the tree is the height of the slowest path, so it is $log_{4/3}(n)$, so $ET(n) \leq log_{4/3}(n)cn = cc_3nlog_2(n) \in O(nlog_2(n))$. Hence $ET(n) \in \Theta(nlog_2(n))$. \\
\end{enumerate}
\item Assume array has no duplicates. \\
$ET(0) = 0$. \\
After partition, $A[s] = p$. \textbf{What does this mean??}\\
Let $m = s -i + 1$. \textbf{why??}\\
After partition, $p$ is $m$-th element of $A[i],A[i + 1],...,A[j]$. \\
Split into fourths, treat the two recursive calls like $Func(k),Func(n - k)$. \\
Check MAX's midterm to figure this one out, do we need to know it???\\
Note that this problem is equivalent to a function with a few lines taking $cn$ time, and a random $k$ being chosen between $n/10$ and $9n/10$, with two recursive calls, one being $T(k)$ and the other being $T(n - k)$. This is equivalent to a random $k$ between $1$ and $n$, with recursive calls $T(n/10)$, $T(9n/10)$. \\
Note we only want the worst case!\\\\
In the worst case, choose $k = n/10$, but either extreme works due to symmetry. So we have: 
\begin{equation}
\begin{aligned}
T(n) &= cn + T(n/10) + T(n - n/10)\\
&= cn + T(n/10) + T(9n/10)\\
\end{aligned}
\end{equation}
Using a recursion tree, we see that each level adds $cn$ to the sum. The height of the tree is the worst case which gives us $log_{10/9}(n)$ levels, and thus a time of $cnlog_{10/9}(n)$. The best case gives us $log_{10}(n)$ levels and hence a time of $cnlog_{10}(n)$. So $T(n) \in \Theta(nlog_2(n))$. 







 







\end{enumerate}











\end{document}


