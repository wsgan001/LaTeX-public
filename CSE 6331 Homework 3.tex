

%	options include 12pt or 11pt or 10pt
%	classes include article, report, book, letter, thesis

\title{}



\author{Brendan Whitaker}

\date{AU17}
\documentclass[10pt,oneside,reqno]{amsart}

\usepackage{graphicx}
\usepackage[margin=1in]{geometry}
\usepackage{amsmath}
\usepackage{amssymb}
\usepackage{amsthm}
\usepackage{bbm}
\usepackage{cancel}
\usepackage{verbatim}
\usepackage{amsrefs}
\usepackage{enumitem}
\usepackage{etoolbox}% http://ctan.org/pkg/etoolbox
\patchcmd{\thmhead}{(#3)}{#3}{}{}
\usepackage{braket}
\usepackage{tikz-cd}
\usepackage{adjustbox}


\theoremstyle{plain}
\newtheorem{Thm}{Theorem}
\newtheorem{Cor}[Thm]{Corollary}
\newtheorem{Prop}[Thm]{Proposition}
\newtheorem{Lem}[Thm]{Lemma}
\newtheorem{Prob}[Thm]{Problem}
\newtheorem{Def}[Thm]{Definition}
\newtheorem{Q}[Thm]{Question}
\newtheorem*{e}{Exercise}
\newtheorem{ee}{Exercise}
\theoremstyle{definition}
\newtheorem{Remark}[Thm]{Remark}
\newtheorem{Tech}[Thm]{Technical Remark}
\newtheorem*{Claim}{Claim}
\newtheorem{Ex}[Thm]{Example}




\newcommand{\Mod}[1]{\ (\mathrm{mod}\ #1)}
\newcommand{\norm}{\trianglelefteq}
\newcommand{\propnorm}{\triangleleft}
\newcommand{\semi}{\rtimes}
\newcommand{\sub}{\subseteq}
\newcommand{\fa}{\forall}
\newcommand{\R}{\mathbb{R}}
\newcommand{\z}{\mathbb{Z}}
\newcommand{\n}{\mathbb{N}}

\newcommand{\bee}{\begin{equation}\begin{aligned}}
\newcommand{\eee}{\end{aligned}\end{equation}}
\newcommand{\nequiv}{\not\equiv}
\newcommand{\lc}[2]{#1_1 + \cdots + #1_{#2}}
\newcommand{\lcc}[3]{#1_1 #2_1 + \cdots + #1_{#3} #2_{#3}}
\newcommand{\ten}{\otimes} %tensor product
\newcommand{\fracc}{\frac}



\renewcommand{\rm}{\normalshape}%text inside math
\renewcommand{\Re}{\operatorname{Re}}%real part
\renewcommand{\Im}{\operatorname{Im}}%imaginary part
\renewcommand{\bar}{\overline}%bar (wide version often looks better)



\begin{document}

\title{CSE 6331 Homework 3}

\date{SP18}

\author[Brendan Whitaker]{Brendan Whitaker}

\maketitle

\begin{enumerate}[label=\arabic*.]
\item \textit{A function $T(n)$ satisfies: 
$$
T(n) = 
\begin{cases}
c, &  n \leq 1\\
3T(\lfloor n/4 \rfloor) + n, &  n > 1
\end{cases},
$$
where $c$ is a positive constant. Prove that $T(n)$ is asymptotically nondecreasing. }

\begin{proof}
We use induction. Let $n = 1$. Then 
$$
T(n + 1) = T(2) = 3T(\lfloor 2/4 \rfloor) + 2 = 3T(0) + 2 = 3c + 2 > c.
$$
So $T(1) = c \leq 3c + 2 = T(2)$. So the base case holds. Now fix $n$ and assume $T(k) \leq T(k + 1),\forall k \leq n$. We wish to show that $T(n + 1) \leq T(n + 2)$. We know 
$T(n + 2) = 3T(\lfloor (n + 2)/4 \rfloor) + n + 2$. And $T(n + 1) = 3T(\lfloor (n + 1)/4 \rfloor) + n + 1$. \\
\textbf{Case 1: }$n + 1 \equiv 3 \mod 4$. Then we know $(n + 2)/4$ is an integer, so let $l + 1 = (n + 2)/4$. Then we have: $\lfloor (n + 1)/4 \rfloor = l$, and then $T(n + 2) = 3T(l + 1) + n + 2$, and $T(n + 1) = 3T(l) + n + 1$. Now since $l \leq n$, we know $T(l) \leq T( l + 1)$. So we have: 
$$
T(n + 1) = 3T(l) + n + 1 \leq 3T(l + 1) + n + 2 = T(n  +2).
$$
\textbf{Case 2: }$n + 1 \nequiv 3 \mod 4$. Then $\lfloor (n + 1)/4 \rfloor = \lfloor (n + 2)/4 \rfloor = l \in \z$. So we have: 
$$
T(n + 1) = 3T(l) + n + 1 \leq 3T(l) + n + 2 = T(n  +2).
$$
Hence we know $T(n + 1) \leq T(n + 2)$ in all cases, so by induction this is true for all $n \in \n$. So $T(n)$ is nondecreasing for positive integers. In particular it is asymptotically nondecreasing. 
\end{proof}

\item \textit{Determine the tight asymptotic complexity of the following function:
$$
T(n) = 
\begin{cases}
b, &  n \leq 3\\
T(\lfloor n/2 \rfloor) + T(\lfloor n/4 \rfloor) + cn, &  n > 3
\end{cases}.
$$
}
We claim $T(n) \in \Theta(n)$. 
\begin{proof}
Assume $n$ is a power of $4$. We use a recursion tree to observe that at each level $i$, the time to divide and combine is $(\frac{3}{4})^in$: 


\adjustbox{scale = 0.78,center}{
\begin{tikzcd}
 &  &  &  & cn \arrow[rrd] \arrow[lld] &  &  &  & =cn \\
 &  & cn/2 \arrow[ld] \arrow[rd] &  &  &  & cn/4 \arrow[ld] \arrow[rd, dotted] &  & =(3/4)cn \\
 & cn/4 \arrow[ldd, dotted] &  & cn/8 &  & cn/8 &  & cn/16 & =(3/4)^2cn \\
 & {} \arrow[rrrrrr, no head, dotted] &  &  &  &  &  & {} &  \\
(3/4)^{\log_2n}T(1) &  &  &  &  &  &  &  & =(3/4)^{\log_2n}b
\end{tikzcd}
}


 And since the height of the tree is given by the longest path to a leaf node, which is along the side with all recursive calls to $T(\lfloor n/2 \rfloor )$, we know the height is $\log_2(n)$. So we have: 
\bee
T(n) &= cn + \left(\frac{3}{4} \right)cn
+ \left(\frac{3}{4} \right)^2cn
+ \cdots
+ \left(\frac{3}{4} \right)^{\log_2n}b\\
&\leq 
 cn \left(1
+ \left(\frac{3}{4} \right)
+ \left(\frac{3}{4} \right)^2
+ \cdots \right)\\
&= 4cn \in O(n).\\
T(n) & \geq cn \in \Omega(n).
\eee
So we know $T(n) \in \Theta(n)$. 
\end{proof}


\item \textit{Determine the tight asymptotic complexity of the following function:
$$
T(n) = 
\begin{cases}
b, &  n \leq 3\\
T(\lfloor n/2 \rfloor) + 2T(\lfloor n/4 \rfloor) + cn, &  n > 3
\end{cases}.
$$
}
We claim $T(n) \in \Theta(n\log(n))$. 
\begin{proof}
Again, we use a recurrence tree: 

\adjustbox{scale = 0.8,center}{
\begin{tikzcd}
 &  &  &  & cn \arrow[d] \arrow[llld] \arrow[rrrd] &  &  &  &  & =cn \\
 & cn/2 \arrow[ld] \arrow[d] \arrow[rd] &  &  & cn/4 \arrow[ld] \arrow[d] \arrow[rd] &  &  & cn/4 \arrow[ld] \arrow[d] \arrow[rd] &  & =cn \\
cn/4 & cn/8 & cn/8 & cn/8 & cn/16 & cn/16 & cn/8 & cn/16 & cn/18 & =cn \\
{} \arrow[rrrrrrrr, no head, dashed] &  &  &  &  &  &  &  & {} & =cn \\
T(1) &  &  &  &  &  &  &  &  & =b
\end{tikzcd}
}

So we have that the time to divide and combine on each level of the tree adds to $cn$. And the height of the tree is $\log_2n$ since the longest path to a tree node is along the left hand side, where all the recursive calls are $T(n/2)$. So we have: 
\bee
T(n) &= cn + cn + \cdots + b\\
&= \log_2n(cn) \in \Theta(n\log n). 
\eee
\end{proof}

\item \textit{Use the master method to solve the following recurrences. }

\begin{enumerate}
\item $T(n) = 4T(n/2) + n^2$. 

Note $a = 4,b = 2,f(n) = n^2$, and $n^{\log_2(4)} =n^2 = \Theta(n^2)$. So by the master method, this is \textbf{Case 3}, and hence $T(n) \in Theta(n^2\log n)$. 
\item $T(n) = 4T(n/2) + n^2\log^2n$. 

Note $a = 4,b = 2,f(n) = n^2\log^2n$. And $n^{\log_2(4)} = n^2$, and $n^2\log^2n \in \Theta(n^2\log^2n)$, so this is \textbf{Case 4}. Hence $T(n) \in \Theta(n^2\log^3n)$. 
\item $T(n) = 4T(n/2) + n^3$. 

Note $a = 4,b = 2,f(n) = n^3$. And $n^{\log_2(4)} = n^2$. So we know $f(n) = n^3 >> n^2$, and thus this is \textbf{Case 2}. So we know $T(n) \in \Theta(n^3)$. 
\end{enumerate}
\item \textit{The running time of an algorithm $A$ is described by the recurrence $T(n) = 7T(n/2) + n^2$. A competing algorithm $A'$ has a running time of $T'(n) = aT'(n/4) + n^2$. What is the largest integer value for $a$ such that $A'$ is asymptotically faster than $A$?}

We apply the master theorem. Note that for $T(n)$, we have $a = 7,b = 2,f(n) = n^2$. And since $n^{\log_2(7)} >> n^2$, we are in \textbf{Case 1}, and $T(n) \in \Theta(n^{\log_2(7)})$. For $T'(n)$, we have $a,b = 4,f(n) = n^2$. If we choose $a = 17$, we have $n^{\log_4(17)} >> n^2 = n^{\log_4(16)}$, and since:
$$
 \log_4(17) < 2.05 < 2.80 < \log_2(7),
 $$
 we know we are in \textbf{Case 1} for $T'(n)$ as well, and $T'(n) \in \Theta(n^{\log_4(a)})$. Observe that $\log_4(7^2) = 2\log_4(7) = 2\fracc{\log_2(7)}{\log_2(4)} = \log_2(7)$. So when we let $a = 7^2 = 49$, we have $T'(n) = \Theta(T(n))$. So we let $a = 7^2 - 1 = 48$, and then $T'(n) \in \Theta(n^{\log_4(48)})$, so since $n^{\log_4(48)} << n^{\log_2(7)}$, we know 48 is the largest integer $a$ s.t. $T'(n)$ is asymptotically faster than $T(n)$. 
\end{enumerate}








\end{document}


