

%	options include 12pt or 11pt or 10pt
%	classes include article, report, book, letter, thesis

\title{CSE 2321 Homework 7}



\author{Brendan Whitaker}

\date{AU17}
\documentclass[10pt,oneside,reqno]{amsart}

\usepackage{graphicx}
\usepackage[margin=1in]{geometry}
\usepackage{amsmath}
\usepackage{amssymb}
\usepackage{amsthm}
\usepackage{bbm}
\usepackage{cancel}
\usepackage{verbatim}
\usepackage{amsrefs}
\usepackage{enumitem}


\theoremstyle{plain}
\newtheorem{Thm}{Theorem}
\newtheorem{Cor}[Thm]{Corollary}
\newtheorem{Prop}[Thm]{Proposition}
\newtheorem{Lem}[Thm]{Lemma}
\newtheorem{Prob}[Thm]{Problem}
\newtheorem{Def}[Thm]{Definition}
\newtheorem{Q}[Thm]{Question}
\theoremstyle{definition}
\newtheorem{Remark}[Thm]{Remark}
\newtheorem{Tech}[Thm]{Technical Remark}
\newtheorem*{Claim}{Claim}
\newtheorem{Ex}[Thm]{Example}



\newcommand{\Mod}[1]{\ (\mathrm{mod}\ #1)}



\begin{document}

\title{CSE 2321 Homework 7}

\date{AU17}

\author[Brendan Whitaker]{Brendan Whitaker}

\maketitle

\begin{enumerate}[label=2.]

\item 

\begin{enumerate}

\item 

\begin{equation}
\begin{aligned}
t &= \sum_{i = 1}^{6n^2} \sum_{j = 3}^n \sum_{k = 1}^n c = c\sum_{i = 1}^{6n^2} \sum_{j = 3}^n (n -1 + 1) = c\sum_{i = 1}^{6n^2} \sum_{j = 3}^n n = c\sum_{i = 1}^{6n^2} (n - 3 + 1)n \\
&= (6n^2 - 1 + 1)cn(n - 2) = 6cn^3(n - 2) = 6cn^4 - 12cn^3 \approx cn^4. 
\end{aligned}
\end{equation}

\item 

\begin{equation}
\begin{aligned}
t &= \sum_{i = 1}^{10} \sum_{j = 3}^n \sum_{k = 1}^{n\lfloor log_2n\rfloor} c = c\sum_{i = 1}^{10} \sum_{j = 3}^n nlog_2n 
 = \sum_{i = 1}^{10} cn(n - 2)log_2n \\
 &= 10cn(n - 2)log_2n = 10cn^2log_2n - 20cnlog_2n \approx cn^2log_2n.
\end{aligned}
\end{equation}

\item 

\[t = \sum_{i = n}^{2n} \sum_{10i + 7}^{10i + 21} c = c\sum_{i = n}^{2n}(10i - 10i + 21 - 7 + 1) = c\sum_{i = n}^{2n} 15 = (n + 1)15c \approx cn.\]

\item 

\[t = \sum_{i = 4}^{n^2} \sum_{j = 6}^{3i\lfloor log_2i \rfloor} c = \sum_{i = 4}^{n^2} c(3ilog_2i - 5).\]
We find the upper bound. Substituting $n^2$ for $i$, the inside of the summation becomes $3cn^2log_2(n^2)$. Plugging in, we have
\begin{equation}
\begin{aligned}
t &\leq \sum_{i = 4}^{n^2} c(3n^2log_2(n^2) - 5) = (n^2 -3)c(3n^2log_2(n^2) - 5) \\ &= 3cn^4log_2(n^2) -5cn^2 -9cn^2log_2(n^2) + 15c
 \leq 3cn^4log_2(n^2) + 15c = 6cn^4log_2n + 15c. 
\end{aligned}
\end{equation}
Now we compute the lower bound. We split the summation
\[t = \sum_{i = 4}^{n^2} c(3ilog_2i - 5) = \sum_{i = 4}^{\frac{n^2 - 2}{2}} c(3ilog_2i - 5) + \sum_{i = \frac{n^2}{2}}^{n^2} c(3ilog_2i - 5). \]
We substitute $n^2/2$ for $i$, and the inside of the summation becomes $c(3(\frac{n^2}{2})log_2(\frac{n^2}{2}) - 5)$. Then we have
\begin{equation}
\begin{aligned}
t  &= \sum_{i = 4}^{\frac{n^2 - 2}{2}} c(3ilog_2i - 5) + \sum_{i = \frac{n^2}{2}}^{n^2} c(3ilog_2i - 5) \geq \sum_{i = \frac{n^2}{2}}^{n^2} c(3ilog_2i - 5) \\&\geq c\sum_{i = \frac{n^2}{2}}^{n^2} \frac{3}{2}n^2log_2(n^2/2) - 5 = c(n^2/2 + 1)\left(\frac{3}{2}n^2log_2(n^2/2) - 5 \right)\\
&= c\left(\frac{3}{4}n^4log_2(n^2/2) + \frac{3}{2}n^2log_2(n^2/2) -5n^2/2 - 5\right)\\
&\geq c\left(\frac{3}{4}n^4log_2(n^2/2)  -5n^2/2 - 5\right) = c\left(\frac{3}{4}n^4(log_2(n^2) - 1)  -5n^2/2 - 5\right)\\
&= \frac{3}{2}cn^4log_2n - \frac{3}{4}cn^4 - \frac{5}{2}cn^2 - 5c.
\end{aligned}
\end{equation}
Now note since the dominating term in both the upper bound and lower bound is of the form $kcn^4log_2(n)$ for some constant $k \in \mathbb{R}^{> 0}$, we have that the upper bound and lower bound give us a running time of $t \approx cn^4log_2(n)$. 




\end{enumerate}

\end{enumerate}











\end{document}


