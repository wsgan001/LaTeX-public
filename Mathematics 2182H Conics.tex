
%	options include 12pt or 11pt or 10pt
%	classes include article, report, book, letter, thesis

\title{Honors Calculus II: Chapter 15 Conics}
\author{Brendan Whitaker}
\date{11 January 2016}

\documentclass[10pt]{article}
\usepackage{tikz}
\usepackage{graphicx}
\usepackage{cancel}
\usepackage{amsmath}
\usepackage{lmodern}
\usepackage[T1]{fontenc}


\begin{document}

\maketitle

\part{Introduction}
Greeks - visual, real world oriented. \\
Symmetry: 
\begin{center}
	\includegraphics[scale=1.0]{conic3.png} 
\end{center}
\begin{enumerate}
\item front/back symmetry in P1
\item top/bottom symmetry in P2
\end{enumerate}

\begin{itemize}
\item Plane parallel to P1 gives intersection "hyperbola"
\item Plane parallel to P2 gives intersection "circle"
\item Plane transverse to P1 gives intersection "parabola"
\item Plane transverse to P2 gives intersection "ellipse"
\end{itemize}
Descartes - introduced the coordinate system\\*
\\*
Cartesian plane or 3-space:\\*
\\*
Cone: \[ x^2+y^2=z^2\]\\
Plane: \[ax+by+cz=d\]\\
Intersection: quadratic in 2-space (found by solving intersection of cone and plane equation). 

\section{Point/Line}
Cartesian coordinates in 2-space

Euclidian distance: 
\[d(P1,P2)=\sqrt{(x_{1}-x_{2})^{2}+(y_{1}-y_{2})^{2}}\]
Describe the locus of points P, such that \[e|PL|=|PF|\]
Where \textbf{e} is some constant describing the eccentricity of the conic section, \textbf{L} is the directrix and \textbf{F} is the focus or foci of the conic. \medskip

\subsection{Algebraic Expression of Conics}

Rotate and Translate to Standard Form: \medskip

(Rotations will not be covered in this course)
\medskip
\begin{center}
	\includegraphics[scale=0.7]{conic4.png} 
\end{center}
Note: Uppercase P denotes a point, while lowercase p denotes a length value.\\*
\begin{equation}
\begin{aligned}
|PF|&=\\*
\sqrt{(x-p)^{2}+(y-0)^{2}}&=e|PL|\\*
\sqrt{(x-p)^{2}+(y-0)^{2}}&=e|x+p|\\*
(\sqrt{(x-p)^{2}+(y)^{2}})^{2}&=(e|x+p|)^{2}\\*
(x-p)^{2}+y^{2}&=e^{2}(x^{2}+2xp+p^{2})\\*
x^{2}-2xp+p^{2}+y^2&=e^2(x^2+2xp+p^2)\\*
\end{aligned}
\end{equation}
\subsubsection{Case 1: Parabola (e=1)}

\begin{equation}
\begin{aligned}
x^{2}-2xp+p^{2}+y^2&=(1)^2(x^2+2xp+p^2)\\*
x^{2}-2xp+p^{2}+y^2&=(x^2+2xp+p^2)\\*
\cancel{x^{2}}-2xp+\cancel{p^{2}}+y^2&=\cancel{x^2}+2xp+\cancel{p^2}
\end{aligned}
\end{equation}
\textbf{Equation of a Parabola:}
\[\boxed{y^2=4px}\]
\begin{center}
Standard Graph:\\*
\bigskip
\includegraphics[scale=0.6]{ellipse1.png}\\*
\bigskip
\bigskip
Variations:\\*
\bigskip
\includegraphics[scale=0.2]{ellipse2.png} 
\includegraphics[scale=0.2]{ellipse3.png} 
\includegraphics[scale=0.2]{ellipse4.png}\\*

\end{center}
\subsubsection{Case 2: Ellipse (0<1<e)}
\bigskip
\begin{center}
Same Picture:
\end{center}
\begin{center}

\bigskip
	\includegraphics[scale=0.4]{conic4.png} 
\end{center}
\bigskip
\[x^{2}-2xp+p^{2}+y^2=e^2(x^2+2xp+p^2)\]\\*











\end{document}


