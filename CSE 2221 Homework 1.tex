
%	options include 12pt or 11pt or 10pt
%	classes include article, report, book, letter, thesis

\title{\Huge Brendan Whitaker}



\author{\huge CSE 2221 Homework 1 \linebreak \\\\ \Large Professor Bucci}

\date{January 12th, 2017}
\documentclass[10pt]{article}
\usepackage{tikz}
\usepackage{graphicx}
\usepackage{cancel}
\usepackage{amsmath}


\begin{document}
\maketitle


\paragraph{1. }
"Going over the line" with respect to acceptable collaboration v. academic misconduct would be: 
\begin{itemize}
 \item turning in an assignment from a previous offering of the course
 \item writing up a solution when someone is providing explicit
 \item writing up a solution when someone is providing explicit,specific advice for fixing your errors
 \item accepting or downloading from any source any part of a solution to an assignment which could help in completing your assignment
 \item giving to others or uploading to the internet any parts of a solution to your assignments
  \item any of above transgressions with respect to all individuals or entities outside of your own team when working on an assignment permitting teamwork
 \end{itemize} 
 
\paragraph{2. }
Your final exam grade constitutes 30\% of your course grade, and additionally, you must earn a passing grade on your final exam in order to pass the course. 

\paragraph{3. }
No, your homework grade will not be better if you look up and copy the answers because you will be caught for plagiarism and receive a zero on the assignment in addition to other, more serious repercussions. 
\paragraph{4. }
The two most important benefits of the Java language are that it is safe, and portable. 
\paragraph{5. }
An algorithm is a sequence of steps for solving a problem which is unambiguous, executable, and terminating. 
\paragraph{6. }
The two main categories of errors to be encountered when programming are compile-time(syntax) errors, and run-time(logic) errors. 
\paragraph{7. }
\subparagraph{a. }
Compile-time error. 
\subparagraph{b. }
Compile-time error. 
\subparagraph{c. }
Run-time error. 
\subparagraph{d. }
Run-time error. 
\subparagraph{e. }
You cannot test a program for run-time errors when it has compile-time errors because by definition, run-time errors occur during the execution of the program, and the program cannot execute if it does not compile, which it won't if it has compile-time errors. 
\paragraph{8. }
The print statement \[System.out.println("39 + 3");\] prints the literal text string "39 + 3" since it is enclosed in quotes. The print statement \[System.out.println(39 + 3);\] computes the quantity \[39 + 3 = 40\] and prints "40" since the input is not enclosed in quotes and is interpreted mathematically. The print statement \[System.out.println("39" + 3);\] prints the literal text string "39" since it is enclosed in quotes, and then computes \[+ 3 = 3\] and also prints the solution of this expression, and thus prints "393". 
\paragraph{Additional Question: }
\subparagraph{}API - Application Program Interface. A set of routines, protocols, and tools for building software applications. 
\subparagraph{}CPU - Central Processing Unit. The circuitry within a computer that executes the instructions of a computer program by performing the basic arithmetic and logic operations specified by the instructions. 
\subparagraph{•}IBM - International Business Machines. An American multinational technology company out of New York which had a pivotal role in the infancy of the digital age. 
\subparagraph{}IDE - Integrated Development Environment. A comprehensive software application designed for programming typically containing a source code editor, build automation tools, and a debugger. 
\subparagraph{•}JDK - Java Development Kit. A software development kit aimed at Java developers which is an implementation of some version of the Java platform. 
\subparagraph{•}JVM - Java Virtual Machine. An abstract computing device that enables computers to run a Java program. 
\subparagraph{•}GUI - Graphical User Interface. A user interface which allows a user to interact with electronic devices through graphical icons and visual indicators. 
\subparagraph{•}LAN - Local Area Network. An interconnected computer network with a limited area (e.g. residence network). 
\subparagraph{•}MAC - Media Access Control. A computer network sublayer between the logical link control and the physical layer. 
\subparagraph{•}RSS - Rich Site Summary. A family of web feed formats used to publish frequently updated online information. 
\subparagraph{•}UUT - Unit Under Test. A manufactured product undergoing testing. 
\subparagraph{•}XML - Extensible Markup Language. A common markup language used for encoding a set of rules which is both human and machine-readable. 





\end{document}


