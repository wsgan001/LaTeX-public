

%	options include 12pt or 11pt or 10pt
%	classes include article, report, book, letter, thesis

\title{Bonus Problems}



\author{AU17}

\date{}
\documentclass[10pt]{article}

\usepackage{graphicx}

\usepackage{amsmath}
\usepackage{amssymb}
\usepackage{amsthm}
\usepackage{bbm}
\usepackage{cancel}
\usepackage{verbatim}
\usepackage{amsrefs}

\theoremstyle{plain}
\newtheorem{Thm}{Theorem}
\newtheorem{Cor}[Thm]{Corollary}
\newtheorem{Prop}[Thm]{Proposition}
\newtheorem{Lem}[Thm]{Lemma}
\newtheorem{Prob}[Thm]{Problem}
\newtheorem{Def}[Thm]{Definition}
\newtheorem{Q}[Thm]{Question}
\newtheorem{e}[Thm]{Exercise}

\newcommand{\Mod}[1]{\ (\mathrm{mod}\ #1)}



\begin{document}

\maketitle

\section*{Week 1}

\begin{e}
Give an example of a countable infinite abelian group which doesn not have proper infinite subgroups. 
\end{e}

\begin{e}
Prove that $\sqrt{2}$ is irrational using the fact that $(\sqrt{2} - 1)^n \rightarrow 0$.
\end{e} 

\begin{e}
Suppose you know $\exists c_1,c_1$ s.t. $c_1 < \frac{p_n}{n\log n} < c_n$ $\forall n > 1$. Prove from this that $\sum \frac{1}{p_n} = \infty$. 
\end{e}

\begin{e}
Prove that $\pi(n) \sim \frac{n}{n \log n} \Leftrightarrow p_n \sim n \log n$, where $\pi(n)$ is the number of primes $\leq n$. 
\end{e}

\begin{e}
Prove that any real number $x > 0$ is a limit of the form $x = \lim_{n \to \infty} \frac{k_n^2}{m_n^2}$, where $k_n,m_n \in \mathbb{N}$. 
\end{e}

\begin{e}
Prove that the set of algebraic numbers is countable. 
\end{e}

\begin{e}
Prove that if $x \in (0,1) \subset \mathbb{Q}$, then its simple continued fraction expansion is finite. 
\end{e}

\begin{e}
Find the simple continued fraction expansions for $\sqrt{2}, \sqrt{3}$. 
\end{e}

\begin{e}
Is there some $k \in \mathbb{N}$ s.t. every $n \in \mathbb{N}$ is a sum of $k$ cubes?
\end{e}

\begin{e}
\textbf{The Waring Problem:} Is it true that for any $k \in \mathbb{N}$, $k \geq 2$, there exists $C(k)$ s.t. any $n \in \mathbb{N}$ is a sum of $C(k)$ $k$-th powers of non-negative integers?
\end{e}

\begin{e}
Prove that the set of non-normal integers is uncountable. 
\end{e}

\begin{e}
Prove that Champernowne is normal. 
\end{e}

\begin{e}
If you replace the first index of each square in the Champernowne number with $17$, is it still normal?
\end{e}

\begin{e}
If you replace each square in the Champernowne number with $17$, is it still normal?
\end{e}

\begin{e}
\textbf{Wall's Theorem: } If $(x_n)$ is normal, then $\forall a,b \in \mathbb{N}$ $(x_{an + b}$ is also normal. 
\end{e}

\begin{e}
Prove that every real number $x > 0$ is a limit of the form $x = \lim_{n \to \infty} \frac{p_n}{q_n}$, where $p_n,q_n \in \mathbb{P}$. 
\end{e}

\begin{e}
Find the simple continued fraction expansions for $\sqrt{n^2 + 1}$ $\forall n \in \mathbb{N}$. 
\end{e}












\end{document}


