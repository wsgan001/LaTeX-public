

%	options include 12pt or 11pt or 10pt
%	classes include article, report, book, letter, thesis

\title{}



\author{Brendan Whitaker}


\documentclass[12pt,oneside,reqno]{amsart}

%-------------------------------------
%--------PREAMBLE---------------------

%    Include referenced packages here.
\usepackage{}
\usepackage[margin=1in]{geometry}
\usepackage{graphicx}
\usepackage[margin=1in]{geometry}
\usepackage{amsmath}
\usepackage{amssymb}
\usepackage{amsthm}
\usepackage{bbm}
\usepackage{cancel}
\usepackage{verbatim}
\usepackage{amsrefs}
\usepackage{enumitem}
\usepackage{hyperref}
\usepackage{tikz-cd}
%\usepackage[pdf]{pstricks}
\usepackage{braket}
\usetikzlibrary{cd}
\hypersetup{
     colorlinks   = true,
     citecolor    = red
}
%\usepackage{adjustbox}
\usepackage[ruled,linesnumbered]{algorithm2e}
\usepackage{adjustbox}
\usepackage{changepage}


\let\oldemptyset\emptyset
\let\emptyset\varnothing

\theoremstyle{plain}
\newtheorem{Thm}{Theorem}
\newtheorem{Prob}[Thm]{Problem}
%\theoremstyle{definition}
\newtheorem{Remark}[Thm]{Remark}
\newtheorem{Tech}[Thm]{Technical Remark}
\newtheorem*{Claim}{Claim}
%----------------------------------------
%CHAPTER STUFF
\newtheorem{theorem}{Theorem}%[chapter]
%\numberwithin{section}%{chapter}
%\numberwithin{equation}%{chapter}
%CHAPTER STUFF
%----------------------------------------
\newtheorem{lem}[theorem]{Lemma}
%\newtheorem{Q}[theorem]{Question}
\newtheorem{Prop}[theorem]{Proposition}
\newtheorem{Cor}[theorem]{Corollary}

\theoremstyle{definition}
\newtheorem{e}{Exercise}
\newtheorem{Def}[theorem]{Definition}
\newtheorem{Ex}[theorem]{Example}
\newtheorem{xca}[theorem]{Exercise}



\theoremstyle{remark}
\newtheorem{rem}[theorem]{Remark}


\newcommand{\Mod}[1]{\ (\mathrm{mod}\ #1)}
\newcommand{\norm}{\trianglelefteq}
\newcommand{\propnorm}{\triangleleft}
\newcommand{\semi}{\rtimes}
\newcommand{\sub}{\subseteq}
\newcommand{\fa}{\forall}
\newcommand{\R}{\mathbb{R}}
\newcommand{\z}{\mathbb{Z}}
\newcommand{\n}{\mathbb{N}}
\newcommand{\Q}{\mathbb{Q}}
\renewcommand{\c}{\mathbb{C}}
\newcommand{\bb}{\vspace{3mm}}

\newcommand{\bee}{\begin{equation}\begin{aligned}}
\newcommand{\eee}{\end{aligned}\end{equation}}
\newcommand{\nequiv}{\not\equiv}
\newcommand{\lc}[2]{#1_1 + \cdots + #1_{#2}}
\newcommand{\lcc}[3]{#1_1 #2_1 + \cdots + #1_{#3} #2_{#3}}
\newcommand{\ten}{\otimes} %tensor product
\newcommand{\fracc}{\frac}
\newcommand{\tens}{\otimes}
\newcommand{\lpar}{\left(}
\newcommand{\rpar}{\right)}
\newcommand{\floor}{\lfloor}

\renewcommand{\rm}{\normalshape}%text inside math
\renewcommand{\Re}{\operatorname{Re}}%real part
\renewcommand{\Im}{\operatorname{Im}}%imaginary part
\renewcommand{\bar}{\overline}%bar (wide version often looks better)
\renewcommand{\phi}{\varphi}

\makeatletter
\newenvironment{restoretext}%
    {\@parboxrestore%
     \begin{adjustwidth}{}{\leftmargin}%
    }{\end{adjustwidth}
     }
\makeatother

%---------END-OF-PREAMBLE---------
%---------------------------------







\begin{document}

\title{CSE 6331 Homework 7}

\date{SP18}

\author[Brendan Whitaker]{Brendan Whitaker}

\maketitle



\begin{enumerate}[label=\arabic*.]

\item \textit{Let $A_n = \Set{a_1,...,a_n}$ be a set of distinct coin types (e.g., $a_1 = 50$ cents, $a_2 = 25$ cents, $a_3 = 10$ cents, etc). Note that $a_i$ bay be any positive integer and $a_1 > a_2 > ... a_n$. Each type is available in unlimited quantity. Given $A_n$ and an integer $C> 0$, the coin changing problem is to make up the exact amount $C$ using a minimum total number of coins. }
\bb

\begin{enumerate}
\item \textit{Show that if $a_n \neq 1$, then there exists an $A_n$ and $C$ for which there is no solution to the changing problem. }

\begin{proof}
Let $A_n = \Set{2}$ and let $C = 1$. We have $n = 1$ and $a_n = a_1 = 2 \neq 1$, and there is clearly no solution since no integer multiple of $a_n = 2$ will give us $C = 1$. 
\end{proof}

\bb
\item \textit{Show that if $a_n  = 1$ then there is always a solution. }

\begin{proof}
Note since $C \in \z$, and $a_n = 1$ we just take $C$ coins of type $a_n$, and since we have an unlimited quantity available, we have a solution. 
\end{proof}
\bb

\item \textit{When $a_n = 1$ a greedy method to the problem will make change by using coin types in the order $a_1,a_2,...,a_n$. When coin type $a_i$ is being considered, as many coins of this type as possible will be used. Show that this algorithm doesn't necessarily generate an optimal solution. }

\begin{proof}
Let $C = 8$, and let $A_n = \Set{5,4,1}$. Our greedy algorithm will use one 5 cent coin, then three 1 cent coins. This is 4 total coins. But this is not optimal since we have a better solution using two 4 cent coins. 
\end{proof}

\bb
\item \textit{Prove that if $A_n = \Set{k^{n - 1},k^{n - 2},...,k^0}$ for some $k > 1$, then the above greedy method always yields an optimal solution. 
\textbf{Hint: } Let $X = (x_{n - 1},...,x_1,x_0)$ be the greedy solution and let $Y = (y_{n - 1},...,y_1,y_0)$ be any optimal solution such that:
$$
C = \sum_{i = 0}^{n - 1}x_ik^i = \sum_{i = 0}^{n - 1}y_ik^i.
$$
Show that $x_i = y_i$ for all $0 \leq i \leq n - 1$. Note: $k^m = 1 + \sum_{i = 0}^{m - 1}(k - 1)k^i$. }



\begin{proof}
We prove by induction on $n$. Let $n = 1$. Then $x_0 = y_0 = C$ since $k^0 = 1$. Now fix $n = l \in \n$. Suppose $x_i = y_i$ for all $0 \leq i \leq l - 1$. We prove that for $A_{l + 1} = \Set{k^l,k^{l - 1},...,k^0}$, $x_i = y_i$ for all $0 \leq i \leq l$. Suppose for contradiction that $x_l \neq y_l$. We can't have $x_l < y_l$ since the greedy algorithm uses as many coins as possible of type $k^l$. So we must have $x_l > y_l$. So set $x_l = y_l + j$. But since $k^i|k^l$ for all $i < l$, we know $\sum_{i = 0}^{l - 1} y_i \geq kj + \sum_{i = 0}^{l - 1} x_i$, since the option using the least amount of coins is by representing all $jk^l$ cents (the amount by which the sum of values of the first coin type differs between $X$ and $Y$) as $k^{l - 1}$ cent coins. But then 
\bee
\sum_{i = 0}^l y_i &\geq x_l - j + kj + \sum_{i = 0}^{l - 1}x_i\\
&= (k - 1)j + \sum_{i = 0}^l x_l.
\eee
And thus $Y$ uses more coins than $X$, which is impossible since $Y$ is optimal, so we have a contradiction. Thus we must have that $x_l = y_l$. Then by our induction hypothesis, we have a problem with $C' = C - x_lk^l$ and $l - 1$ coins, for which we know $x_i = y_i$ for all $i \leq l - 1$. Thus $x_i = y_i$ for all $i$ with any $n \in \n$, and so the greedy method always yields an optimal solution. 
\end{proof}

\bb




\end{enumerate}

\item \textit{Let $G = (V,E)$ be an undirected graph. A subset $U \sub V$ is called a \textbf{node cover} if each edge in $E$ is incident upon at least one node in $U$. Finding a minimum node cover for a general graph is NP-hard, but if the graph is a tree, then a minimum node cover can be obtained by the greedy method. Design a greedy algorithm that always generates an optimal solution. (Explain your algorithm in plain English.)}

\bb
\begin{restoretext}
\begin{algorithm}[H]\label{alg1}
\KwData{The tree $T$, set $U$. }
\KwResult{A complete and optimal node cover $U$. }
\Begin{
\textbf{initialize} $C \longleftarrow \text{Children}(v)$\;


\For{$c\in C$}{
	\textbf{initialize} $C' \longleftarrow \text{Children}(c)$\;
	\If{$C' \neq \emptyset$}{
		\textbf{Add} $c$ \KwTo $U$\;
		Node-Cover(c)\;
	}
}
\Return\;
}
\caption{Node-Cover$(v)$}
\end{algorithm}
\end{restoretext}
\bb

We have a global variable $T$ for the tree and a global set $U$ initialized to the empty set, and we assume our program already has access to these. Our initial call is on $v$, the root of $T$. We set $C$ to be the set of children of $v$. For each child, we set $C'$ to be the children of $c$ and if this set is nonempty, we add $c$ to $U$, the node cover, and call Node-Cover$(c)$. Thus it adds every vertex but the leaf nodes, which is exactly the optimal node cover for a tree. 
\bb

\item \textit{Consider the Activity problem discussed in class. Suppose now we want to maximize the \textbf{total sum} of selected intervals, $\sum_{i \in A}(f_i - s_i)$, where $A$ is the set of selected intervals. Solve this problem using any method. Your algorithm must be $O(n^2)$. }

We assume the intervals are sorted by finish time, which takes $O(n\log n)$ time. We also assume all start and finish times are positive and real-valued. Define $F(i)$ to be the maximum total sum of selected intervals with finish time $\geq f_i$. We define:
\bee
F(i) = \max_{i < j \leq n + 1} \begin{cases}

(f_i - s_i) + F(j) & \text{ if }s_j \geq f_i\\
F(i + 1)\\
\end{cases}.
\eee

Our boundary conditions are:
\bee
s_{n + 1} &= \infty\\
F(n + 1) &= 0.
\eee

Our goal is to compute $F(1)$. We give an algorithm to compute our goal:
\bb
\begin{restoretext}
\begin{algorithm}[H]\label{alg1}
\KwData{Arrays of start times and end times $S[1..n + 1],E[1..n]$.}
\KwResult{The array $F[i]$ computed for $1 \leq i \leq n$. }
\Begin{
\textbf{global array} $F[1..n + 1]$\;
\textbf{initialize} $F[n + 1] \longleftarrow 0$\;
\textbf{initialize} $S[n + 1] \longleftarrow \infty$\;

\For{$i \leftarrow n$ \KwTo $1$}{
$F[i] \longleftarrow \max_{j:S[j] \geq F[i]}\Set{E[i] - S[i] + F[j],F[i + 1]}$\;
}
\Return $F$\;
}
\caption{Activity-Compute-$F$}
\end{algorithm}
\end{restoretext}
\bb
And we give an algorithm to print out the set of selected intervals $A$:

\begin{restoretext}
\begin{algorithm}[H]\label{alg2}
\KwData{The computed array $F[1..n+1]$, and the arrays of start times and end times $S[1..n],E[0..n]$, global set $A$. }
\KwResult{Computes the set $A$ of selected intervals. }
\Begin{
\textbf{global set} $A$\;
\tcc{Assume $F[1..n + 1]$ has already been computed. }
\If{$i < n + 1$}{
	\uIf{$F[i] = E[i] - S[i] + F[i + 1]$}{
		\textbf{Add} $i$ \KwTo $A$;
		
		Activity$(i + 1)$;
	} \Else{
		Activity$(i + 1)$;
	}
}
\Return $A$\;
}
\caption{Activity$(i)$}
\end{algorithm}
\end{restoretext}

\bb

The sort takes $n \log n$ time, the computation of $F$ takes $O(n^2)$ time, since the max over $j$ takes at most $O(n)$ time where $i = 1$, and the computation of $A$ from $F$ takes $O(n)$ time. So the whole algorithm takes $O(n^2)$ time. 






















































































\end{enumerate}












\end{document}


