

%	options include 12pt or 11pt or 10pt
%	classes include article, report, book, letter, thesis

\title{}



\author{Brendan Whitaker}


\documentclass[10pt,oneside,reqno]{amsart}

%-------------------------------------
%--------PREAMBLE---------------------

%    Include referenced packages here.
\usepackage{}
\usepackage[margin=1in]{geometry}
\usepackage{graphicx}
\usepackage[margin=1in]{geometry}
\usepackage{amsmath}
\usepackage{amssymb}
\usepackage{amsthm}
\usepackage{bbm}
\usepackage{cancel}
\usepackage{verbatim}
\usepackage{amsrefs}
\usepackage{enumitem}
\usepackage{hyperref}
\usepackage{tikz-cd}
%\usepackage[pdf]{pstricks}
\usepackage{braket}
\usetikzlibrary{cd}
\hypersetup{
     colorlinks   = true,
     citecolor    = red
}
%\usepackage{adjustbox}
\usepackage[ruled,linesnumbered]{algorithm2e}
\usepackage{adjustbox}
\usepackage{changepage}


\let\oldemptyset\emptyset
\let\emptyset\varnothing

\theoremstyle{plain}
\newtheorem{Thm}{Theorem}
\newtheorem{Prob}[Thm]{Problem}
%\theoremstyle{definition}
\newtheorem{Remark}[Thm]{Remark}
\newtheorem{Tech}[Thm]{Technical Remark}
\newtheorem*{Claim}{Claim}
%----------------------------------------
%CHAPTER STUFF
\newtheorem{theorem}{Theorem}%[chapter]
%\numberwithin{section}%{chapter}
%\numberwithin{equation}%{chapter}
%CHAPTER STUFF
%----------------------------------------
\newtheorem{lem}[theorem]{Lemma}
%\newtheorem{Q}[theorem]{Question}
\newtheorem{Prop}[theorem]{Proposition}
\newtheorem{Cor}[theorem]{Corollary}

\theoremstyle{definition}
\newtheorem{e}{Exercise}
\newtheorem{Def}[theorem]{Definition}
\newtheorem{Ex}[theorem]{Example}
\newtheorem{xca}[theorem]{Exercise}



\theoremstyle{remark}
\newtheorem{rem}[theorem]{Remark}


\newcommand{\Mod}[1]{\ (\mathrm{mod}\ #1)}
\newcommand{\norm}{\trianglelefteq}
\newcommand{\propnorm}{\triangleleft}
\newcommand{\semi}{\rtimes}
\newcommand{\sub}{\subseteq}
\newcommand{\fa}{\forall}
\newcommand{\R}{\mathbb{R}}
\newcommand{\z}{\mathbb{Z}}
\newcommand{\n}{\mathbb{N}}
\newcommand{\Q}{\mathbb{Q}}
\renewcommand{\c}{\mathbb{C}}
\newcommand{\bb}{\vspace{3mm}}

\newcommand{\bee}{\begin{equation}\begin{aligned}}
\newcommand{\eee}{\end{aligned}\end{equation}}
\newcommand{\nequiv}{\not\equiv}
\newcommand{\lc}[2]{#1_1 + \cdots + #1_{#2}}
\newcommand{\lcc}[3]{#1_1 #2_1 + \cdots + #1_{#3} #2_{#3}}
\newcommand{\ten}{\otimes} %tensor product
\newcommand{\fracc}{\frac}
\newcommand{\tens}{\otimes}
\newcommand{\lpar}{\left(}
\newcommand{\rpar}{\right)}
\newcommand{\floor}{\lfloor}

\renewcommand{\rm}{\normalshape}%text inside math
\renewcommand{\Re}{\operatorname{Re}}%real part
\renewcommand{\Im}{\operatorname{Im}}%imaginary part
\renewcommand{\bar}{\overline}%bar (wide version often looks better)
\renewcommand{\phi}{\varphi}

\makeatletter
\newenvironment{restoretext}%
    {\@parboxrestore%
     \begin{adjustwidth}{}{\leftmargin}%
    }{\end{adjustwidth}
     }
\makeatother

%---------END-OF-PREAMBLE---------
%---------------------------------







\begin{document}

\title{CSE 6331 Homework 6}

\date{SP18}

\author[Brendan Whitaker]{Brendan Whitaker}

\maketitle



\begin{enumerate}[label=\arabic*.]

\item We wish to compute the weight of the minimum weight sequence of moves by which a pawn can go from first to last row. Let $i$ the the index of the current row of the pawn. Then define $f(i,j)$ the be the weight of the minimum weight sequence from position $(i,j)$ to row $m$. Then we have the recurrence relation:
\bee
f(i,j) = \text{weight}(i,j) + \min \begin{cases}
f(i + 1,j - 1)\\
f(i + 1,j)\\
f(i + 1,j + 1)
\end{cases}.
\eee
where $1 \leq i,j \leq m$. And our boundary conditions are:
\bee
f(m + 1,j) &= 0\\
f(i,0) &= \infty\\
f(i,m + 1) &= \infty.
\eee
Our goal is to compute $f(1,1)$. We give a non-recursive algorithm to compute $f(1,1)$ by the array $F[1..m + 1,0..m + 1]$:
\bb
\begin{restoretext}
\begin{algorithm}[H]\label{alg1}
\KwData{The weight matrix $Weight[1..m,1..m]$. }
\KwResult{The array $F[i,j]$ computed for $1 \leq i,j \leq m$. }
\Begin{
\textbf{global array} $F[1..m + 1,0..m + 1]$\;
\textbf{initialize} $F[i, m + 1] \longleftarrow 0,F[i,0] \longleftarrow 0$ for $1 \leq i \leq m + 1$;\\
\textbf{initialize} $F[m + 1,j] \longleftarrow 0$ for $0 \leq j \leq m + 1$;

\For{$i \leftarrow m$ \KwTo $1$}{
	\For{$j \leftarrow m$ \KwTo $1$}{
		$F[i,j] \longleftarrow Weight[i,j] + \min\Set{F[i + 1,j - 1],F[i + 1,j],F[i + 1,j + 1]}$;
		
	
	}
}
}
\caption{Chessboard-Array-$F$}
\end{algorithm}
\end{restoretext}
The asymptotic running time is $\Theta(n^2)$. The problem statement says it is not necessary to actually print the sequence, but if we wanted to, we would simply use a while loop, saying while $i \neq m + 1$ starting at $i = 1$ and initializing $j = 1$, print $(i,j)$, then compute $F[i,j] - Weight[i,j]$ and then we know that one of the elements of $\Set{F[i + 1,j - 1],F[i + 1,j],F[i + 1,j + 1]}$ must be equal in value to this quantity. So we use an if-elseif-else statement to find the first of these which has the desired value, and then update $i,j$ values to match the chosen next move. 






\bb






\item We wish to compute the minimum number of operations needed to transform $A = a_1\cdots a_m$ into $B = b_1\cdots b_n$. We let $X_i$ be the decision to delete/add/change/not change the character $a_i$ such that it matches $b_i$, where we have at most $\max\Set{m,n}$ such decisions. Then let $f(i,j)$ be the minimum number of operations needed to transform $A_i = a_i\cdots a_m$ into $B_i = b_i \cdots b_n$. Then we have:
\bee
f(i,j) =  \begin{cases}
1 + \min\begin{cases}
f(i + 1,j) & (\text{delete } a_i)\\
f(i, j + 1) & (\text{add new }a_i)\\
f(i + 1,j + 1) & (\text{replace } a_i \text{ with } b_j)\\
\end{cases} & \text{ if } a_i \neq b_j\\
f(i + 1,j + 1) & \text{ if } a_i = b_j
\end{cases},
\eee
where $1 \leq i \leq m$ and $1 \leq j \leq m$. And our boundary conditions are:
\bee
f(n + 1,m + 1) = 0,\\
f(i,n + 1) = 1,\\
f(m + 1,j) = 1,
\eee
where $1 \leq i \leq m$ and $1 \leq j \leq n$. 
Our goal is to compute $f(1,1)$. We represent the values of $f(i,j)$ in an array $F[1..m + 1,1..n + 1]$, which we compute in the following algorithm:


\begin{restoretext}
\begin{algorithm}[H]\label{alg2}
\KwData{Two strings $A = a_1\cdots a_m$ and $B = b_1\cdots b_n$. }
\KwResult{The array $F[i,j]$ computed for $1 \leq i \leq m, 1 \leq j \leq n$. }
\Begin{
\textbf{global array} $F[1..m + 1,1..n + 1]$\;
\textbf{initialize} $F[m + 1, n + 1] \longleftarrow 0$;\\
\textbf{initialize} $F[i, n + 1] \longleftarrow 1$ for $1 \leq i \leq m$;\\
\textbf{initialize} $F[m + 1, j] \longleftarrow 1$ for $1 \leq j \leq n$;\\

\For{$i \leftarrow m$ \KwTo $1$}{
	\For{$j \leftarrow n$ \KwTo $1$}{
		\uIf{$a_i = b_j$}{
			$F[i,j] \longleftarrow F[i + 1,j + 1]$;
		}\Else{
			$F[i,j] \longleftarrow 1 + \min\Set{F[i + 1,j],F[i,j + 1]F[i + 1,j + 1]}$;
		
		}
	}
}
}
\caption{String-Array-$F$}
\end{algorithm}
\end{restoretext}
The asymptotic running time is $\Theta(mn)$. The problem statement does not ask us to print out the actual number of operations, but if we wanted to, we would loop while $i \neq m  + 1$ and $j \neq n + 1$ starting from $i = j = 1$, and check if $a_i = a_j$. If it is, we increment both $i,j$ and go to next iteration, otherwise, we increment our operation counter, and find the minimum of $\Set{F[i + 1,j],F[i,j + 1]F[i + 1,j + 1]}$ and set $i,j$ to its indices, then go to next iteration. 





\bb



\item We have a set $J$ of $N$ jobs with corresponding processing time $t_i$ for each, rewards $p_i$ for finishing by time $T$, and penalties $q_i$ for not finishing by $T$. We wish to compute a subset of $S \sub J$ s.t.:
\bee
\sum_{i \in S} t_i &\leq T,
\eee
and we maximize:
\bee
f(S) = \sum_{i \in S}p_i - \sum_{i \notin S}q_i.
\eee
So let the $i$-th subproblem denote our decision of whether or not to include job $i$ in $S$. So define $g(i,t)$ to be the maximum profit of any subset $S_i$ of jobs in $J$ with index $\geq i$ s.t. $\sum_{k \in S_i} t_k \leq t$. We give a recurrence relation:
\bee
g(i,t) = \begin{cases}
\max \begin{cases}
p_i + g(i + 1,t + t_i)\\
-q_i + g(i + 1,t)
\end{cases} & \text{ if }t < T\\
g(i,t) = -q_i + g(i + 1,t)  & \text{ if }t \geq T
\end{cases}. 
\eee
where $1 \leq i \leq N$. Our boundary condition is:
\bee
g(N + 1,t) &= 0.
\eee

Our goal is to compute $g(1,0)$. We give a non-recursive algorithm:

Define $t' = t - t_i$. 
\begin{restoretext}
\begin{algorithm}[H]\label{alg3}
\KwData{The set $J$ of $N$ jobs, sets $P,Q$ of rewards and penalties for each job, and the set $\mathcal{T}$ of times $t_i$ for each job. }
\KwResult{The array $G[i,t]$ computed for $1 \leq i \leq N, 1 \leq j \leq T$. }
\Begin{
\textbf{global array} $G[1..N + 1,1..T]$\;
\textbf{initialize} $G[N + 1, t] \longleftarrow 0$;\\

\For{$i \leftarrow N$ \KwTo $1$}{
	
	\For{$t \leftarrow T$ \KwTo $0$}{
		\textbf{int} $t' \longleftarrow t + t_i$;\\
		\uIf{$t' \leq T$}{
			$G[i,t] \longleftarrow \max\Set{P[i] + G[i + 1,t'],-Q[i] + G[i + 1,t]}$;\\
		}\Else{
			$G[i,t] \longleftarrow -Q[i] + G[i + 1,t]$;\\
		
		}
	}
}
}
\caption{Job-Array-$G$}
\end{algorithm}
\end{restoretext}
The running time is $O(NT)$. 





























\end{enumerate}












\end{document}


