

%	options include 12pt or 11pt or 10pt
%	classes include article, report, book, letter, thesis

\title{CSE 2321 Homework 10}



\author{Brendan Whitaker}

\date{AU17}
\documentclass[10pt,oneside,reqno]{amsart}

\usepackage{graphicx}
\usepackage[margin=1in]{geometry}
\usepackage{amsmath}
\usepackage{amssymb}
\usepackage{amsthm}
\usepackage{bbm}
\usepackage{cancel}
\usepackage{verbatim}
\usepackage{amsrefs}
\usepackage{enumitem}


\theoremstyle{plain}
\newtheorem{Thm}{Theorem}
\newtheorem{Cor}[Thm]{Corollary}
\newtheorem{Prop}[Thm]{Proposition}
\newtheorem{Lem}[Thm]{Lemma}
\newtheorem{Prob}[Thm]{Problem}
\newtheorem{Def}[Thm]{Definition}
\newtheorem{Q}[Thm]{Question}
\theoremstyle{definition}
\newtheorem{Remark}[Thm]{Remark}
\newtheorem{Tech}[Thm]{Technical Remark}
\newtheorem*{Claim}{Claim}
\newtheorem{Ex}[Thm]{Example}



\newcommand{\Mod}[1]{\ (\mathrm{mod}\ #1)}



\begin{document}

\title{CSE 2321 Homework 10}

\date{AU17}

\author[Brendan Whitaker]{Brendan Whitaker}

\maketitle

\begin{enumerate}[label=\arabic*.]

\item 

\begin{enumerate}

\item $f(n) = 4096$; $g(n) = log_2(2048)$. Let $c_1 = c_2 = \frac{4096}{log_2(2048)}$. Then $c_1g(n) = c_2g(n) = 4096$. Hence $c_1g(n) \leq f(n) \leq c_2g(n)$ $\forall n \in \mathbb{N}$ since $f(n) = 4096$, hence $f(n) \in \Theta(g(n))$. \\

\item $f(n) = log_2(n^2)$; $g(n) = n^2log_2(n)$. Note $f(n) = 2log_2(n)$. Let $c = 2$. Then note that $f(n) \leq cg(n)$ $\forall n \in \mathbb{N}$, since $2log_2(n) \leq 2n^2log_2(n)$ $ \forall n \in \mathbb{N}$. We prove this by induction. \\

\begin{proof}
Let $n = 1$. Then $2log_2(1) = 0 \leq 0 = 2\cdot1^2log_2(1)$. So the base case holds. Now assume for some fixed $k \in \mathbb{N}$ that $2log_2(k) \leq 2k^2log_2(k)$. Then we must have that $1 \leq k^2$. Then note that $2log_2(k + 1) \leq 2(k + 1)^2log_2(k + 1) \Leftrightarrow 1 \leq (k + 1)^2 = k^2 + 2k + 1$, but we know that $1 \leq k^2$, and $2k + 1 \geq 1$, so this must be true. Hence the claim holds.  
\end{proof}
Thus $f(n) \in O(g(n))$ by definition. \\
\item $f(n) = 3^{n  +1}$; $g(n) = 3^n$. Note that $f(n) = 3^{n  +1} = 3\cdot 3^n = 3g(n)$. So since $g(n) \leq 3g(n) = f(n) \leq 6g(n)$ $\forall n \in \mathbb{N}$, we have that $f(n) \in \Theta(g(n))$. \\
\item $f(n) = 2^n$; $g(n) = 4^n$. Note $g(n) = 4^n = (2^2)^n = 2^{2n} = (2^n)^2 = (f(n))^2$. So $x \leq x^2$ for all positive values of $x$ and $f(n) = 2^n$ is positive for all $n \in \mathbb{N}$, we must have that $f(n) \leq (f(n))^2  = g(n)$ $\forall n \in \mathbb{N} \Rightarrow f(n) \in O(g(n))$ by definition. \\

\end{enumerate}

\item 

\begin{enumerate}

\item \begin{equation}
\begin{aligned}
T(n) &= c + 3T(n/3)\\
&= c +3c + 3^2T(n/3^2)\\
&= c+ 3c + 3^2c + 3^3T(n/3^3)\\
&= c + 3c + 3^2c + 3^3c + \cdots + 3^{log_3(n) - 1}c +  3^{log_3(n)}T(1)\\
&= c + 3c + 3^2c + 3^3c + \cdots + 3^{log_3(n) - 1}c +  c'n\\
&= c  + \sum_{i = 1}^{\lfloor log_3(n)\rfloor - 1} 3^ic \\
&= c + 3\frac{1-3^{log_3(n) - 1}}{1 - 3}\\
&= c + \frac{n -3}{2} \in \Theta(n). 
\end{aligned}
\end{equation}

\item 

\begin{equation}
\begin{aligned}
T(n) &= cn^2 + T(n - 1) \\
&= cn^2 + c(n - 1)^2 + T(n - 2) \\
&= cn^2 + c(n - 1)^2 + c(n - 2)^2 + T(n - 3)\\
&= cn^2 + c(n - 1)^2 + c(n - 2)^2 + \cdots + c(n - (n - 1))^2 + T(0)\\
&= c + c(2)^2 + c(3)^2 + \cdots + c(n - 2)^2+ c(n - 1)^2 + cn^2 + c'\\
&= c' + c\sum_{i = 1}^n i^2\\
&= c' + c \left(\sum_{i = 1}^n (i^2)\right)\\
&= c' + c \left(\frac{n(2n + 1)(n + 1)}{6} \right)\\
&= c' +  \frac{2cn^3 + 3cn^2 + cn}{6} \in \Theta(n^3). 
\end{aligned}
\end{equation}

\item 

\begin{equation}
\begin{aligned}
T(n) &= clog_2(n) + T(n - 1) + T(n - 4)\\
&\geq clog_2(n) + 2T(n - 4)\\
&= clog_2(n) + 2clog_2(n - 4) + 2^2T(n - 8)\\
&= clog_2(n) + 2clog_2(n - 4) + 2^2clog_2(n - 8) + \cdots+ 2^{\frac{n}{4} - 1}2c +  2^{\frac{n}{4}}T(n - 4\frac{n}{4})\\
&\geq 2^{\frac{n}{4}}T(0) \\ c'2^{\frac{n}{4}} \in \Omega(2^{\frac{n}{4}}). 
\end{aligned}
\end{equation}
So since $T(n) \in \Omega(2^{\frac{n}{4}})$, the running time has an exponential lower bound. \\

\end{enumerate}

\item .\\\\\\\\\\\\\\\\\\\\\\\\\\\\\\\\\\\\\\\\\\\\\\\\\\\\\\\\\\\\\\\\\\\\\\\\\\\\\\\\\\\\\\\\\\\\\\\\\\\\\\\\\\\\\\\\\\\\\\\\\\\\\\\\\\\\\\\\\\\\\\\\\\\\\\\\\\\

\item The inner for loop executes $n - i$ times and the outer for loop executes $n$ times. Let the running time for the two for loops be given by $F(n)$. Then for some constant $c$, we have:
\begin{equation}
\begin{aligned}
F(n) &= \sum_{i = 1}^n \sum_{j = i}^{n  -1} c = c\sum_{i = 1}^n (n - i) = c\left(\sum_{i = 1}^nn - \sum_{i = 1}^ni \right)\\
&= c\left(n^2 - \frac{n^2 + n}{2} \right) \in \Theta(n^2).  
\end{aligned}
\end{equation}
Thus, for some constant $k$, the recurrence relation is: 
\begin{equation}
\begin{aligned}
T(n) &= kn^2 + T(n - 5) \\
&= kn^2 + k(n - 5)^2 + T(n - 2\cdot 5) \\
&= kn^2 + k(n - 5)^2 + k(n - 10)^2 + T(n - 3 \cdot 5)\\
&= kn^2 + k(n - 5)^2 + k(n - 10)^2 + \cdots +k(n - (\frac{n}{5} - 1)5)^2 +  T(n - \frac{n}{5}5)\\
&= kn^2 + k(n - 5)^2 + k(n - 10)^2 + \cdots +k(5)^2 +  T(0)\\
&= T(0) + \sum_{i = 1}^{\lfloor n/5 \rfloor} k(5i)^2\\
&= T(0) + 25k \frac{(n/5)((n/5) + 1)(2(n/5) + 1)}{6} \in \Theta(n^3). 
\end{aligned}
\end{equation}







 







\end{enumerate}











\end{document}


