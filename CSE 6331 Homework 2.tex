

%	options include 12pt or 11pt or 10pt
%	classes include article, report, book, letter, thesis

\title{Math 5590H Homework 6}



\author{Brendan Whitaker}

\date{AU17}
\documentclass[10pt,oneside,reqno]{amsart}

\usepackage{graphicx}
\usepackage[margin=1in]{geometry}
\usepackage{amsmath}
\usepackage{amssymb}
\usepackage{amsthm}
\usepackage{bbm}
\usepackage{cancel}
\usepackage{verbatim}
\usepackage{amsrefs}
\usepackage{enumitem}
\usepackage{etoolbox}% http://ctan.org/pkg/etoolbox
\patchcmd{\thmhead}{(#3)}{#3}{}{}
\usepackage{braket}


\theoremstyle{plain}
\newtheorem{Thm}{Theorem}
\newtheorem{Cor}[Thm]{Corollary}
\newtheorem{Prop}[Thm]{Proposition}
\newtheorem{Lem}[Thm]{Lemma}
\newtheorem{Prob}[Thm]{Problem}
\newtheorem{Def}[Thm]{Definition}
\newtheorem{Q}[Thm]{Question}
\newtheorem*{e}{Exercise}
\newtheorem{ee}{Exercise}
\theoremstyle{definition}
\newtheorem{Remark}[Thm]{Remark}
\newtheorem{Tech}[Thm]{Technical Remark}
\newtheorem*{Claim}{Claim}
\newtheorem{Ex}[Thm]{Example}




\newcommand{\Mod}[1]{\ (\mathrm{mod}\ #1)}
\newcommand{\norm}{\trianglelefteq}
\newcommand{\propnorm}{\triangleleft}
\newcommand{\semi}{\rtimes}
\newcommand{\sub}{\subseteq}
\newcommand{\fa}{\forall}
\newcommand{\R}{\mathbb{R}}
\newcommand{\z}{\mathbb{Z}}
\newcommand{\n}{\mathbb{N}}
\newcommand{\bee}{\begin{equation}\begin{aligned}}
\newcommand{\eee}{\end{aligned}\end{equation}}



\begin{document}

\title{CSE 6331 Homework 2}

\date{SP18}

\author[Brendan Whitaker]{Brendan Whitaker}

\maketitle
\setcounter{Thm}{4}
\begin{Thm}
If $T(n)$ is asymptotically nondecreasing and $f(n)$ is smooth, then $T(n) = O(f(n)|n$ is a power of $b)$ implies $T(n) = O(f(n))$. 
\end{Thm}
\begin{enumerate}[label=\arabic*.]
\item \textit{Show that Theorem 5 would not hold if $T(n)$ is not asymptotically nondecreasing. }


Let $f(n) = n^2$, and let $b = 2$. We define: 
$$
T(n) = 
\begin{cases}
n^2 & n = 2^k, k \in \mathbb{N}\\
n^n & otherwise
\end{cases}. 
$$
Observe that $f(n) = n^2$ is smooth: $f(2n) = (2n)^2 = 4n^2 \leq 4 \cdot n^2,\forall n \in \mathbb{N}$, so $f(2n) \in O(n^2)$. Also $T(n) \in O(n^2|n$ is a power of $2)$, since $T(n) = n^2$ when $n$ is a power of $2$. We show $T(n) \notin O(n^2)$. Suppose for contradiction that there was a positive constant $c \in \mathbb{N}$ and $N \in \mathbb{N}$ s.t. $\forall n > N$, $T(n) \leq cn^2$. Then this must hold for all powers of $3$, so we restrict ourselves to the case when $n \in B = \{n: n= 3^k,k \in \mathbb{N}\}$. Then $n$ is not a power of $2$, so $T(n) = n^n$. So we have $n^n \leq cn^2,\forall n > N,n \in B$. Then we would have: 
$$
n^2 \cdot n^{n - 2} \leq cn^2,
$$
$$
n^{n - 2} \leq c,
$$
for all $n > N,n \in B$, which is a contradiction, since $c$ is a constant. So we must have $T(n) \notin O(n^2)$. \\


\item \textit{Show that Theorem 5 would not hold if $f(n)$ is nondecreasing but not smooth (even if $T(n)$ is asymptotically nondecreasing). }

Let $f(n) = 2^n$. Let:
$$
T(n) = 2^{2^{\lceil log_2n \rceil}}.
$$
When $n$ is a power of $2$, $n = 2^k$, we have $\lceil log_2n \rceil= k$, so $T(n) = 2^n$, so $T(n) \in O(2^n)$ when $n$ is a power of $2$. But is $T(n) \in O(2^n)$ in general? Let $n = 2^k + 1$, so $n$ runs through the positive integers which are one more than a power of 2. Then for $k \in \mathbb{N}$ we have $T(n) = 2^{2^{k + 1}} = 2^{2 \cdot n} \notin O(2^n)$, so Theorem 5 does not hold in this case. 

\item \textit{Prove \textbf{Theorem 6:} If $T(n)$ is asymptotically nondecreasing and $f(n)$ is smooth, then $T(n) = \Omega(f(n)|n$ is a power of $b)$ implies $T(n) = \Omega(f(n))$. }

\begin{proof}
We know $T(n) \geq c_1f(n)$ for $n$ sufficiently large and a power of $b$. And by definition of smooth, we know $f(bn) \in O(f(n)) \Rightarrow f(n) \in \Omega (f(bn))$. So we have $f(n) \geq c_2f(bn)$ for sufficiently large $n$, and $\forall n$ there is $k \in \mathbb{N}$ s.t. $b^k \leq n < b^{k + 1}$. When $n$ is sufficiently large, we have: 
$$
T(n) \geq T(b^k) \geq c_1f(b^k)  \geq c_1c_2f(b\cdot b^k) \geq c_1c_2f(b^{k + 1}) \geq c_1c_2f(n).
$$
Thus we know $T(n) \in \Omega(f(n))$. 
\end{proof}
\end{enumerate}








\end{document}


