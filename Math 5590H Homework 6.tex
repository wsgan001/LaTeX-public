

%	options include 12pt or 11pt or 10pt
%	classes include article, report, book, letter, thesis

\title{Math 5590H Homework 6}



\author{Brendan Whitaker}

\date{AU17}
\documentclass[10pt,oneside,reqno]{amsart}

\usepackage{graphicx}
\usepackage[margin=1in]{geometry}
\usepackage{amsmath}
\usepackage{amssymb}
\usepackage{amsthm}
\usepackage{bbm}
\usepackage{cancel}
\usepackage{verbatim}
\usepackage{amsrefs}
\usepackage{enumitem}
\usepackage{etoolbox}% http://ctan.org/pkg/etoolbox
\patchcmd{\thmhead}{(#3)}{#3}{}{}
\usepackage{braket}


\theoremstyle{plain}
\newtheorem{Thm}{Theorem}
\newtheorem{Cor}[Thm]{Corollary}
\newtheorem{Prop}[Thm]{Proposition}
\newtheorem{Lem}[Thm]{Lemma}
\newtheorem{Prob}[Thm]{Problem}
\newtheorem{Def}[Thm]{Definition}
\newtheorem{Q}[Thm]{Question}
\newtheorem*{e}{Exercise}
\newtheorem{ee}{Exercise}
\theoremstyle{definition}
\newtheorem{Remark}[Thm]{Remark}
\newtheorem{Tech}[Thm]{Technical Remark}
\newtheorem*{Claim}{Claim}
\newtheorem{Ex}[Thm]{Example}




\newcommand{\Mod}[1]{\ (\mathrm{mod}\ #1)}
\newcommand{\norm}{\trianglelefteq}
\newcommand{\propnorm}{\triangleleft}



\begin{document}

\title{Math 5590H Homework 6}

\date{AU17}

\author[Brendan Whitaker]{Brendan Whitaker}

\maketitle


\begin{Prop}
If $|x| = n < \infty$ then $|x^a| = \frac{n}{(n,a)}$. 
\end{Prop}

\begin{proof}
Write $n = (n,a)b$, $a = (n,a)c$. Then $(b,c) = 1$. We wish to show $|x^a| = b$, since $b = \frac{n}{(n,a)}$. Note
\[(x^a)^b = x^{ab} = x^{(n,a)cb} = x^{(n,a)bc} = x^{nc} = 1. \]
Since $|\langle x^a \rangle| = |x^a|$, we have that $|x^a| \mid b$. Let $|x^a| = k$. Th.en since $x^{ak} = 1$, we must have that $n \mid ak$ since $|\langle x \rangle| = n$. So $(n,a)b \mid (n,a)ck$. Thus $b \mid ck$. Then since $(b,c) = 1$, we must have that $b \mid k$, and so since $k \mid b$ we have that $b = k$. 
\end{proof}
\begin{Prop}
If $G$ is a finite, abelian group, and $p$ is a prime dividing $|G|$, the $G$ contains an element of order $p$. 
\end{Prop}

\begin{proof}
We induct on the order of $G$. We know $G > 1$, else no primes divide $|G|$. Then we have some non-identity element $x$. Let $p$ be a prime dividing the order of $G$. We assume $|G| \neq p$, else the result is trivial by Lagrange. Assume $|G| > p$. 

Suppose $p \mid |x|$, then write $|x| = pn$ for some $n \in \mathbb{Z}$. Then $|x^n| = \frac{pn}{(pn,n} = \frac{pn}{n} = p$. Then we again have an element of order $p$. 

So assume $p \nmid |x|$. Let $N = \langle x \rangle $, and $N$ is normal in $G$ since $G$ abelian. And $|G/N| = \frac{|G|}{|N|}$ by Lagrange. Since $N \neq 1$, we also have that $|G/N| < |G|$. Since $|N||G/N| = |G|$, $p \nmid |N| = |\langle x\rangle |$, and $p \mid |G|$, we must have that $p \mid |G/N|$. Since $N$ is proper, the order of the factor group $G/N$ must be less than $|G|$, so we have that there exists an element $\bar{y} = yN$ of $G/N$ with order $p$. Now $yN \neq N$, and $y^pN = N$, so suppose $\langle y^p \rangle = \langle y \rangle$, then $\exists k \in \mathbb{Z}$ s.t. $y = (y^p)^k$, so $y \in N$ by closure of $\langle y^p \rangle$. But this is a contradiction, so we must have $|\langle y^p \rangle| \neq |\langle y \rangle|$.  And we cannot have that $|\langle y^p \rangle| > |\langle y \rangle|$ since $y^p$ is generated by $y$. So we have that $|\langle y^p \rangle| < |\langle y \rangle|$. So $|y^p| < |y|$, and we have $|y^p| = \frac{|y|}{(|y|,p)}$ by Proposition 1, so $(|y|,p) > 1 \Rightarrow p \mid |y|$. But this situation was treated in the previous paragraph, so we have an element of order $p$. 
\end{proof}

\begin{e}[\textbf{3.4.2}]
Exhibit all $3$ composition series for $Q_8$ and all $7$ composition series for $D_8$. List the composition factors in each case. 
\end{e}
Composition series for $D_8$: 
\begin{itemize}
\item $1 \norm \braket{s} \norm \braket{s,r^2} \norm D_8$
\item $1 \norm \braket{r^2} \norm \braket{s,r^2} \norm D_8$
\item $1 \norm \braket{sr^2} \norm \braket{s,r^2} \norm D_8$
\item $1 \norm \braket{r^2} \norm \braket{r} \norm D_8$

\item $1 \norm \braket{sr} \norm \braket{sr, r^2} \norm D_8$
\item $1 \norm \braket{sr^3} \norm \braket{sr, r^2} \norm D_8$
\item $1 \norm \braket{r^2} \norm \braket{sr, r^2} \norm D_8$

\end{itemize}
The composition factors are all isomorphic to $\mathbb{Z}_2$. 

Composition series for $Q_8$: 

\begin{itemize}
\item $1 \norm \braket{-1} \norm \braket{i} \norm Q_8$
\item $1 \norm \braket{-1} \norm \braket{j} \norm Q_8$
\item $1 \norm \braket{-1} \norm \braket{k} \norm Q_8$
\end{itemize}
The composition factors are all isomorphic to $\mathbb{Z}_2$. \\\\

\begin{e}[\textbf{3.4.5}]
Prove that subgroups and quotient groups of a solvable group are solvable. 
\end{e}

\begin{proof}
We first prove that the subgroups are solvable.
Let $G$ be solvable. Then there is a chain of subgroups
\[1 = G_0 \norm G_1 \norm G_2 \norm \cdots \norm G_s = G,\]
such that $G_{i + 1} / G_i$ is abelian for all $i$. 

Let $H < G$, and let $H_i = G_i \cap H$. Then let $x \in H_i$, and $y \in H_{i + 1}$. Then $yxy^{-1} \in H$ since $x,y \in H$, and $yxy^{-1} \in H_i$ since $G_i \propnorm G_{i  +1}$. Thus $H_i \propnorm H_{i + 1}$. So we have
\[1 = H_0 \norm H_1 \norm H_2 \norm \cdots \norm H_s = H.\]
Since $G_i \propnorm G_{i  +1}$, by the second isomorphism theorem, we have
\[\frac{H_{i + 1}}{H_i} = \frac{H_{i + 1}}{H_{i + 1} \cap G_i} \cong \frac{H_{i + 1}G_i}{G_i} \leq \frac{G_{i + 1}}{G_i}\]
since $H_{i + 1} \leq G_{i + 1}$ and $G_i \leq G_{i + 1} \Rightarrow H_{i + 1}G_i \leq G_i$ by closure of $G_i$. And $ \frac{G_{i + 1}}{G_i}$ is abelian, so $\frac{H_{i + 1}}{H_i}$ is abelian. And thus $H$ is solvable by definition. 

Next we prove that the quotient groups are solvable. Let $N \norm G$. Then consider the quotient groups $G_iN/N$. We know
 \[G_0N/N = 1N/N = N/N = 1,\]
  and 
  \[G_sN/N = GN/N = G/N.\] We prove that $G_iN/N \propnorm G_{i + 1}N/N$. Let 
  $$\bar{y} = ynN = yN \in G_{i  +1}N/N,$$
   and let 
   $$\bar{x} = xnN = xN \in G_iN/N,$$
    s.t. $y \in G_{i + 1}$ and $x \in G_i$. 
    Then we have
    $$\bar{y}\bar{x}\bar{y}^{-1} = (yN)(xN)(y^{-1}N) = yxy^{-1}N \in G_iN/N,$$ since $yxy^{-1} \in G_i$ and
    $G_i \propnorm G_{i + 1}$ give us 
    $$yxy^{-1}N = yxy^{-1}nN \in G_iN/N,$$ 
    for all $n \in N$. Hence $G_iN/N \propnorm G_{i + 1}N/N$ for all $i$.

Now we show that $A = \frac{G_{i + 1}N/N}{G_iN/N}$ is abelian for all $i$. Let $\bar{x}G_iN/N = xn_1N(G_iN/N) = xN(G_iN/N) \in A$ and $\bar{y}G_iN/N = yn_2N(G_iN/N) = yN(G_iN/N) \in A$ s.t. $x,y \in G_i$ and $n_1,n_2 \in N$. Then 
\begin{equation}
\begin{aligned}(\bar{x}G_iN/N)(\bar{y}G_iN/N) &= xN(G_iN/N)yN(G_iN/N) = xyN(G_iN/N) = yxN(G_iN/N)\\
 &= yN(G_iN/N)xN(G_iN/N) = (\bar{y}G_iN/N)(\bar{x}G_iN/N),
\end{aligned}
\end{equation}
since $G_{i + 1}/G_i$ is abelian for all $i$, thus $A$ is abelian for all $i$. Hence the chain 
\[1 = G_0N/N \norm G_1N/N \norm \cdots \norm G_sN/N = G/N\]
satisfies the condition for $G/N$ to be solvable, so all quotient groups of $G$ are solvable. 
\end{proof}

\begin{e}[\textbf{5.1.14}]
Let $G = A_1 \times A_2 \times \cdots \times A_n$ and for each $i$ let $B_i$ be normal in $A_i$. Prove that $B=B_1 \times B_2 \times \cdots \times B_n \norm G$ and that 
\[(A_1 \times A_2 \times \cdots \times A_n)/(B_1 \times B_2 \times \cdots \times B_n) \cong (A_1/B_1) \times (A_2/B_2) \times \cdots \times (A_n/B_n) = C. \]

\end{e}
\begin{proof}
Let $a = (a_1,a_2,...,a_n) \in G$ and let $b= (b_1,b_2,...,b_n) \in B$. Then
\begin{equation}
\begin{aligned}
aba^{-1} &= (a_1,a_2,...,a_n)(b_1,b_2,...,b_n)(a_1,a_2,...,a_n)^{-1} = (a_1,a_2,...,a_n)(b_1,b_2,...,b_n)(a_1^{-1},a_2^{-1},...,a_n^{-1})\\
&= (a_1b_1a_1^{-1},a_2b_2a_2^{-1},...,a_nb_na_n^{-1}). 
\end{aligned}
\end{equation}
But each of $a_ib_ia_i^{-1}$ is in $B_i$ since $B_i \norm A_i$ for all $i$. So $(a_1b_1a_1^{-1},a_2b_2a_2^{-1},...,a_nb_na_n^{-1}) \in B \Rightarrow B \norm G$. 

Then we know $G/B$ is a group, so we let $\phi: G/B \rightarrow C$ be given by $$\phi((a_1,a_2,...,a_n)B) = (a_1B_1,a_2B_2,...,a_nB_n).$$ 
We prove that $\phi$ is an isomorphism.\\
\textbf{Homomorphism: } Let $(x_1,x_2,...,x_n)B,(y_1,y_2,...,y_n)B \in G/B$, then 
\begin{equation}
\begin{aligned}
\phi(((x_1,x_2,...,x_n)B)((y_1,y_2,...,y_n)B)) &= \phi(((x_1,x_2,...,x_n)(y_1,y_2,...,y_n))B)\\ 
&= \phi((x_1y_1,x_2y_2,...,x_ny_n)B) \\
&= (x_1y_1B_1,x_2y_2B_2,...,x_ny_nB_n)\\
 &= (x_1B_1,x_2B_2,...,x_nB_n)(y_1B_1,y_2B_2,...,y_nB_n)\\
  &= \phi((x_1,x_2,...,x_n)B)\phi((y_1,y_2,...,y_n)B), 
\end{aligned}
\end{equation}
By the direct product operation on $G/B$ and $C$, so $\phi$ is a homomorphism. \\ 
\textbf{Injection: } Let $(x_1,x_2,...,x_n)B,(y_1,y_2,...,y_n)B \in G/B$, and let 
\begin{equation}
\begin{aligned}
\phi((x_1,x_2,...,x_n)B) &= \phi((y_1,y_2,...,y_n)B)\\
\Rightarrow  (x_1B_1,x_2B_2,...,x_nB_n)&=(y_1B_1,y_2B_2,...,y_nB_n). 
\end{aligned}
\end{equation}
So then we have that $x_iB_i = y_iB_i$ for all $i$, thus
 \begin{equation}
\begin{aligned}(y_1,y_2,...,y_n)B &= (y_1,y_2,...,y_n)(B_1 \times B_2 \times \cdots \times B_n)=
(y_1B_1 \times y_2B_2 \times \cdots \times y_nB_n) \\
&= (x_1B_1 \times x_2B_2 \times \cdots \times x_nB_n) = (x_1,x_2,...,x_n)B
\end{aligned}
\end{equation}
 by the direct product operation, so $\phi$ is in injective. \\
\textbf{Surjection: } Let $(a_1B_1,a_2B_2,...,a_nB_n) \in C$. Then we must have that $a_i \in A_i$ for all $i$ by definition of $C$ and the quotient groups $A_i/B_i$ so $(a_1,a_2,...,a_n) \in G \Rightarrow (a_1,a_2,...,a_n)B \in G/B$, and $\phi((a_1,a_2,...,a_n)B) = (a_1B_1,a_2B_2,...,a_nB_n)$, so $\phi$ is surjective by definition. Hence $\phi$ is an isomorphism, and $G/B \cong C$. 
\end{proof}

\begin{e}[\textbf{5.2.1(a)}]
Find the number of nonisomorphic abelian groups of order $100$. 
\end{e}
We go about finding the invariant factors of abelian groups of order $100$. We know the total number of nonisomorphic groups is the product of the partition numbers of each of the powers of the unique primes in the prime factorization of $n = 100$. 


First, we find the prime decomposition of $100$, which is $100=2^25^2$.


The partition number of $2$ is $2$ so there are $2 \cdot 2 = 4$ possible abelian groups of order $100$. 

\begin{comment} We must have $2\cdot 5 \mid n_1$, so our possible values for $n_1$ are
\begin{itemize}
\item $2\cdot 5,$
\item $2^2\cdot 5,$
\item $2\cdot 5^2,$
\item $2^2\cdot 5^2.$
\end{itemize}
If $n_1 = 2 \cdot 5$, then the possible values for $n_2$ are $2,5,2\cdot 5$. If $n_2$ is $2$ or $5$ we must have a third term since $\prod n_i = n$ and we must have $n_3 = n_2$ since $n_3 \mid n_2$ which is not possible since then we would not have $\prod n_i = n$. So $n_2 = 2 \cdot 5$. 

If $n_1$ is $2^2 \cdot 5$ or $2\cdot 5^2$ the only possible values for $n_2$ are $5$ and $2$ respectively, and these complete the list.


And if $n_1 = 2^2\cdot 5^2$, we are done. So the possible abelian groups are $\mathbb{Z}_{100}, \mathbb{Z}_{20} \times \mathbb{Z}_5, \mathbb{Z}_{50} \times \mathbb{Z}_2, \mathbb{Z}_{10}^2$, and there are 4 or them. \end{comment}
\begin{e}[\textbf{5.2.1(b)}]
Find the number of nonisomorphic abelian groups of order $576$. 
\end{e}
The prime decomposition of $576$ is $2^6 \cdot 3^2$. So again, we find the partition numbers. The partition number of $6$ is $11$ and the partition number of $2$ is $2$ so we have $2 \cdot 11 = 22$ different abelian groups. 

\begin{comment}

So the possible invariant factor lists are: 
\begin{itemize}
\item $2 \cdot 3,2 \cdot 3,2,2,2,2$. 
\item $2 \cdot 3^2,2,2,2,2,2$.
\item $2^2 \cdot 3, 2 \cdot 3,2,2,2$. 
\item $2^2 \cdot 3, 2^2 \cdot 3,2,2$. 
\item $2^2 \cdot 3, 2^2 \cdot 3,2^2$. 
\item $2^2 \cdot 3^2, 2^2, 2^2$. 
\item $2^2 \cdot 3^2, 2^2,2,2$.
\item $2^2 \cdot 3^2, 2,2,2,2$. 
\item $2^3 \cdot 3^2, 2,2,2$. 
\item $2^3 \cdot 3, 2^3 \cdot 3$. 
\item $2^3 \cdot 3, 2^2 \cdot 3, 2$. 
\item $2^3 \cdot 3, 2 \cdot 3, 2, 2$. 
\item $2^3 \cdot 3^2, 2^3$. 
\item $2^3 \cdot 3^2, 2^2, 2$. 
\item $2^3 \cdot 3^2, 2,2, 2$. 
\item $2^4 \cdot 3^2, 2^2$. 
\item $2^4 \cdot 3^2, 2,2$. 
\item $2^4 \cdot 3, 2^2 \cdot 3$. 
\item $2^4 \cdot 3, 2 \cdot 3, 2$. 
\item $2^5 \cdot 3^2, 2$. 
\item $2^5 \cdot 3, 2 \cdot 3$.  
\item $2^6 \cdot 3^2$. 


\end{itemize}
So there are 22 different abelian groups of order 576. 

\end{comment}

\begin{e}[\textbf{5.2.4(b)}]
Determine which pairs of 
\[\{2^2, 2 \cdot 3^2\}, \{2^2 \cdot 3, 2 \cdot 3,\}, \{2^3, 3^2\}, \{2^2 \cdot 3^2, 2\}\]
are isomorphic, where $\{a_1,...,a_k\}$ denotes $\mathbb{Z}_{a_1} \times \cdots \times \mathbb{Z}_{a_k}$. 


Let $G_1 =\{2^2, 2 \cdot 3^2\}$, then by decomposing into prime powers each factor in the direct product, we see that $G_1 \cong \mathbb{Z}_4 \times  \mathbb{Z}_2 \times \mathbb{Z}_9$. Similarly: 

$G_2 =\{2^2 \cdot 3, 2 \cdot 3,\} \Rightarrow G_2 \cong \mathbb{Z}_4 \times \mathbb{Z}_3 \times \mathbb{Z}_2 \times \mathbb{Z}_3$

$G_3 =\{2^3, 3^2\} \Rightarrow G_3 \cong \mathbb{Z}_8 \times \mathbb{Z}_9$. 

$G_4 =\{2^2 \cdot 3^2, 2\} \Rightarrow G_4 \cong \mathbb{Z}_4 \times \mathbb{Z}_9 \times \mathbb{Z}_2$. 

So $G_1 \cong G_4$ and none of the other pairs are isomorphic. 



\end{e}

\begin{e}[6]
Prove that every $p$-group is solvable. 

\end{e}

\begin{proof}
Let $G$ be a $p$-group. Then $|G| = p^k$ for some $k \in \mathbb{Z}$. We know that $G$ is finite, and a finite group is solvable if and only if for each divisor $n$ of $|G|$ such that $(n,\frac{|G|}{n}) = 1$, $G$ has a subgroup of order $n$. We proceed by induction on $k$. Clearly groups of prime order are solvable, since if $H$ has prime order, $1 \norm H$, and $H/1 = H \cong \mathbb{Z}_{|H|}$, and indeed the only normal subgroups of $H$ are $H,1$. So the claim is true when $k = 1$. Assume the claim holds for all $i < {k + 1}$. We will prove $G$ is solvable when  $|G| = p^{k+ 1}$.  We know each divisor $n$ of $|G|$ is of the form $p^i$ where $i < {k + 1}$. And the only $n$ for which $(n,\frac{|G|}{n})  = (p^i,\frac{p^{k  +1}}{p^i}) =1$ holds are $n = 1$ or $n = p^{k  +1}$ and in both of these cases, $G$ has a subgroup of this order, namely $1$ and $G$ itself, so $G$ is solvable for all $k$. 
\end{proof}

\begin{e}[7]
Prove that if the orders of $A,B$ are coprime, then any subgroup of $A \times B$ is of the form $H \times K$ where $H \leq A$ and $K \leq B$. 

\end{e}

\begin{proof}


If $H \times K$ is $1,A \times B$, then the result is trivial, so let $H \times K$ be a nontrivial, proper subgroup. Let $x = (h_1,k_1), y= (h_2,k_2) \in H \times K$ such that $xy^{-1} \neq 1 \Rightarrow x \neq 1 \neq y$. Then $(h_1,k_1)((h_2^{-1},k_2^{-1}) = (h_1h_2^{-1},k_1k_2^{-1}) \in H \times K$ 
since $H \times K$ is a subgroup, so $h_1h_2^{-1} \in H$ and $k_1k_2^{-1} \in K$. So we have inverses and closure for both $H,K$ and we need only show that they are subsets of $A,B$, respectively. Since $A,B$ have relatively prime orders, $A \cap B = 1$. Let $h = h_1h_2^{-1}$ and $k =k_1k_2^{-1}$.  Since $|h| \mid |H|,|A|$, we must have that $|h| \nmid |K|,|B|$, else and similarly, $|k| \nmid |H|,|A|$ since otherwise, $A,B$ do not have coprime orders. So $h \notin K$ and $k \notin H$, and $H \subset A, K \subset B \Rightarrow H \leq A, K \leq B$. \end{proof}












\end{document}


