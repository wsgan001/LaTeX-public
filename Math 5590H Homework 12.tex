

%	options include 12pt or 11pt or 10pt
%	classes include article, report, book, letter, thesis

\title{Math 5590H Bonus}



\author{Brendan Whitaker}

\date{AU17}
\documentclass[10pt,oneside,reqno]{amsart}

\usepackage{graphicx}
\usepackage[margin=1in]{geometry}
\usepackage{amsmath}
\usepackage{amssymb}
\usepackage{amsthm}
\usepackage{bbm}
\usepackage{cancel}
\usepackage{verbatim}
\usepackage{amsrefs}
\usepackage{enumitem}
\usepackage{etoolbox}% http://ctan.org/pkg/etoolbox
\patchcmd{\thmhead}{(#3)}{#3}{}{}
\usepackage{braket}


\theoremstyle{plain}
\newtheorem{Thm}{Theorem}
\newtheorem{Cor}[Thm]{Corollary}
\newtheorem{Prop}[Thm]{Proposition}
\newtheorem{Lem}[Thm]{Lemma}
\newtheorem{Prob}[Thm]{Problem}
\newtheorem{Def}[Thm]{Definition}
\newtheorem{Q}[Thm]{Question}
\newtheorem*{e}{Exercise}
\newtheorem{ee}{Exercise}
\theoremstyle{definition}
\newtheorem{Remark}[Thm]{Remark}
\newtheorem{Tech}[Thm]{Technical Remark}
\newtheorem*{Claim}{Claim}
\newtheorem{Ex}[Thm]{Example}




\newcommand{\Mod}[1]{\ (\mathrm{mod}\ #1)}
\newcommand{\norm}{\trianglelefteq}
\newcommand{\propnorm}{\triangleleft}



\begin{document}

\title{Math 5590H Homework 12}

\date{AU17}

\author[Brendan Whitaker]{Brendan Whitaker}

\maketitle

\begin{e}[\textbf{7.6.7}]
Let $n|m$, $n,m \in \mathbb{N}$. Prove that the natural surjective ring projection $\pi: \mathbb{Z}_m \to \mathbb{Z}_n$ is also surjective on the units: $\mathbb{Z}_m^{\times} \to \mathbb{Z}_n^{\times}$. 
\end{e}

\begin{proof}
Let $n|m$, $n,m \in \mathbb{N}$. By corollary 18 and the Chinese remainder theorem, we know that if $m = p_1^{\alpha_1} \cdots p_k^{\alpha_k}$, $m = p_1^{\beta_1} \cdots p_k^{\beta_k}$, where $\beta_i \leq \alpha_i$ $\forall i$, then 
\begin{equation}
\begin{aligned}
\mathbb{Z}_m &= \mathbb{Z}_{p_1^{\alpha_1}} \times \cdots \times \mathbb{Z}_{p_k^{\alpha_k}},\\
\mathbb{Z}_n &= \mathbb{Z}_{p_1^{\beta_1}} \times \cdots \times \mathbb{Z}_{p_k^{\beta_k}},\\
\mathbb{Z}_m^{\times} &= \mathbb{Z}_{p_1^{\alpha_1}}^{\times} \times \cdots \times \mathbb{Z}_{p_k^{\alpha_k}}^{\times},\\
\mathbb{Z}_n^{\times} &= \mathbb{Z}_{p_1^{\beta_1}}^{\times} \times \cdots \times \mathbb{Z}_{p_k^{\beta_k}}^{\times}.\\
\end{aligned}
\end{equation}
So if $\pi:\mathbb{Z}_m \to \mathbb{Z}_n$ is the natural projection homomorphism, we know $\pi$ is surjective, and $\pi_i:\mathbb{Z}_{p_i^{\alpha_i}} \to \mathbb{Z}_{p_i^{\beta_i}}$ is also surjective. We want to show $\pi_i:\mathbb{Z}_{p_i^{\alpha_i}}^{\times} \to \mathbb{Z}_{p_i^{\beta_i}}^{\times}$ is surjective. Let $x_i \in \mathbb{Z}_{p_i^{\beta_i}}^{\times}$. Then we must have that $(x_i,p_i) = 1$. And $\pi_i^{-1}(x_i) = \{l_ip_i^{\beta_i} + x_i: 0\leq l_i \leq p_i^{\alpha_i - \beta_i} - 1, l_i \in \mathbb{Z}\}$. Suppose $(l_ip_i^{\beta_i} + x_i,p_i^{\alpha_i}) = a_i > 1$ $\forall l_i$. We know if $a_i|l_ip_i^{\beta_i}$, and $a_i|p_i^{\alpha_i}$, then $a_i= p_i^{\gamma_i}$ for some non-negative integer $\gamma_i \leq \alpha_i$. Let $l_i = 1$. Then $(p_i^{\beta_i} + x_i,p_i^{\alpha_i}) = p_i^{\gamma_i}$. So $\exists r_i \in \mathbb{Z}$ s.t. $p_i^{\beta_i} + x_i = r_ip_i^{\gamma_i} \Rightarrow x_i =  r_ip_i^{\gamma_i} - p_i^{\beta_i}$. We assume $\beta_i > 0$ else $\mathbb{Z}_{p_i^{\beta_i}} = 1$, and $\pi_i$ is the trivial homomorphism, hence surjective on the units. And $\gamma_i > 0$, else $a_i = 1$, so $p_i|x_i$, which is a contradiction since we said $(x_i,p_i) = 1$, so we must have that $a_i = p_i^{\gamma_i} = 1$ $\forall i$, so $(p_i^{\beta_i} + x_i,p_i^{\alpha_i}) = 1$ $\forall i$, so $\exists y_i = x_i + p_i^{\beta_i} \in \mathbb{Z}_{p_i^{\alpha_i}}^{\times}$ s.t. $\pi_i(y_i) = x_i$, so $\pi_i:\mathbb{Z}_{p_i^{\alpha_i}}^{\times} \to \mathbb{Z}_{p_i^{\beta_i}}^{\times}$ is surjective on the units, hence $\pi:\mathbb{Z}_m^{\times} \to \mathbb{Z}_n^{\times}$ is surjective on the units. 
\end{proof}

\vspace{1mm}

\begin{e}[\textbf{8.3.3}]
Determine all the representations of the integer $2130797 = 17^2 \cdot 73 \cdot 101$ as a sum of two squares. 
\end{e}
We first find all the representations of $73 \cdot 101 = 7373$. Now $73 = (8 + 3i)(8 -3i)$ and $101 = (10 + i)(10 - i)$, so if we wish to write $7373 = A^2 + B^2$, the possible factorizations of $A + Bi$ in the Gaussian integers are 
\begin{equation}
\begin{aligned}
(8 + 3i)(10  + i) &= 80 + 38i - 3 = 77 + 38i,\\
(8 + 3i)(10  - i) &= 80 +  22i + 3 = 83 + 22i,\\
(8 - 3i)(10  + i) &= 80 - 22i + 3 = 83 - 22i,\\
(8 - 3i)(10  - i) &= 80 - 38i - 3 = 77- 38i.
\end{aligned}
\end{equation}
Then $7373 = (\pm 77)^2 + (\pm 38)^2 = (\pm 83)^2 + (\pm 22)^2$, which gives us 8 possible combinations, and switching the order of $A$ and $B$ gives us another $8$, for a total of $16$ representations. So by multiplying each of $A$ and $B$ by 17, we get $16$ unique representations of 2130797 as a sum of two squares of the form $(17A)^2 + (17B)^2$. But since $17 \equiv 1 \mod 4$, we have additional representations. Note that $17^2 = (4 + i)^2(4 - i)^2$. Thus we have several factorizations if $A = Bi$ s.t. $A^2 + B^2 = 2130797$ given by:
\begin{equation}
\begin{aligned}
(4 + i)^2(8 + 3i)(10  + i) &= 851 + 1186i,\\
(4 + i)^2(8 + 3i)(10  - i) &= 1069 + 994i,\\
(4 + i)^2(8 - 3i)(10  + i) &= 1421 + 334i,\\
(4 + i)^2(8 - 3i)(10  - i) &= 1459  + 46i,\\
(4 - i)^2(8 + 3i)(10  + i) &=1459  - 46i,\\
(4 - i)^2(8 + 3i)(10  - i) &= 1421 - 334i,\\
(4 - i)^2(8 - 3i)(10  + i) &= 1069 - 994i,\\
(4 - i)^2(8 - 3i)(10  - i) &= 851 - 1186i.\\
\end{aligned}
\end{equation}
So we have $2130797 = (\pm851)^2 + (\pm1186)^2 = (\pm 1069)^2 + (\pm 994)^2 = (\pm 1421)^2 + (\pm 334)^2 = (\pm 1459)^2 + (\pm 46)^2$. This gives us 16 combinations, and switching the order of $A$ and $B$ gives us another 16, so we have $32$ additional representations of $2130797$ as a sum of two squares, for a total of $48$. \\

\begin{e}[\textbf{8.3.6}]
\end{e}
\begin{enumerate}
\item[]
\begin{enumerate}
\item \textit{Prove that the quotient ring $Q = \mathbb{Z}[i]/(1 + i)$ is a field of order $2$. }\\

\begin{proof}
Observe:
\begin{equation}
\begin{aligned}
\mathbb{Z}[i]/(1 + i) \cong \mathbb{Z}[x]/(x^2 + 1, x + 1) \cong \mathbb{Z}[-1]/((-1)^2 + 1)= \mathbb{Z}/(2) \cong \mathbb{Z}_2. 
\end{aligned}
\end{equation}
\end{proof}
\vspace{3mm}
\item \textit{Let $q \in \mathbb{Z}$ be a prime with $q \equiv 3\mod 4$. Prove that the quotient ring $\mathbb{Z}[i]/(q)$ is a field with $q^2$ elements. \\}
\begin{proof}
Note since $q \equiv 3 \mod 4$, we know $q$ is irreducible, and since the Gaussian integers are a UFD, we know $q$ is also prime by Proposition 8.3.18. Then we must have that $(q)$ is a prime ideal. Since $\mathbb{Z}[i]$ is also a principal ideal domain, we know that $(q)$ is also a maximal ideal by Proposition 8.2.7, which means $\mathbb{Z}[i]/(q)$ is a field since $\mathbb{Z}[i]$ is commutative, by Proposition 7.4.12. Then since $1,i$ generate $\mathbb{Z}[i]$, they also generate $\mathbb{Z}[i]/(q)$, so we have two cyclic subgroups $\langle 1 \rangle, \langle i \rangle$, each of order $q$. Now $\mathbb{Z}[i]/(q) = \langle 1 \rangle +  \langle i \rangle$ and $ \langle 1 \rangle \cap \langle i \rangle = 0$ so we must have that $\mathbb{Z}[i]/(q) \cong \langle 1 \rangle \times  \langle i \rangle \cong \mathbb{Z}_q^2$ as groups. Thus our field must have $q^2$ elements. 
\end{proof}


\vspace{3mm}
\item \textit{Let $p = \pi \overline{\pi} \equiv 1 \mod 4$ be a prime in $\mathbb{Z}$. Show that the hypotheses for the Chinese remainder theorem are satisfied, and that $\mathbb{Z}[i]/(p) \cong \mathbb{Z}[i]/(\pi) \times \mathbb{Z}[i]/(\overline{\pi})$ as rings. Show that the quotient ring $\mathbb{Z}[i]/(p)$ has order $p^2$ and conclude that $\mathbb{Z}[i]/(\pi)$ and $ \mathbb{Z}[i]/(\overline{\pi})$ are both fields of order $p$. }\\

\begin{proof}
By Proposition 8.3.18, we know that $p$ can be written as a sum of two squares, so $p = a^2 + b^2 = (a + bi)(a - bi) = \pi\overline{\pi}$. And by the same Proposition, we know $\pi,\overline{\pi}$ are irreducibles in $\mathbb{Z}[i]$, and since we are in a UFD, we know these elements are also prime. Then we know that they are coprime and so since $\mathbb{Z}[i]$ is a Euclidean domain, we know $\exists r,s \in \mathbb{Z}[i]$ s.t. $r\pi + s \overline{\pi} = 1$, so since $(1) = \mathbb{Z}[i]$, we know that $(\pi) + (\overline{\pi}) = \mathbb{Z}[i]$, and hence they are comaximal ideals, and thus the hypotheses for the Chinese remainder theorem are satisfied. 
Then from this, we know $(\pi) \cap (\overline{\pi}) = (\pi)( \overline{\pi}) = (p)$. Hence we have $\mathbb{Z}[i]/(p) = \mathbb{Z}[i]/((\pi) \cap (\overline{\pi})) \cong \mathbb{Z}[i]/(\pi) \times \mathbb{Z}[i]/(\overline{\pi})$ as rings. Then since $\mathbb{Z}[i]$ is also a PID, we know $(\pi),(\overline{\pi})$ are maximal ideals, and thus $\mathbb{Z}[i]/(\pi),\mathbb{Z}[i]/(\overline{\pi})$ are fields since $\mathbb{Z}[i]$ is commutative. Finally, $\mathbb{Z}[i]/(p) \cong \mathbb{Z}_p[i] = \{a + bi: a,b \in \mathbb{Z}\}$ which means we have $p$ distinct choices for each of $a,b$, hence $p^2$ total elements in our ring. 
\end{proof}
\end{enumerate}

\end{enumerate}
\begin{e}[\textbf{9.2.5}]
Exhibit all the ideals in the ring $F[x]/(p(x))$, where $F$ is a field and $p(x)$ a polynomial in $F[x]$. 
\end{e}
Factor $p(x)$ into irreducibles $p(x) = q_1(x) \cdots q_k(x)$. Then since the ideals in $F[x]/(p(x))$ are of the form $I/(p(x))$ for any ideal $I$ in $F[x]$ s.t. $(p(x)) \subset I$. Then all ideals are of the form $(\Pi^rq_i(x))/(p(x))$ where $r \leq k$. 

\begin{e}[\textbf{9.2.9}]
Determine the greatest common divisor of $a(x) = x^5 + 2x^3 + x^2 = x + 1$ and the polynomials $b(x) = x^5 + x^4 + 2x^3 + 2x^2 + 2x + 1$ in $\mathbb{Q}[x]$ and write it as a linear combination. 
\end{e}
We compute the gcd:
\begin{equation}
\begin{aligned}
b(x) &= a(x) + x^4 + x^2 + x,\\
a(x) &= x(x^4 + x^2 + x) + x^3 + x + 1,\\
x^4 + x^2 + x &= x( x^3 + x + 1),
\end{aligned}
\end{equation}
and thus the gcd is $ x^3 + x + 1$. So we write:
\begin{equation}
\begin{aligned}
x^3 + x + 1 &= a(x) - x(x^4 + x^2 + x)\\
&= a(x) - x(b(x) - a(x))\\
&= (x + 1)a(x) - xb(x).
\end{aligned}
\end{equation}
and so we've written the gcd as a linear combination of $a(x),b(x)$. \\

\begin{e}[\textbf{9.3.4}]
Let $R = \mathbb{Z} + x\mathbb{Q}[x] \subset \mathbb{Q}[x]$ be the set of polynomials in $x$ with rational coefficients whose constant terms is an integer. 
\end{e}

\begin{enumerate}
\item[]

\begin{enumerate}
\item \textit{Prove that $R$ is an integral domain and its units are $\pm 1$. }\\

\begin{proof}
Note that $\mathbb{Q}$ is a field, hence $\mathbb{Q}[x]$ is a PID, and thus an integral domain, and since $R$ is a subset of this integral domain, it too can't possibly have any zero divisors, hence it is also an integral domain. Also since the units in $\mathbb{Q}[x]$ are just the elements of $\mathbb{Q}$, since $\mathbb{Q}$ is a field, we know that the units in $R$ must also be constant terms, and the only integers with inverses are $\pm 1$, hence these are the units in $R$. 
\end{proof}
\vspace{3mm}
\item \textit{Show that the irreducibles in $R$ are $\pm p$ where $p$ is a prime in $\mathbb{Z}$ and the polynomials $f(x)$ which are irreducible in $\mathbb{Q}[x]$ and have constant term $\pm 1$. Prove that these irreducibles are prime in $R$. }\\

\begin{proof}
As noted above, $\mathbb{Q}[x]$ must be a PID, and thus we have that it is also a UFD, and so by Proposition 8.3.12, every irreducible element is also prime. Let $f(x) \in R$ be irreducible. We first consider the case where $f$ is constant. Then $f$ is an integer, and so since we already know the units are $\pm 1$ in $R$, exactly the primes in $\mathbb{Z}$ are irreducible in this case. Now suppose that $f(x)$ is a nonconstant polynomial, and let the constant term $a_0 \neq \pm 1$. If the constant terms is $0$, we can divide by $x$, which is not a unit, and in the case where $f(x) = \alpha x$, we know $\alpha x = \alpha\frac{1}{2}x$, and so $\alpha x$ is not irreducible. Then clearly we may divide by $|a_0|$, and since all the other coefficients are rationals, we have constructed some $g(x) \neq \pm 1$ s.t. $f(x) = |a_0|g(x)$, i.e. $f$ is the product of two non-units, hence reducible. Thus we must have that $a_0  = \pm 1$ if and only if $f$ is irreducible in this case. 
\end{proof}
\vspace{3mm}
\item \textit{Show that $x$ cannot be written as the product of irreducibles in $R$ (in particular, $x$ is not irreducible) and conclude that $R$ is not a UFD. }\\
\begin{proof}
Let $f,g,...,h$ be irreducibles in $R$. But then we cannot have any non-constant terms, since that would imply a degree $\geq 2$ by part (b), so all are constant, which is impossible, so $x$ is not a product of irreducibles. And we proved in part (b) that $x$ is not irreducible, and so $R$ cannot be a UFD since $x$ cannot be written as a product of primes. 
\end{proof}
\vspace{3mm}
\item \textit{Show $x$ is not a prime and describe $R/(x)$. }\\
\begin{proof}
Suppose $x$ were a prime. Then since $R$ is an integral domain, $x$ would also be irreducible, but it's not, so it can't be. 
\end{proof}

$R/(x) =\{ax + b:a \in \mathbb{Q}, b \in \mathbb{Z}\}$. Every non-unit is a zero divisor, since we can multiply by $\frac{1}{b}x$. Hence we are not in an integral domain, but we do have $\pm 1$ units and the ring is still commutative. 
\end{enumerate}

\begin{e}[\textbf{9.5.5}]
Prove that $\sum\phi(d) = n$ where $d$ runs through the divisors of $n$. 
\end{e}
\begin{proof}
Consider $\mathbb{Z}_n$. Let $d |n$, then write $n = dk$, and consider $\langle k \rangle$. This a cyclic subgroup of order $d$. This subgroup is unique because if it were not, we would have $l$ a generator of another subgroup of order $d$ in $\mathbb{Z}_n$, hence $dk \equiv dl \equiv 0 \mod n$, and since $d$ is the minimal integer such that $dk \equiv 0 \mod n$, we must have that $dk|dl$, but this is also true for $l$, so $dl|dk$ and so $dk = dl$ and $l = k$, a contradiction. And so the number of elements of order $d$ in $R$ is $\phi(d)$, since $\mathbb{Z}_d$ has exactly that many generators, and we have only one such subgroup. If there were another element of order $d$ not in $\langle k \rangle$, it would form another subgroup which is impossible. Hence since every integer from 1 to $n$ is an element of order $d$ for some $d |n$ we have the desired equality. 
\end{proof}
\end{enumerate}
















\end{document}



