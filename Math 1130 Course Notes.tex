%% filename: Math 5591H Notes.tex
%% version: 1.0
%% date: 2018/03/24
%%
%% American Mathematical Society
%% Technical Support
%% Publications Technical Group
%% 201 Charles Street
%% Providence, RI 02904
%% USA
%% tel: (401) 455-4080
%%      (800) 321-4267 (USA and Canada only)
%% fax: (401) 331-3842
%% email: tech-support@ams.org
%% 
%% Copyright 2006, 2008-2010, 2014 American Mathematical Society.
%% 
%% This work may be distributed and/or modified under the
%% conditions of the LaTeX Project Public License, either version 1.3c
%% of this license or (at your option) any later version.
%% The latest version of this license is in
%%   http://www.latex-project.org/lppl.txt
%% and version 1.3c or later is part of all distributions of LaTeX
%% version 2005/12/01 or later.
%% 
%% This work has the LPPL maintenance status `maintained'.
%% 
%% The Current Maintainer of this work is the American Mathematical
%% Society.
%%
%% ====================================================================

%    AMS-LaTeX v.2 driver file template for use with amsbook
%
%    Remove any commented or uncommented macros you do not use.

\documentclass[12pt]{amsbook}

%    For use when working on individual chapters
%\includeonly{}

%    Include referenced packages here.
\usepackage{}
\usepackage[margin=1in]{geometry}
%\usepackage{graphicx}
\usepackage{amsmath}
\usepackage{amssymb}
\usepackage{amsthm}
%\usepackage{bbm}
%\usepackage{cancel}
\usepackage{verbatim}
\usepackage{amsrefs}
\usepackage{enumitem}
%\usepackage{tikz}
%\usepackage{environ}
\usepackage{tikz-cd}
%\usepackage[pdf]{pstricks}
\usepackage{braket}
\usetikzlibrary{cd}
%\usepackage[ruled,linesnumbered]{algorithm2e}
%\usepackage{adjustbox}
%\usepackage{changepage}
%\usepackage{import}
%\usepackage{newclude}
\usepackage[all,cmtip]{xy}
\usepackage[]{titlesec}
\usepackage[english]{babel}
\usepackage[utf8x]{inputenc}
\usepackage{graphicx}







\usepackage{hyperref}

\hypersetup{
     colorlinks   = true,
     citecolor    = red
}








\titlespacing{\section}{0mm}{7mm}{5mm}[0mm]
\titleformat{\section}[frame]
{\normalfont\scshape}
{\filright
\footnotesize
\enspace SECTION \thesection\enspace}
{8pt}
{\Large\scshape\filcenter}



\titleformat{\chapter}[display]
{\scshape\LARGE}
{\filleft\MakeUppercase{\chaptertitlename} \Huge\thechapter}
{4ex}
{\titlerule
\vspace{2ex}%
\filright}
[\vspace{2ex}%
\titlerule]



\let\oldemptyset\emptyset
\let\emptyset\varnothing
\theoremstyle{plain}
\newtheorem{Thm}{Theorem}
\newtheorem{Prob}[Thm]{Problem}
%\theoremstyle{definition}
\newtheorem{Remark}[Thm]{Remark}
\newtheorem{Tech}[Thm]{Technical Remark}
\newtheorem*{Claim}{Claim}
%----------------------------------------
%CHAPTER STUFF
\newtheorem{theorem}{Theorem}[chapter]
\numberwithin{section}{chapter}
\numberwithin{equation}{chapter}
%CHAPTER STUFF
%----------------------------------------
\newtheorem{lem}[theorem]{Lemma}
%\newtheorem{Q}[theorem]{Question}
\newtheorem{Prop}[theorem]{Proposition}
\newtheorem{Cor}[theorem]{Corollary}

\theoremstyle{definition}
\newtheorem{e}{Exercise}
\newtheorem{Def}[theorem]{Definition}
\newtheorem{Ex}[theorem]{Example}
\newtheorem{xca}[theorem]{Exercise}

\theoremstyle{remark}
\newtheorem{rem}[theorem]{Remark}
%NAMED THEOREMS
\theoremstyle{plain}
\newtheorem*{namedthm}{\namedthmname}
\newcounter{namedthm}
\makeatletter
	\newenvironment{named}[2]
	{\def\namedthmname{#1}
	\refstepcounter{namedthm}
	\namedthm[#2]\def\@currentlabel{#1}}
	{\endnamedthm}
\makeatother






\newcommand{\Mod}[1]{\ (\mathrm{mod}\ #1)}
\newcommand{\norm}{\trianglelefteq}
\newcommand{\propnorm}{\triangleleft}
\newcommand{\semi}{\rtimes}
\newcommand{\sub}{\subseteq}
\newcommand{\fa}{\forall}
\newcommand{\R}{\mathbb{R}}
\newcommand{\z}{\mathbb{Z}}
\newcommand{\n}{\mathbb{N}}
\newcommand{\Q}{\mathbb{Q}}
\renewcommand{\c}{\mathbb{C}}
\newcommand{\F}{\mathbb{F}}
\newcommand{\bb}{\vspace{3mm}}
\newcommand{\heart}{\ensuremath\heartsuit}
\newcommand{\mc}{\mathcal}
\newcommand{\bee}{\begin{equation}\begin{aligned}}
\newcommand{\eee}{\end{aligned}\end{equation}}
\newcommand{\nequiv}{\not\equiv}
\newcommand{\lc}[2]{#1_1 + \cdots + #1_{#2}}
\newcommand{\lcc}[3]{#1_1 #2_1 + \cdots + #1_{#3} #2_{#3}}
\newcommand{\ten}{\otimes} %tensor product
\newcommand{\fracc}{\frac}
\newcommand{\tens}{\otimes}
\newcommand{\lpar}{\left(}
\newcommand{\rpar}{\right)}
\newcommand{\floor}{\lfloor}
\newcommand{\Tau}{\mc{T}}
\newcommand{\rank}{\text{rank}}
\DeclareMathOperator{\coker}{coker}
\newcommand*\pp{{\rlap{\('\)}}}






\renewcommand{\leq}{\leqslant}
\renewcommand{\geq}{\geqslant}
\renewcommand{\tt}{\text}
\renewcommand{\rm}{\normalshape}%text inside math
\renewcommand{\Re}{\operatorname{Re}}%real part
\renewcommand{\Im}{\operatorname{Im}}%imaginary part
\renewcommand{\bar}{\overline}%bar (wide version often looks better)
\renewcommand{\phi}{\varphi}


\makeatletter
\newenvironment{restoretext}%
    {\@parboxrestore%
     \begin{adjustwidth}{}{\leftmargin}%
    }{\end{adjustwidth}
     }
\makeatother


%---------END-OF-PREAMBLE---------
%---------------------------------

%    For a single index; for multiple indexes, see the manual
%    "Instructions for preparation of papers and monographs:
%    AMS-LaTeX" (instr-l.pdf in the AMS-LaTeX distribution).
\makeindex

\begin{document}

\begin{titlepage}

\newcommand{\HRule}{\rule{\linewidth}{0.5mm}} % Defines a new command for the horizontal lines, change thickness here

\center % Center everything on the page
 
%----------------------------------------------------------------------------------------
%	HEADING SECTIONS
%----------------------------------------------------------------------------------------

\textsc{\LARGE BRENDAN WHITAKER }\\[0.3cm] % Name of your university/college
\textsc{\LARGE THE OHIO STATE UNIVERSITY  }\\[0.3cm]
%\textsc{\Large JALANDHAR-144011, PUNJAB(INDIA) }\\[0.3cm]
\textsc{\Large Mathematics}\\[0.5cm] % Major heading such as course name
 % Minor heading such as course title

%----------------------------------------------------------------------------------------
%	TITLE SECTION
%----------------------------------------------------------------------------------------

\HRule \\[0.4cm]
{ \Huge \bfseries COLLEGE ALGEBRA\\
\vspace{1mm}  (MATH 1130)}\\[0.03cm] % Title of your document
\HRule \\[1.5cm]

 
%----------------------------------------------------------------------------------------
%	AUTHOR SECTION
%----------------------------------------------------------------------------------------

\begin{minipage}{0.4\textwidth}
\begin{flushleft} \large
\emph{Author:}\\
Brendan Whitaker \\Undergraduate\\3rd Year % Your name
\end{flushleft}
\end{minipage}
~
\begin{minipage}{0.4\textwidth}
\begin{flushright} \large
\emph{Instructor:} \\
Alexander Leibman\\Professor\\Dept. of Mathematics % Supervisor's Name
\end{flushright}
\end{minipage}\\[1cm]

% If you don't want a supervisor, uncomment the two lines below and remove the section above
%\Large \emph{Author:}\\
%John \textsc{Smith}\\[3cm] % Your name

%----------------------------------------------------------------------------------------
%	DATE SECTION
%----------------------------------------------------------------------------------------

{\large January-April, 2018 \\Course Notes}\\[1cm] % Date, change the \today to a set date if you want to be precise

%----------------------------------------------------------------------------------------
%	LOGO SECTION
%----------------------------------------------------------------------------------------

\vspace{15mm}
\includegraphics[scale=0.15]{osuLogo.png}\\[1cm] % Include a department/university logo - this will require the graphicx package
 
%----------------------------------------------------------------------------------------

\vfill % Fill the rest of the page with whitespace

\end{titlepage}

%\frontmatter

\title{ABSTRACT ALGEBRA II NOTES}

%    Remove any unused author tags.

%    author one information
\author{BRENDAN WHITAKER}
\address{}
\curraddr{}
\email{}
\thanks{}

%    author two information
\author{}
\address{}
\curraddr{}
\email{}
\thanks{}

\subjclass[2010]{12-XX}

\keywords{}

\date{SP18}

\begin{abstract}
A comprehensive set of notes for Professor Alexander Leibman's Abstract Algebra II course, taken SP18 at The Ohio State University. The material starts with Part III of Dummit and Foote's \textit{Abstract Algebra}, which covers Modules and Vector Spaces. 
\end{abstract}

%\maketitle

%    Dedication.  If the dedication is longer than a line or two,
%    remove the centering instructions and the line break.
%\cleardoublepage
%\thispagestyle{empty}
%\vspace*{13.5pc}
%\begin{center}
%  Dedication text (use \\[2pt] for line break if necessary)
%\end{center}
%\cleardoublepage

%    Change page number to 6 if a dedication is present.
\setcounter{page}{4}

\tableofcontents






\bibliography{}
%    See note above about multiple indexes.
\printindex

\end{document}

%-----------------------------------------------------------------------
% End of amsbook-template.tex
%-----------------------------------------------------------------------
