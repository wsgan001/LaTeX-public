
%	options include 12pt or 11pt or 10pt
%	classes include article, report, book, letter, thesis

\title{\Huge }



\author{\huge CSE 2221 Lecture Notes \linebreak \\\\ \Large Professor Bucci}

\date{SP17}
\documentclass[10pt]{article}

\usepackage{graphicx}

\usepackage{amsmath,amsthm,amssymb}
\usepackage{listings}





\begin{document}

\maketitle

When you trace a recursive method, and you get to the recursive call, treat it like a pre-existing method. i.e. just plug in what the method contract says the method call should return/do. \\\\
Subclasses vs superclasses. The compiler can't see methods declared in the subclass. 

\subsection*{Lecture: Tuesday, April 4th}
\paragraph*{Sequence} This component family allows you to mess with strings of entries of any type T through direct access by position. \\
It is an alternative to an array, and is a generic container type like Queue and Set. \\\\
Standard still has clear, newInstance, and transferFrom. The sequence type also has a SequenceKernel. \\
The mathematical model is a string of T. You can use a sequence as a Queue with full functionality, you would prefer a Queue over a Sequence if you wanted anyone reading your code to understand that you only wanted them to modify elements at the ends. \\
No-argument constructor yields $<>$. \\


\paragraph{void add(int pos, T x)}:\\
Adds x and position pos of this. \\
x is aliased!\\
Requires: $0 \leq pos$ and $pos \leq |this|$. \\\\

\paragraph{T remove(int pos)}:\\
removes and returns the entry at position pos of this. \\
Updates this.\\
Requires: $0 \leq pos$ and $pos \leq |this|$. \\\\

\paragraph{T entry(int pos)}:\\
Reports entry at pos\\
Aliases the reference returned by entry. \\
Requires: $0 \leq pos$ and $pos \leq |this|$. \\
ensures $<entry> = \#this[pos, pos+1)$. \\\\
\paragraph{T replaceEntry(int pos, T x)}:\\
replaces the entry at pos with x\\
aliases reference x. \\

\paragraph{Stack} Model is a string of T's again. Queue gave us first in first out behavior. Stack exhibits LIFO last-in-first-out behavior, i.e. like a stack of pancakes, you can't mess with the bottom. 
No-argument constructor yields $<>$. \\\\
\paragraph{void push(T x)}:\\
Adds x at the top (left end) of this. \\
Aliases x again.\\
Updates this.\\
Ensures: $this = <x> + \#this$\\
\paragraph{T pop()}:\\
Removes and returns the entry at the top of the stack. \\
Require $this \neq \emptyset$. \\
\paragraph{int length()}:\\
self-explanatory.\\
\paragraph{T top()}:\\
Returns the entry at the top of this. \\
Aliases whatever it returns.\\
Require $this \neq \emptyset$. \\
Also restores the Stack.\\
\paragraph{T replaceTop(T x)}:\\
replaces the top of x with this and returns the old top. \\\\
\paragraph{More Queue} Let's look at the secondary method "sort" for Queue. 
\paragraph{void sort(Comparator<T>)}:\\
This interface contains a header for only one method:\\
 \textbf{int compare(T o1,T o2)} When we implement the interface, we are allowed to define any ordering so long as the method returns a negative integer if the first is less than the second, positive if greater, and 0 if equal. Where less, greater, equal are defined by the method and specified by the type T being used. \\
 



















\end{document}


