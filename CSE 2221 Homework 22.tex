
%	options include 12pt or 11pt or 10pt
%	classes include article, report, book, letter, thesis

\title{\Huge Brendan Whitaker}



\author{\huge CSE 2221 Homework 22 \linebreak \\\\ \Large Professor Bucci}

\date{18 April 2017}
\documentclass[10pt]{article}



\usepackage{amsmath,amsthm,amssymb}
\usepackage{listings}

%\usepackage[margin=1in]{geometry}&


\begin{document}
\maketitle


\paragraph{1. } Controller1

\begin{lstlisting}
import components.stack.Stack;

/**
 * Controller class.
 *
 * @author Bruce W. Weide
 * @author Paolo Bucci
 */
public final class AppendUndoController1 implements 
 AppendUndoController {

    /**
     * Model object.
     */
    private final AppendUndoModel model;

    /**
     * View object.
     */
    private final AppendUndoView view;

    /**
     * Updates view to display model.
     *
     * @param model
     *            the model
     * @param view
     *            the view
     */
    private static void updateViewToMatchModel(AppendUndoModel model,
            AppendUndoView view) {
        /*
         * Get model info
         */
        String input = model.input();
        Stack<String> output = model.output();
        String outputString = "";
        String helper = "";
        for (String elem : output) {
            helper = outputString;
            outputString = helper + elem;
        }

        /*
         * Update view to reflect changes in model
         */
        view.updateInputDisplay(input);
        view.updateOutputDisplay(outputString);
    }

    /**
     * Constructor; connects {@code this} to the 
     * model and view it coordinates.
     *
     * @param model
     *            model to connect to
     * @param view
     *            view to connect to
     */
    public AppendUndoController1(AppendUndoModel model, 
    AppendUndoView view) {
        this.model = model;
        this.view = view;
        /*
         * Update view to reflect initial value of model
         */
        updateViewToMatchModel(this.model, this.view);
    }

    /**
     * Processes reset event.
     */
    @Override
    public void processResetEvent() {
        /*
         * Update model in response to this event
         */
        this.model.setInput("");
        Stack<String> clearStack = this.model.output();
        clearStack.clear();
        /*
         * Update view to reflect changes in model
         */
        updateViewToMatchModel(this.model, this.view);
    }

    /**
     * Processes Append event.
     *
     * @param input
     *            value of input text
     */
    @Override
    public void processAppendEvent(String input) {
        /*
         * Update model in response to this event
         */
        Stack<String> out = this.model.output();
        out.push(input);
        this.model.setInput("");

        /*
         * Update view to reflect changes in model
         */
        updateViewToMatchModel(this.model, this.view);
    }

    @Override
    public void processUndoEvent() {
        Stack<String> out = this.model.output();

        String in = out.pop();
        this.model.setInput(in);
        /*
         * Update view to reflect changes in model
         */
        updateViewToMatchModel(this.model, this.view);

    }

}

\end{lstlisting}

\paragraph{2. } Model1

\begin{lstlisting}
import components.stack.Stack;
import components.stack.Stack1L;

/**
 * Model class.
 *
 * @author Bruce W. Weide
 * @author Paolo Bucci
 */
public final class AppendUndoModel1 implements AppendUndoModel {

    /**
     * Model variables.
     */
    private String input;
    /**
     * More Model variables.
     */
    private Stack<String> output;

    /**
     * Default constructor.
     */
    public AppendUndoModel1() {
        /*
         * Initialize model; both variables start as empty strings
         */
        Stack<String> nullStack = new Stack1L<String>();
        this.input = "";
        this.output = nullStack;
    }

    @Override
    public void setInput(String input) {
        this.input = input;
    }

    @Override
    public String input() {
        return this.input;
    }

    @Override
    public Stack<String> output() {
        return this.output;
    }

}

\end{lstlisting}

\paragraph{3. } View1

\begin{lstlisting}
import java.awt.Cursor;
import java.awt.GridLayout;
import java.awt.event.ActionEvent;

import javax.swing.JButton;
import javax.swing.JFrame;
import javax.swing.JPanel;
import javax.swing.JScrollPane;
import javax.swing.JTextArea;

/**
 * View class.
 *
 * @author Bruce W. Weide
 * @author Paolo Bucci
 */
@SuppressWarnings("serial")
public final class AppendUndoView1 extends 
JFrame implements AppendUndoView {

    /**
     * Controller object.
     */
    private AppendUndoController controller;

    /**
     * GUI widgets that need to be in scope in 
      * actionPerformed method, and
     * related constants. (Each should have its own 
     * Javadoc comment, but these
     * are elided here to keep the code shorter.)
     */
    private static final int LINES_IN_TEXT_AREAS = 5,
            LINE_LENGTHS_IN_TEXT_AREAS = 20, 
            ROWS_IN_BUTTON_PANEL_GRID = 1,
            COLUMNS_IN_BUTTON_PANEL_GRID = 3, 
            ROWS_IN_THIS_GRID = 3,
            COLUMNS_IN_THIS_GRID = 1;

    /**
     * Text areas.
     */
    private final JTextArea inputText, outputText;

    /**
     * Buttons.
     */
    private final JButton resetButton, appendButton, undoButton;

    /**
     * No-argument constructor.
     */
    public AppendUndoView1() {
        // Create the JFrame being extended

        /*
         * Call the JFrame (superclass) constructor 
         * with a String parameter to
         * name the window in its title bar
         */
        super("Simple GUI Demo With MVC");

        // Set up the GUI widgets --------------------------------------------

        /*
         * Create widgets
         */
        this.inputText = new JTextArea("", LINES_IN_TEXT_AREAS,
                LINE_LENGTHS_IN_TEXT_AREAS);
        this.outputText = new JTextArea("", LINES_IN_TEXT_AREAS,
                LINE_LENGTHS_IN_TEXT_AREAS);
        this.resetButton = new JButton("Reset");
        this.appendButton = new JButton("Append");
        this.undoButton = new JButton("Undo");
        /*
         * Text areas should wrap lines, and outputText 
         * should be read-only
         */
        this.inputText.setEditable(true);
        this.inputText.setLineWrap(true);
        this.inputText.setWrapStyleWord(true);
        this.outputText.setEditable(false);
        this.outputText.setLineWrap(true);
        this.outputText.setWrapStyleWord(true);
        /*
         * Create scroll panes for the text areas in
         *  case text is long enough to
         * require scrolling in one or both dimensions
         */
        JScrollPane inputTextScrollPane
         = new JScrollPane(this.inputText);
        JScrollPane outputTextScrollPane
         = new JScrollPane(this.outputText);
        /*
         * Create a button panel organized using grid layout
         */
        JPanel buttonPanel = new JPanel(new GridLayout(
                ROWS_IN_BUTTON_PANEL_GRID, 
                COLUMNS_IN_BUTTON_PANEL_GRID));
        /*
         * Add the buttons to the button panel, 
         * from left to right and top to
         * bottom
         */
        buttonPanel.add(this.resetButton);
        buttonPanel.add(this.appendButton);
        buttonPanel.add(this.undoButton);
        /*
         * Organize main window using grid layout
         */
        this.setLayout(new GridLayout(ROWS_IN_THIS_GRID,
         COLUMNS_IN_THIS_GRID));
        /*
         * Add scroll panes and button panel to 
         * main window, from left to right
         * and top to bottom
         */
        this.add(inputTextScrollPane);
        this.add(buttonPanel);
        this.add(outputTextScrollPane);

        // Set up the observers ----------------------------------------------

        /*
         * Register this object as the observer for all GUI events
         */
        this.resetButton.addActionListener(this);
        this.appendButton.addActionListener(this);
        this.undoButton.addActionListener(this);

        // Start the main application window --------------------------------

        /*
         * Make sure the main window is appropriately sized for the widgets in
         * it, that it exits this program when closed, and that it becomes
         * visible to the user now
         */
        this.pack();
        this.setDefaultCloseOperation(JFrame.EXIT_ON_CLOSE);
        this.setVisible(true);
    }

    /**
     * Register argument as observer/listener of this; this must be done first,
     * before any other methods of this class are called.
     *
     * @param controller
     *            controller to register
     */
    @Override
    public void registerObserver(AppendUndoController controller) {
        this.controller = controller;
    }

    /**
     * Updates input display based on String provided as argument.
     *
     * @param input
     *            new value of input display
     */
    @Override
    public void updateInputDisplay(String input) {
        this.inputText.setText(input);
    }

    /**
     * Updates output display based on String provided as argument.
     *
     * @param output
     *            new value of output display
     */
    @Override
    public void updateOutputDisplay(String output) {
        this.outputText.setText(output);
    }

    @Override
    public void actionPerformed(ActionEvent event) {
        /*
         * Set cursor to indicate computation on-going; 
         * this matters only if
         * processing the event might take a noticeable 
         * amount of time as seen
         * by the user
         */
        this.setCursor(Cursor.getPredefinedCursor
        (Cursor.WAIT_CURSOR));
        /*
         * Determine which event has occurred that 
         * we are being notified of by
         * this callback; in this case, the source 
         * of the event (i.e, the widget
         * calling actionPerformed) is all we need 
         * because only buttons are
         * involved here, so the event must be a 
         * button press; in each case,
         * tell the controller to do whatever is needed
         *  to update the model and
         * to refresh the view
         */
        Object source = event.getSource();
        if (source == this.resetButton) {
            this.controller.processResetEvent();
        } else if (source == this.appendButton) {
            this.controller.processAppendEvent(this.
            inputText.getText());
        }
        /*
         * Set the cursor back to normal (because we 
         * changed it at the beginning
         * of the method body)
         */
        this.setCursor(Cursor.getDefaultCursor());
    }

    /**
     * Updates display of whether undo operation is allowed.
     *
     * @param allowed
     *            true iff undo is allowed
     */
    @Override
    public void updateUndoAllowed(boolean allowed) {

    }

}

\end{lstlisting}











\end{document}