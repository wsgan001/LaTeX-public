

%	options include 12pt or 11pt or 10pt
%	classes include article, report, book, letter, thesis

\title{CSE 2321 Notes}



\author{Brendan Whitaker}

\date{AU17}
\documentclass[10pt]{article}

\usepackage{graphicx}

\usepackage{amsmath}
\usepackage{amssymb}
\usepackage{amsthm}
\usepackage{bbm}
\usepackage{cancel}
\usepackage{verbatim}
\usepackage{amsrefs}

\theoremstyle{plain}
\newtheorem{Thm}{Theorem}
\newtheorem{Cor}[Thm]{Corollary}
\newtheorem{Prop}[Thm]{Proposition}
\newtheorem{Lem}[Thm]{Lemma}
\newtheorem{Prob}[Thm]{Problem}
\newtheorem{Def}[Thm]{Definition}
\newtheorem{Q}[Thm]{Question}
\theoremstyle{definition}
\newtheorem{Remark}[Thm]{Remark}
\newtheorem{Tech}[Thm]{Technical Remark}
\newtheorem*{Claim}{Claim}
\newtheorem{Ex}[Thm]{Example}


\newtheorem{theorem}{Theorem}[section]
\newtheorem{theorem2}{Theorem}
\newtheorem{corollary}{Corollary}[theorem]
\newtheorem{lemma}[theorem]{Lemma}
\newtheorem*{prop}{Proposition}
\newtheorem{comb}{Combinatorial Formula for the Coefficients}

\newcommand{\Mod}[1]{\ (\mathrm{mod}\ #1)}



\begin{document}

\maketitle

\subsection*{Lecture: Friday, August 25th}

\begin{itemize}
\item Today is Tuesday and today is Friday. \textbf{False.} 
\item Today is Friday and I attended class. \textbf{True. }
\item Today is Tuesday or I attended class. \textbf{True. }
\item Either today is Friday or I attended class.  \textbf{False.} 
\item Today is not Tuesday and today is Friday and I attended class. \textbf{True. }
\item Today is not Tuesday or today is not Friday or I didn't attend class? \textbf{True.} 
\end{itemize}

\begin{tabular}{c|c|c|c|c|c}
P & Q & R & $\neg$ P & $\neg$ P and Q & $\neg$ P and Q and R \\
\hline
T & T & T & F & F & F\\
T & T & F & F & F & F\\
T & F & T & F & F & F\\
T & F & F & F & F & F\\
\hline
F & T & T & T & T & T\\
F & T & F & T & T & F\\
F & F & T & T & F & F\\
F & F & F & T & F & F\\


\end{tabular}\\\\\\

\begin{tabular}{c|c|c|c|c|c|c|c}
P & Q & R & $\neg$ P & $\neg$ Q & $\neg$ R & $\neg$ P or $\neg$ Q & $\neg$ P or $\neg$ Q or $\neg$ R\\
\hline
T & T & T & F & F & F & F & F\\
T & T & F & F & F & T & F & T\\
T & F & T & F & T & F & T & T\\
T & F & F & F & T & T & T & T\\
\hline
F & T & T & T & F & F & T & T\\
F & T & F & T & F & T & T & T\\
F & F & T & T & T & F & T & T\\
F & F & F & T & T & T & T & T\\


\end{tabular}\\\\\

\begin{Def}
A compound proposition(statement) that is always true no matter what the truth values of the propositions that occur in it, is called a \textbf{tautology}. 
\end{Def}

\begin{Def}
A compound proposition(statement) that is always false no matter what the truth values of the propositions that occur in it, is called a \textbf{contradiction}. 
\end{Def}

\begin{Def}
A compound proposition(statement) that is neither a tautology nor a contradiction is called a \textbf{contingency}. 
\end{Def}

\begin{Def}
Let P and Q be propositions. The \textbf{implication $\boldsymbol{P \Rightarrow Q}$} is the proposition that is false when P is true and Q is false and true otherwise. 
\end{Def}


\begin{tabular}{c|c|c|c}
P & Q & $P \Rightarrow Q$\\
\hline
T & T & T\\
T & F & F\\
F & T & T\\
F & F & T\\

\end{tabular}\\\\\


\begin{Def}
The proposition $Q \Rightarrow P$ is called the \textbf{converse} of $P \Rightarrow Q$. 
\end{Def}

\begin{Def}
The \textbf{contrapositive} of $P \Rightarrow Q$ is the proposition  $\neg Q \Rightarrow \neg P$. 
\end{Def}

\begin{Def}
The \textbf{inverse} of $P \Rightarrow Q$ is the proposition  $\neg P \Rightarrow \neg Q$. 
\end{Def}

\begin{Def}
Let P and Q be propositions. The \textbf{biconditional} $\boldsymbol{P \Leftrightarrow Q}$ is true precisely when both the implications $P \Rightarrow Q$ and $Q \Rightarrow P$ are true. 
\end{Def}


\subsection*{Lecture: Monday, August 28th}

\begin{tabular}{cc|ccc|ccc|ccc}
P & Q & $(P$ & $\Rightarrow$ & $Q)$ & $\Rightarrow$ & $\Leftrightarrow$ & $\Leftarrow$ & $( \neg Q$ & $\Rightarrow$ & $\neg P$ ) \\
\hline
T & T & T & T & T & T & T & T & F & T & F\\
T & F & T & F & F & T & T & T & T & F & F\\
F & T & F & T & T & T & T & T & F & T & T\\
F & F & F & T & F & T & T & T & T & T & T\\
\end{tabular}\\\\\\

\begin{Thm}
\textbf{Double Negation Laws} $\neg(\neg P) = P$. 
\end{Thm}

\begin{Thm}
\textbf{Commutative Laws} 
\[P \vee Q \Leftrightarrow Q \vee P\]
\[P \wedge Q \Leftrightarrow Q \wedge P\]
\[P \oplus Q \Leftrightarrow Q \oplus P\]
\end{Thm}

\begin{Thm}
\textbf{Distributive Laws} 
\[P \vee ( Q \wedge R) \Rightarrow (P \vee Q) \wedge (P \vee R)\]
\[(P \wedge Q) \wedge R \Leftrightarrow P \wedge (Q \wedge R)\]
\[(P \oplus Q) \oplus R \Leftrightarrow P \oplus(Q \oplus R)\]
\end{Thm}

\begin{tabular}{ccc|cc|ccc|ccc}
P & Q & R & $P \wedge $ & $(Q \vee R)$ & $\Rightarrow$ & $\Leftrightarrow$ & $\Leftarrow$ & $ (P \wedge Q) $ & $\vee$ & $(P \wedge R)$ \\
\hline
T & T & T & T & T & T & T & T & T & T & T\\
T & T & F & T & T & T & T & T & T & T & F\\
T & F & T & T & T & T & T & T & F & T & T\\
T & F & F & F & F & T & T & T & F & F & F\\

F & T & T & F & T & T & T & T & F & F & F\\
F & T & F & F & T & T & T & T & F & F & F\\
F & F & T & F & T & T & T & T & F & F & F\\
F & F & F & F & F & T & T & T & F & F & F\\
\end{tabular}\\\\\

\begin{Q}
Suppose P = "I am a C.S. student" and Q = "I study CSE 2321", write the English statement for the following: 

\begin{itemize}
\item If I am a CS student, then I study CSE 2321. 
\item If I am a CS student, I study CSE 2321. 
\item I am a CS student implies I study CSE 2321. 
\item Whenever I am a CS student, I study CSE 2321. 
\item I am a CS student is sufficient for studying CSE 2321. 
\item I am not a CS student unless I study CSE 2321. 
\item I am a CS student only if I study CSE 2321. 
\item I study CSE 2321 if I am a CS student. 
\item I study CSE 2321 whenever I am a CS student. 
\item Studying CSE 2321 is necessary for being a CS student. 
\end{itemize}
\end{Q}

\subsection*{Lecture: Friday, September 1st}

$D = \{cats\}$. \\
$P(x) = $ "x has a tail"\\
$Q(x) = $ "x has fur"\\\\
Some cats don't have fur. \\
$\exists x \in D$ s.t. $\neg Q(x)$. \\
None of the cats have tails. \\
$\neq P(x)$ $\forall x \in D$. \\
Some cats have fur and some cats don't have tails. \\
$\exists x \in D$ s.t. $Q(x)$ $\wedge$ $\exists y \in D$ s.t. $\neq P(y)$. 

\subsection*{Lecture: Wednesday, October 4th}

\begin{Def}
Given a graph $G = (V,E)$, a \textbf{spanning tree} of $G$ is $G' = (V,E')$ where $E' \subset E$ and $G'$ is a tree. 
\end{Def}

If complete bipartite, and the partition sizes differ by one, then it has a hamiltonian path which begins in the larger one. 

If not complete bipartite, and partition sizes differ by one, no hamiltonian cycle, possible hamiltonian path. 

If partitions equal, not complete, then cycle or path possible. 

If partitions differ by more than one in any case (complete or not) then it has neither. 









\end{document}


