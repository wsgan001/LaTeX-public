
%	options include 12pt or 11pt or 10pt
%	classes include article, report, book, letter, thesis

\title{\Huge Brendan Whitaker}



\author{\huge CSE 2221 Homework 4 \linebreak \\\\ \Large Professor Bucci}

\date{1/24/2017}
\documentclass[10pt]{article}
\usepackage{tikz}
\usepackage{graphicx}
\usepackage{cancel}
\usepackage{amsmath}
\usepackage{listings}
\usepackage[margin=1in]{geometry}


\begin{document}
\maketitle




\paragraph{1. } { }
Implementation of the pointIsInCircle method: 
\begin{lstlisting}
    private static boolean pointIsInCircle(double xCoord, double yCoord) {
        double x = xCoord;
        double y = yCoord;
        boolean inCircle = false;
        if ((((x - 1.0) * (x - 1.0)) + ((y - 1.0) * (y - 1.0))) < 1.0) {
            inCircle = true;
        }
        return inCircle;
    }
\end{lstlisting}

\paragraph{2. } { }
Implementation of the numberOfPointsInCircle method: 
\begin{lstlisting}
    private static int numberOfPointsInCircle(int n) {
        int ptsInInterval = 0, ptsInSubinterval = 0;
        Random rnd = new Random1L();
        while (ptsInInterval < n) {

            double r = rnd.nextDouble();
            double x = 2 * r;
            r = rnd.nextDouble();
            double y = 2 * r;

            ptsInInterval++;
            if (pointIsInCircle(x, y) == true) {
                ptsInSubinterval++;
            }
        }
        return ptsInSubinterval;

    }
\end{lstlisting}

\paragraph{3. } { }
Implementation of the main method of the MonteCarlo lab using the two above methods: 
\begin{lstlisting}
    public static void main(String[] args) {
        /*
         * Open input and output streams
         */
        SimpleReader input = new SimpleReader1L();
        SimpleWriter output = new SimpleWriter1L();
        /*
         * Ask user for number of points to generate
         */
        output.print("Number of points: ");
        int n = input.nextInteger();
        int ptsInCircle = numberOfPointsInCircle(n);

        /*
         * Estimate proportion of points generated in [0.0,2.0)x[0.0,2.0)
         * interval that fall in the set {(x,y) in R:x^2 + y^2 < 1}
         */
        double proportion = (double) ptsInCircle / n;
        double areaOfCircle = 4.0 * proportion;
        output.println("Estimate for the area of a circle of radius r=1: "
                + areaOfCircle);

        /*
         * Close input and output streams
         */
        input.close();
        output.close();
    }
\end{lstlisting}







\end{document}


