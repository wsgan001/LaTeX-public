

%	options include 12pt or 11pt or 10pt
%	classes include article, report, book, letter, thesis

\title{Math 5590H Bonus}



\author{Brendan Whitaker}

\date{AU17}
\documentclass[10pt,oneside,reqno]{amsart}

\usepackage{graphicx}
\usepackage[margin=1in]{geometry}
\usepackage{amsmath}
\usepackage{amssymb}
\usepackage{amsthm}
\usepackage{bbm}
\usepackage{cancel}
\usepackage{verbatim}
\usepackage{amsrefs}
\usepackage{enumitem}
\usepackage{etoolbox}% http://ctan.org/pkg/etoolbox
\patchcmd{\thmhead}{(#3)}{#3}{}{}
\usepackage{braket}


\theoremstyle{plain}
\newtheorem{Thm}{Theorem}
\newtheorem{Cor}[Thm]{Corollary}
\newtheorem{Prop}[Thm]{Proposition}
\newtheorem{Lem}[Thm]{Lemma}
\newtheorem{Prob}[Thm]{Problem}
\newtheorem{Def}[Thm]{Definition}
\newtheorem{Q}[Thm]{Question}
\newtheorem*{e}{Exercise}
\newtheorem{ee}{Exercise}
\theoremstyle{definition}
\newtheorem{Remark}[Thm]{Remark}
\newtheorem{Tech}[Thm]{Technical Remark}
\newtheorem*{Claim}{Claim}
\newtheorem{Ex}[Thm]{Example}




\newcommand{\Mod}[1]{\ (\mathrm{mod}\ #1)}
\newcommand{\norm}{\trianglelefteq}
\newcommand{\propnorm}{\triangleleft}



\begin{document}

\title{Math 5590H Homework 11}

\date{AU17}

\author[Brendan Whitaker]{Brendan Whitaker}

\maketitle

\begin{e}[\textbf{8.2.5}]
Let $R$ be the quadratic integer ring $\mathbb{Z}[\sqrt{-5}]$. Define the ideals $I_2 = (2,1 + \sqrt{-5})$, $I_3 = (3,2 + \sqrt{-5})$, and $I_3' = (3,2 - \sqrt{-5})$. 
\end{e}
\begin{enumerate}
\item[]
\begin{enumerate}
\item \textit{Prove that these are all nonprincipal ideals in $R$. }
\begin{proof}
Suppose $I_2 = (a + b\sqrt{-5})$, with $a,b \in \mathbb{Z}$ was principal. Then $\exists \alpha,\beta \in R$ such that 
\begin{equation}
\begin{aligned}
2 &= \alpha(a + b\sqrt{-5}),\\
1 + \sqrt{-5} &= \beta(a + b\sqrt{-5}).
\end{aligned}
\end{equation}
 Then we have
 \begin{equation}
\begin{aligned}
4 &= N(\alpha)(a^2 + 5b^2),\\
6 &= N(\beta)(a^2 + 5b^2).
\end{aligned}
\end{equation}
Then $(a^2 + 5b^2) = 1,2$, or $4$. It cannot be $4$, since there is no integer value for $N(\beta)$ s.t. $6 = 4N(\beta)$. It cannot be $2$ since there are no integer solutions to $2 = a^2 + 5b^2$. And if it is $1$, then we must have $a = \pm 1$, and $b =0$, so $I_2 = (\pm 1) = R$. Then $1$ is in $I_2$, so $\exists \gamma,\delta \in R$ s.t. $2\gamma + \delta(1 + \sqrt{-5}) = 1$. But that would give us
\begin{equation}
\begin{aligned}
2\gamma (1- \sqrt{-5}) + 6\delta &= 1 - \sqrt{-5},\\
2(\gamma (1- \sqrt{-5} + 3\delta) &= 1 - \sqrt{-5},\\
\end{aligned}
\end{equation}
which is impossible, since $(1 - \sqrt{-5})$ is not divisible by $2$. Thus $I_2$ cannot be principal. 
\end{proof}
\vspace{3mm}
We make a similar argument for $I_3$. 
\begin{proof}
Suppose $I_3 = (a + b\sqrt{-5})$, with $a,b \in \mathbb{Z}$ was principal. Then $\exists \alpha,\beta \in R$ such that 
\begin{equation}
\begin{aligned}
3 &= \alpha(a + b\sqrt{-5}),\\
2 + \sqrt{-5} &= \beta(a + b\sqrt{-5}).
\end{aligned}
\end{equation}
 Then we have
 \begin{equation}
\begin{aligned}
9 &= N(\alpha)(a^2 + 5b^2),\\
9 &= N(\beta)(a^2 + 5b^2).
\end{aligned}
\end{equation}
So $a^2 + 5b^2 = $  1, 3, or 9. If it is $9$, then then $N(\alpha) = 1$, and thus $\alpha = \pm 1$, and then $(a + b\sqrt{-5}) = \pm 3$, which is impossible, since $2 + \sqrt{-5}$ is not divisible by $3$. If $a^2 + 5b^2 = 3$, then since there are not integer solutions to $a^2 + 5b^2 = 3$, we have a contradiction. And if $a^2 + 5b^2 = 1$, then we have that $(a + b\sqrt{-5}) = \pm 1$, so $I_3 = (\pm 1) = R$. Then we must have $\delta,\gamma \in R$ s.t. $3\gamma + \delta(2 + \sqrt{-5}) = 1$. But then we have
\begin{equation} \label{eq6}
\begin{aligned}
3\gamma (2 - \sqrt{-5}) + 9\delta &= (2 - \sqrt{-5}),\\
3(\gamma (2 - \sqrt{-5}) + 3\delta) &= (2 - \sqrt{-5}),
\end{aligned}
\end{equation}
which is impossible because $ (2 - \sqrt{-5})$ is not divisible by $3$. Thus $I_3$ cannot be principal. 
\end{proof}
\vspace{3mm}
Again, we make a similar argument for $I_3'$. 
\begin{proof}
Following precisely the same argument as in the above proof for $I_3$, but multiplying instead by $(2 + \sqrt{-5})$ in equation \eqref{eq6}, we have that $I_3'$ cannot be principal. 
\end{proof}
\vspace{3mm}
\item \textit{Prove that the product of two nonprincipal ideals can be principal by showing that $I_2^2 = (2)$ in $R$. }

\begin{proof}
Let $\alpha,\beta$ be arbitrary elements of $I_2$, where $\alpha = 2a + (1 + \sqrt{-5})b$, and $\beta = 2c + (1 + \sqrt{-5})d$, for $a,b,c,d \in \mathbb{Z}$.  Then any element of $I_2^2$ is of the form 
\begin{equation} 
\begin{aligned}
\alpha\beta &= (2a + (1 + \sqrt{-5})b)(2c + (1 + \sqrt{-5})d)\\
&= 4ac + 2ad(1 + \sqrt{-5}) + 2cb(1 + \sqrt{-5}) +  bd(1 + 2\sqrt{-5} + -5)\\
&= 4ac + 2ad + 2ad\sqrt{-5} + 2cb + 2cb\sqrt{-5}  + 2bd\sqrt{-5} - 4bd\\
&= 2(2ac + ad + ad\sqrt{-5} + cb + cb\sqrt{-5}  + bd\sqrt{-5} - 2bd)\\
&= 2((2ac + ad + cb -2bd) + (ad + cb + bd)\sqrt{-5}),
\end{aligned}
\end{equation}

and since $a,b,c,d$ are integers, we know that $\alpha\beta$ is of the form $2r$ for $r \in R$. Thus $I_2^2 = (2)$, since for appropriate choice of $a,b,c,d$ we may let $r$ be any element of $R$. Hence $I_2^2$ is a principal ideal. 
\end{proof}

\item \textit{Prove similarly that $I_2I_3 = (1 - \sqrt{-5})$, and $I_2I_3' = (1 + \sqrt{-5})$. Conclude that the principal ideal $(6)$ is the product of 4 ideals: $(6) = I_2^2I_3I_3'$. }

\begin{proof}
We show that $1 - \sqrt{-5}$ can be written as the product of the generators of $I_2$ and $I_3$, respectively, to show $(1 - \sqrt{-5}) \subset I_2I_3$, and we show each of the generators of $I_2$ and $I_3$ are generated by $1 - \sqrt{-5}$ to show $I_2I_3 \subset (1 - \sqrt{-5})$. Note 

\begin{equation} 
\begin{aligned}
1 - \sqrt{-5} &= 3 - (2 + \sqrt{-5}),
\end{aligned}
\end{equation}
so $(1 - \sqrt{-5}) \subset I_2I_3$, let $\alpha = 1 - \sqrt{-5}$, and also note 
\begin{equation} 
\begin{aligned}
I_2I_3 &= (2,1 + \sqrt{-5})(3, 2 + \sqrt{-5}) = (6, 4 + 2\sqrt{-5},3 + 3\sqrt{-5}, -3 + 3\sqrt{-5}),\\
6 &= \alpha \overline{\alpha},\\
4 + 2\sqrt{-5} &= -\alpha^2,\\
3 + 3\sqrt{-5} &= \alpha(-2 + \sqrt{-5}),\\
-3 + 3\sqrt{-5} &= -3\alpha.
\end{aligned}
\end{equation}
Thus each of the generators of $I_2I_3$ is in $(1 - \sqrt{-5})$, so $I_2I_3 \subset (1 - \sqrt{-5}) \Rightarrow I_2I_3 = (1 - \sqrt{-5})$. The fact that $I_2I_3' = (1 + \sqrt{-5})$ follows from precisely the same argument by taking complex conjugates. Now $I_2^2I_3I_3' = I_2I_3 \cdot I_2I_3' = (1 - \sqrt{-5})(1 + \sqrt{-5}) = (6)$. 
\end{proof}

\end{enumerate}

\end{enumerate}
\vspace{3mm}

\begin{e}[\textbf{8.3.8}]
Let $R, I_2, I_3, I_3'$ be as defined in Exercise 8.2.5. Again, let $\alpha = 1 - \sqrt{-5}$. 
\end{e}

\begin{enumerate}
\item[]
\begin{enumerate}
\item \textit{Prove that $2,3,\alpha, \overline{\alpha}$ are all irreducibles in $R$, none of which are associate, and that $6 = 2 \cdot 3 = \cdot \alpha \overline{\alpha}$ are two distinct factorizations of $6$ into irreducibles in $R$. }
\begin{proof}
We use the fact that $R$ is an integral domain. Let $2 =r(a + b\sqrt{-5})$, where $r,(a + b\sqrt{-5}) \in R$. Then taking the associated field norm, we have
\begin{equation} 
\begin{aligned}
4 = N(r)(a^2 + 5b^2),
\end{aligned}
\end{equation}
and since $a^2 + 5b^2$ is a positive integer, it must be $1,2$, or $4$. If it is 4, then $N(r) = 1 \Rightarrow r = \pm 1 \Rightarrow r$ is a unit, so in this case 2 is irreducible. Suppose $a^2 + 5b^2 = 2$. This is impossible as the equation is insoluble in integers. So let $a^2 + 5b^2= 1$. Then $a = \pm 1$, $ b= 0$, and thus $(a + b\sqrt{-5})$ is a unit, so again $2$ is irreducible. Note that $3$ is irreducible by precisely the same argument, using the equation $9 = N(r)(a^2 + 5b^2)$, since $3 = a^2 + 5b^2$ is not soluble in integers, and the factors of $9$ are $1,3,9$. \\

Now let $\alpha = r(a + b\sqrt{-5})$. Taking norms we have
\begin{equation} 
\begin{aligned}
6 = N(r)(a^2 + 5b^2),
\end{aligned}
\end{equation}
where the possible values for $a^2 + 5b^2$ are 1, 2, 3, or 6. We immediately have that $\alpha$ is irreducible in the case where the value is 1, since then $(a + b\sqrt{-5}) = \pm 1$, a unit, or $6$, since then $r$ is a unit. And the other cases, where $a^2 + 5b^2$ is 2 or 3, follow directly from the insolubility in integers of the equations mentioned above. Thus $\alpha$ is irreducible, and $\overline{\alpha}$'s irreducibility follows from precisely the same argument, since $\alpha,\overline{\alpha}$ have the same norm. Note 2 and 3 could not possibly be associates with each other or the other two elements in question, since they differ in norm. It remains to be shown that $\alpha,\overline{\alpha}$ are not associate. Since units in $R$ must have norm 1, and this implies $b = 0$ for any element of the form $a + b\sqrt{-5}$, we know all units $\pm 1$. And clearly $-\alpha \neq \overline{\alpha}$, so they are not associate. 
\end{proof}

\item \textit{Prove that $I_2,I_3,I_3'$ are prime ideals. }

\begin{proof}
Note we proved these are all nonprincipal ideals in a previous exercise. Let $a + b\sqrt{-5} \in R$, then we have
\begin{equation} 
\begin{aligned}
a + b\sqrt{-5} \equiv a - b \equiv 0 \text{ or } 1 \mod I_2
\end{aligned}
\end{equation}
since $1 + \sqrt{-5} \equiv 0 \mod I_2$, and $2 \equiv 0 \mod I_2$. So we have at most 2 elements. Thus we must have $R/I_2 \cong \mathbb{F}_2$. And since $\mathbb{F}_2$ is an integral domain, we have that $I_2$ must be prime, since if $R$ is commutative, then $I$ is prime if and only if $R/I$ is an integral domain. Similarly, we have
\begin{equation} 
\begin{aligned}
a + b\sqrt{-5} \equiv a - 2b \equiv 0,1 \text{ or } 2 \mod I_3
\end{aligned}
\end{equation}
So we must have $R/I_3 \cong \mathbb{F}_3$, since our quotient ring can have at most 3 elements, and we get all three by appropriate choices of $a,b$. Hence again, since $\mathbb{F}_3$ is an integral domain, we have that $I_3$ must be a prime ideal. And $I_3'$ is a prime ideal by the same reasoning, since the only thing that changes is that we have $a + 2b \equiv 0,1 \text{ or } 2 \mod I_3'$, which again gives us $\mathbb{F}_3$ since $a + 2b \in \mathbb{Z}$. 
\end{proof}


\item \textit{Show that the factorizations in (a) imply the equality of the ideals $(6) = (2)(3)$, and $(6) = (1 + \sqrt{-5})(1 - \sqrt{-5}) = (\alpha)(\overline{\alpha})$. }

\begin{proof}
By multiplication of principle ideals, we know
\begin{equation} 
\begin{aligned}
(6) = (2)(3) = (\alpha)(\overline{\alpha}).
\end{aligned}
\end{equation}
Also, note
\begin{equation} 
\begin{aligned}
I_3I_3' = (3,2 + \sqrt{-5})(3, 2 - \sqrt{-5}) = (9, 6 - 3\sqrt{-5}, 6 + 3\sqrt{-5}). 
\end{aligned}
\end{equation}
So $I_3I_3' \subset (3)$ since 3 divides all the above generators. Also
\begin{equation} 
\begin{aligned}
3 = 9 + 6 - 3\sqrt{-5} + 6 + 3\sqrt{-5},
\end{aligned}
\end{equation}
so $(3) \subset I_3I_3'$, and so $(3) = I_3I_3'$. And so, using the results of the previous exercise, we have
\begin{equation} 
\begin{aligned}
(6) = (2)(3) = (I_2^2)(I_3I_3') = (\alpha)(\overline{\alpha}) = (I_2I_3)(I_2I_3'),
\end{aligned}
\end{equation}
and thus the factorization of the ideals is unique. 
\end{proof}


\end{enumerate}

\end{enumerate}





\end{document}



