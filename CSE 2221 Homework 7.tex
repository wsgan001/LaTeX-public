
%	options include 12pt or 11pt or 10pt
%	classes include article, report, book, letter, thesis

\title{\Huge Brendan Whitaker}



\author{\huge CSE 2221 Homework 7 \linebreak \\\\ \Large Professor Bucci}

\date{7 February 2017}
\documentclass[10pt]{article}
\usepackage{tikz}
\usepackage{graphicx}
\usepackage{cancel}
\usepackage{amsmath}
\usepackage{listings}
%\usepackage[margin=1in]{geometry}&


\begin{document}
\maketitle


\paragraph{1. }
\begin{list}{} {}
\item {a. }
Prints only the uppercase letters in the String: 
\begin{lstlisting}
    out.print("Enter a string: ");
    String input = in.nextLine();
    out.print("The uppercase letters in the string are: ");
    for (int i = 0; i < input.length(); i++) {
        if (Character.isUpperCase(input.charAt(i))) {
            out.print(input.charAt(i));
        }
    }
\end{lstlisting}

\item {b. }
Prints every second letter of the String: 
\begin{lstlisting}
    out.print("Enter a string: ");
    String input = in.nextLine();
    out.print("Every second letter: ");
    for (int i = 0; i < input.length(); i++) {
        if (i % 2 == 1) {
            out.print(input.charAt(i));
        }
    }
\end{lstlisting}
\item {c. } Prints the String with all vowels replaced by an underscore: 
\begin{lstlisting}
    char[] vowels = { 'a', 'e', 'i', 'o', 'u' };
    int numVowels;
    out.print("Enter a string: ");
    String input = in.nextLine();
    out.print("All the vowels in the string: ");
    for (int i = 0; i < input.length(); i++) {
        numVowels = 0;
        for (int j = 0; j < vowels.length; j++) {
            if ((input.charAt(i)) == vowels[j]) {
                numVowels++;
                out.print("_");
            }
        }
        if (numVowels == 0) {
            out.print(input.charAt(i));
        }
    }
\end{lstlisting}

\item {d. } Prints the number of vowels in the String: 
\begin{lstlisting}
    char[] vowels = { 'a', 'e', 'i', 'o', 'u' };
    int numVowels = 0;
    out.print("Enter a string: ");
    String input = in.nextLine();
    for (int i = 0; i < input.length(); i++) {
        for (int j = 0; j < vowels.length; j++) {
            if ((input.charAt(i)) == vowels[j]) {
                numVowels++;
            }
        }
    }
    out.println("The number of vowels is: " + numVowels);
\end{lstlisting}

\item {e. } Prints the position of all vowels in the String: 
\begin{lstlisting}
    char[] vowels = { 'a', 'e', 'i', 'o', 'u' };
    out.print("Enter a string: ");
    String input = in.nextLine();
    for (int i = 0; i < input.length(); i++) {
        for (int j = 0; j < vowels.length; j++) {
            if ((input.charAt(i)) == vowels[j]) {
                out.println("The vowel " + input.charAt(i)
                        + " is at position: " + i);
            }
        }
    }
\end{lstlisting}
\end{list}

\paragraph{2. }
\begin{list}{} {}
\item {a. }
\begin{lstlisting}
	[1, 1, 1, 1, 1, 1, 1, 1, 1, 1]
\end{lstlisting}

\item {b. }
\begin{lstlisting}
	[1, 1, 2, 3, 4, 5, 4, 3, 2, 1]
\end{lstlisting}

\item {c. }
\begin{lstlisting}
	[2, 3, 4, 5, 4, 3, 2, 1, 0, 0]
\end{lstlisting}

\item {d. }
\begin{lstlisting}
	[0, 0, 0, 0, 0, 0, 0, 0, 0, 0]
\end{lstlisting}

\item {e. }
\begin{lstlisting}
	[1, 3, 6, 10, 15, 19, 22, 24, 25, 25]
\end{lstlisting}

\item {f. }
\begin{lstlisting}
	[1, 0, 3, 0, 5, 0, 3, 0, 1, 0]
\end{lstlisting}

\item {g. }
\begin{lstlisting}
	[1, 2, 3, 4, 5, 1, 2, 3, 4, 5]
\end{lstlisting}

\item {h. }
\begin{lstlisting}
	[1, 1, 2, 3, 4, 4, 3, 2, 1, 0]
\end{lstlisting}

\end{list}

\paragraph{3. }  Java code for a loop that simultaneously computes both the maximum and minimum of an array of ints called a: 
\begin{lstlisting}
	//initializes to lowest/highest possible 
	//integer such that the algorithm
    //functions for all possible integer 
    //values in the array. 
    int max = -2147483648;
    int min = 2147483647;
    for (int i = 0; i < a.length; i++) {
        if (a[i] <= min) {
            min = a[i];
        }
        if (a[i] >= max) {
            max = a[i];
        }
    }
    out.println("min is: " + min);
    out.println("max is: " + max);
\end{lstlisting}

\paragraph{4. }  Java code for a loop that sets boolean variable isOrdered to true if the elements of a given array of ints called a are in non-decreasing order, otherwise it sets isOrdered to false: 
\begin{lstlisting}
    int currentMax = a[0];
    boolean isOrdered = true;
    int i = 0;
    while ((i < a.length) && (a[i] >= currentMax)) {
        currentMax = a[i];
        i++;
    }
    if (i < a.length) {
        isOrdered = false;
    }
\end{lstlisting}

\paragraph{5. }  
\begin{list}{} {}
\item {a. }
\vspace{120mm}
\item {b. }
\begin{list}{} {}

\item {i. }
\begin{lstlisting}
    true
\end{lstlisting}

\item {ii. }
\begin{lstlisting}
    breakfast_menu
\end{lstlisting}

\item {iii. }
\begin{lstlisting}
    4
\end{lstlisting}

\item {iv. }
\begin{lstlisting}
    false
\end{lstlisting}

\item {v. }
\begin{lstlisting}
	<food calories="630">
	  <name>Blueberry Pancakes</name>
	  <price>$4.95</price>
	</food>
\end{lstlisting}

\item {vi. }
\begin{lstlisting}
	true
\end{lstlisting}

\item {vii. }
\begin{lstlisting}
	food
\end{lstlisting}

\item {viii. }
\begin{lstlisting}
	2
\end{lstlisting}

\item {ix. }
\begin{lstlisting}
	true
\end{lstlisting}

\item {x. }
\begin{lstlisting}
	630
\end{lstlisting}

\item {xi. }
\begin{lstlisting}
	true
\end{lstlisting}

\item {xii. }
\begin{lstlisting}
	false
\end{lstlisting}

\item {xiii. }
\begin{lstlisting}
	Blueberry Pancakes
\end{lstlisting}
\end{list}

\item {c. }
\begin{lstlisting}
	<breakfast_menu>
	  <food calories="650">
	    <name price="$5.95">Belgian Waffles</name>
	  </food>
	  <food calories="630">
	    <name price="$4.95">Blueberry Pancakes</name>
	  </food>
	  <food calories="600">
	    <name price="$4.50">French Toast</name>
	  </food>
	  <food calories="950">
	    <name price="$6.95">Homestyle Breakfast</name>
	  </food>
	</breakfast_menu>
\end{lstlisting}

\item {d. }
\vspace{120mm}
\end{list}



\end{document}


