

%	options include 12pt or 11pt or 10pt
%	classes include article, report, book, letter, thesis

\title{Math 5590H Homework 6}



\author{Brendan Whitaker}

\date{AU17}
\documentclass[10pt,oneside,reqno]{amsart}

\usepackage{graphicx}
\usepackage[margin=1in]{geometry}
\usepackage{amsmath}
\usepackage{amssymb}
\usepackage{amsthm}
\usepackage{bbm}
\usepackage{cancel}
\usepackage{verbatim}
\usepackage{amsrefs}
\usepackage{enumitem}
\usepackage{etoolbox}% http://ctan.org/pkg/etoolbox
\patchcmd{\thmhead}{(#3)}{#3}{}{}
\usepackage{braket}


\theoremstyle{plain}
\newtheorem{Thm}{Theorem}
\newtheorem{Cor}[Thm]{Corollary}
\newtheorem{Prop}[Thm]{Proposition}
\newtheorem{Lem}[Thm]{Lemma}
\newtheorem{Prob}[Thm]{Problem}
\newtheorem{Def}[Thm]{Definition}
\newtheorem{Q}[Thm]{Question}
\newtheorem*{e}{Exercise}
\newtheorem{ee}{Exercise}
\theoremstyle{definition}
\newtheorem{Remark}[Thm]{Remark}
\newtheorem{Tech}[Thm]{Technical Remark}
\newtheorem*{Claim}{Claim}
\newtheorem{Ex}[Thm]{Example}




\newcommand{\Mod}[1]{\ (\mathrm{mod}\ #1)}
\newcommand{\norm}{\trianglelefteq}
\newcommand{\propnorm}{\triangleleft}



\begin{document}

\title{CSE 6331 Homework 1}

\date{SP18}

\author[Brendan Whitaker]{Brendan Whitaker}

\maketitle

\begin{enumerate}[label=\arabic*.]
\item \textit{Order the given functions by asymptotic dominance. That is, produce an order $f_1(n),f_2(n),...$ such that $f_i(n) = O(f_{i + 1}(n)$. }


\begin{equation}
\begin{aligned}
1/n^2, &2^{23}, \log_2 \log_2(n^5 + n^2), 27 \log_7(n) + \sqrt{\log_2(n)}, (log_2n)^3,\\ &3n^{2.1} + 19n^{1.5} + \sqrt{n},
5^{log_2n} + \sqrt{n},  4^{\sqrt{n}}, n!n^2, n^{2n}, 2^{2^n}. 
\end{aligned}
\end{equation}

\begin{equation}
\begin{aligned}
c,b,f,h,i,g,j,d,a,e,k
\end{aligned}
\end{equation}

\item \textit{Let $f(n)$ be a function defined for $\mathbb{N}$. Prove or disprove: If $f(n) \in \Theta(n^2)$, then $f(n)$ is asymptotically, monotonically non-decreasing, i.e. $f(n) \leq f(n + 1)$ for all sufficiently large $n$. }

It is false, we can construct a piecewise function by for odd $n$ taking $f(n) = n^2$ and taking $f(n) = f(n - 1)$ for even $n$. Multiplying this by scalars gives us upper and lower bounds for $n^2$. We need to use induction to be rigorous. 

Consider the function $f:\mathbb{N} \to \mathbb{N}$ given by: 
$$
f(n) = 
\begin{cases}
n^2 & n \text{ is odd},\\
f(n - 1) - 1 & n \text{ is even}.\\
\end{cases}
$$
Note that this function is not asymptotically monotonically non-decreasing since for each odd integer $n$, $f(n + 1) = f(n) - 1 < f(n)$. We prove that $f \in \Theta(n^2)$. Define $g(n)  = n^2$. 

We first prove that $f(n) \in O(n^2)$. Note that $\frac{1}{2}f(4) =  4$, and $g(4) = 16$. Define $h(n) = \frac{1}{2}f(n)$.   We claim that $h(n) < g(n),$ $\forall n \geq 4$. We prove this by induction. We have already checked our base case and verified that $h(4) < g(4)$. We fix $n \in \mathbb{N}$, and define our induction hypothesis as $h(n) < g(n)$. We wish to show that $h(n + 1) < g(n + 1)$. 

\textbf{Case 1: }$n$ is odd. 
Observe: 
\begin{equation}
\begin{aligned}
h(n + 1) &= \frac{1}{2}f(n + 1) = \frac{1}{2}(f(n) - 1) = \frac{1}{2}f(n) - \frac{1}{2} = h(n) - \frac{1}{2}\\
&< g(n) - \frac{1}{2} < g(n + 1).
\end{aligned}
\end{equation}
In the above expression, the first equality comes from the definition of $h$, the second comes from the definition of $f$ and the fact that $n$ is odd, and the third comes from the definition of $h$ again. The inequality on the second line comes from the induction hypothesis, and since $g(n) = n^2$ is an increasing function for positive $n$, we have the final inequality and our desired result for the first case. 

\textbf{Case 2: }$n$ is even. Observe: 
\begin{equation}
\begin{aligned}
h(n + 1) &= \frac{1}{2}f(n + 1) = \frac{1}{2}(n  + 1)^2 =\frac{1}{2} g(n + 1) < g(n + 1). 
\end{aligned}
\end{equation}
In the above expression, the first equality comes from the definition of $h$, the second comes from the definition of $f$ and the fact that $n$ is even, and thus $n + 1$ is odd. The third equality comes from the definition of $g$ and the last is because $g$ is positive for $n \geq 4$. 

Hence we have the desired inequality for all $n \geq 4$, and the induction is complete. So we know $f(n) < 2n^2$ $\forall n\geq 4$, hence $f(n) \in O(n^2)$. 

Now we will show $f(n) \in \Omega(n^2)$. So we choose $c = \frac{1}{2}$, and we will show that $f(n) \geq \frac{1}{2}n^2$ for all sufficiently large $n$. Note that $f(5) = 25$. Define $g'(n) =  \frac{1}{2}n^2$. Note that $g'(5) = 25/2 < 25 = f(5)$, so $f(5) > g'(5)$. We prove by induction that $f(n) > g'(n)$ $\forall n \geq 5$. We already finished our base case, so we fix $n$, and assume for our induction hypothesis that $f(n) > g'(n)$. We show $f(n + 1) > g'(n + 1)$. 

\textbf{Case 1: } $n$ is odd. Observe: 
$$
f(n + 1) = f(n) - 1 = n^2 - 1 = (n + 1)(n - 1) > (n + 1)(n + 1) > \frac{1}{2}(n  +1)^2 = g'(n  +1). 
$$

\textbf{Case 2: }$n$ is even. Observe: 
$$
f(n + 1) = (n + 1)^2 < \frac{1}{2}(n + 1)^2 = g'(n + 1).
$$
So in both cases our inductive step holds, so we have proven that $f(n) > \frac{1}{2}n^2,$ $\forall n \geq 5$. So $f(n) \in Omega(n^2)$, hence we know $f(n) \in \Theta(n^2)$. But since $f$ is not asymptotically monotonically non-decreasing, we have found a valid counterexample. 
\item \textit{Let $f(n)$ be defined on $\mathbb{N}$. Prove or disprove the following statement: if $f(n) \in O(g(n))$, then $2^{f(n)} \in O(2^{g(n)})$. }

Let $f(n) = 2n$, and let $g(n) = n$. Then $f(n) \in O(g(n))$, since $f(n) = 2n \leq 3g(n) = 3n$ $\forall n$. But note that $2^{2n} \notin O(2^{n})$, since $2^{2n} = (2^n)^2 = 4^n$, and we know $4^n \notin O(2^n)$, since $2 < 4$. 

\item \textit{Find how many dollar signs the given procedure will print. }

Note that the inner while loop executes $\log_2(l)$ times where $2^l = n$. So we have that the inner while loop takes $\log\log(n)$ time. The outer for loop executes $\log_2(n)$ times. Hence our running time is given by: 
$$
T(n) = \Theta(\log(n)\log(\log(n))). 
$$


\end{enumerate}








\end{document}


