

%	options include 12pt or 11pt or 10pt
%	classes include article, report, book, letter, thesis

\title{Math 5590H Bonus}



\author{Brendan Whitaker}

\date{AU17}
\documentclass[10pt,oneside,reqno]{amsart}

\usepackage{graphicx}
\usepackage[margin=1in]{geometry}
\usepackage{amsmath}
\usepackage{amssymb}
\usepackage{amsthm}
\usepackage{bbm}
\usepackage{cancel}
\usepackage{verbatim}
\usepackage{amsrefs}
\usepackage{enumitem}
\usepackage{etoolbox}% http://ctan.org/pkg/etoolbox
\patchcmd{\thmhead}{(#3)}{#3}{}{}
\usepackage{braket}


\theoremstyle{plain}
\newtheorem{Thm}{Theorem}
\newtheorem{Cor}[Thm]{Corollary}
\newtheorem{Prop}[Thm]{Proposition}
\newtheorem{Lem}[Thm]{Lemma}
\newtheorem{Prob}[Thm]{Problem}
\newtheorem{Def}[Thm]{Definition}
\newtheorem{Q}[Thm]{Question}
\newtheorem*{e}{Exercise}
\newtheorem{ee}{Exercise}
\theoremstyle{definition}
\newtheorem{Remark}[Thm]{Remark}
\newtheorem{Tech}[Thm]{Technical Remark}
\newtheorem*{Claim}{Claim}
\newtheorem{Ex}[Thm]{Example}




\newcommand{\Mod}[1]{\ (\mathrm{mod}\ #1)}
\newcommand{\norm}{\trianglelefteq}
\newcommand{\propnorm}{\triangleleft}



\begin{document}

\title{Math 5590H Homework 9}

\date{AU17}

\author[Brendan Whitaker]{Brendan Whitaker}

\maketitle

\begin{e}[\textbf{7.1.14}]
Let $x$ be a nilpotent element of the commmutative ring $R$. 
\end{e}
\begin{enumerate}
\item[]
\begin{enumerate}
\item \textit{Prove that $x$ is either zero or a zero divisor. }
\begin{proof}
\vspace{3mm}
Since $x$ is nilpotent, we know $\exists m \in \mathbb{Z}^+$ s.t. $x^m = 0$. If $x$ is zero, we are done, so assume $x \neq 0$. We also assume $m$ is the minimal such positive integer, since $0 \cdot 0 = 0$. Then we also know $x^{m - 1} \neq 0$. So we have two nonzero elements, $x,x^{m - 1}$, such that $xx^{m - 1} = x^{m} = 0 \Rightarrow$ $x$ is a zero divisor. 
\end{proof}
\vspace{3mm}
\item 
\textit{Prove that $rx$ is nilpotent $\forall r \in R$. }

\begin{proof}
\vspace{3mm}
Again we assume $x^m = 0$, where $m$ is the least such positive integer where this holds true. Then we have
\[(rx)^m = r^mx^m = r^m \cdot 0 = 0,\]
since $R$ is commutative. 
\end{proof}
\vspace{3mm}
\item \textit{Prove that $1+ x$ is a unit in $R$. }
\begin{proof}
\vspace{3mm}
Note that $x^m + 1 = 1$. Then observe \[(1 + x)(x^{m - 1} - x^{m - 2} + \cdots + (-1)^{m - 2}x + (-1)^{m - 1}) = x^m + 1 = 1.\]
Thus $(1 + x)$ has a multiplicative inverse and hence is a unit. 
\end{proof}
\vspace{3mm}
\item \textit{Prove that the sum of a nilpotent element and a unit is a unit. }

\begin{proof}
\vspace{3mm}
Let $x$ be nilpotent, so $x^m = 0$, where $m$ is the minimial positive integer such that this is true, and let $y$ be a unit. Then $\exists y^{-1} \in R$ s.t. $yy^{-1} = 1$. Suppose for contradiction that $x + y$ is not a unit. Then it has no multiplicative inverse, so $\forall z \in R$, $z(x + y) = r$ for some $r \in R$ s.t. $r \neq 1$. So let $z = y^{-1}$. Then we have
\[y^{-1}(x + y) = r\]
for some $r \in R$ s.t. $r \neq 1$. Thus
\[y^{-1}x + y^{-1}y = r.\]
We multiply by $x^{m - 1}$ on both sides
\begin{equation}
\begin{aligned}
x^{m - 1}xy^{-1} + x^{m - 1}yy^{-1} &= x^{m-1}r,\\
x^{m - 1} &= x^{m - 1}r.
\end{aligned}
\end{equation}
Thus we must have that $r = 1$, and this is contradiction, hence $x + y$ must be a unit. 
\end{proof}

\end{enumerate}





\end{enumerate}

\begin{e}[\textbf{7.1.15}]
Prove that every boolean ring is commutative. 
\end{e}

\begin{proof}
Let $R$ be a boolean ring. Suppose $a,b \in R$ s.t. $a,b \neq 0$, $a \neq b$. Then observe
\begin{equation}
\begin{aligned}
a + b = (a + b)^2 = a^2 + ab + ba + b^2 &= a + ab + ba + b\\
0 &= ab + ba\\
-ba &= ab
\end{aligned}
\end{equation}
Also note for $a \neq 0$,
\[(-a) = (-a)(-a) = -(a)(-a) = -(-a^2) = a^2 = a,\]
which gives us $ab = ba$, hence $R$ is commutative. 
\end{proof}
\vspace{3mm}
\begin{e}[\textbf{7.1.21}]
Let $X$ be any nonempty set and let $\mathcal{P}(X)$ be the power set of $X$. 
\end{e}



\begin{enumerate}
\item[]
\begin{enumerate}
\item \textit{Prove $\mathcal{P}(X)$ is a ring.  }

\begin{proof}
We first prove that $\mathcal{P}(X)$ is an abelian group under the addition operation, where $\forall A,B \in \mathcal{P}(X)$, $A + B = (A \setminus B) \cup (B \setminus A)$. Let $A,B \subset X$, then $A \setminus B \subset A \subset X$ and $B \setminus A \subset B \subset X \Rightarrow A + B = (A \setminus B) \cup (B \setminus A) \subset X \Rightarrow$ we have closure under addition. Also $\varnothing$ is $0$ since $\forall A \in \mathcal{P}(X)$, we have \[\varnothing  + A = (\varnothing \setminus A) \cup (A \setminus \varnothing) = (A \setminus \varnothing)\cup(\varnothing \setminus A)  =  \varnothing \cup A = A.\]
Now $\forall A \in \mathcal{P}(X)$, we have $-A = A$, since 
\[A + A = (A \setminus A) \cup (A \setminus A) = \varnothing,\]
so we have inverses. Let $A,B,C \in \mathcal{P}(X)$. Then
\begin{equation}
\begin{aligned}
A + (B + C) &= (A \setminus ((B \setminus C) \cup (C \setminus B))) \cup (((B \setminus C) \cup (C \setminus B)) \setminus A)\\
&= ((A \setminus (B \setminus C)) \cap (A \setminus (C \setminus B))) \cup (((B \setminus C) \setminus A) \cup ((C \setminus B) \setminus A))\\
&= ((A \setminus (B \setminus C)) \cap (A \setminus (C \setminus B)))  \cup (B \setminus (C \cup A)) \cup (C \setminus (B \cup A))\\
&=(A \setminus (B \cup C)) \cup (A \cap B \cap C) \cup (B \setminus (C \cup A)) \cup (C \setminus (B \cup A))\\
(A + B) + C &= (((A \setminus B) \cup (B \setminus A))\setminus C) \cup (C \setminus ((A \setminus B) \cup (B \setminus A)))\\
&= (((A \setminus B) \setminus C) \cup ((B \setminus A) \setminus C) \cup (C \setminus(A \setminus B)) \cap (C \setminus (B \setminus A)))\\
&= (A \setminus (B \cup C)) \cup (B \setminus (A \cup C)) \cup (C \setminus (A \cup B)) \cup (A \cap B \cap C),
\end{aligned}
\end{equation}
thus we have associativity of addition. So $\mathcal{P}(X)$ is a group under addition. Additionally, 
\[A + B = (A \setminus B) \cup (B \setminus A) = (B \setminus A) \cup (A \setminus B) = (B + A),\]
because of commutativity of set unions, and thus we have that $\mathcal{P}(X)$ is abelian. We now check that multiplication, defined as $A \times B = A \cap B$, satisfies is associative and satisfies distributivity. Note associativity is immediate since set intersections are commutative, and 
\begin{equation}
\begin{aligned}
 (A + B) \times C &= ((A \setminus B) \cup (B \setminus A)) \cap C \\
 &= ((A \setminus B) \cap C) \cup ((B \setminus A) \cap C)\\
 (A \times C) + (B \times C) &= ((A \cap C) \setminus (B \cap C)) \cup ((B \cap C) \setminus (A \cap C)) \\
 &= ((A \setminus B) \cap C) \cup ((B \setminus A ) \cap C).
\end{aligned}
\end{equation}
Also $A \times (B + C) = (A \times B) + (A \times C)$ holds as well because of commutativity of set intersections. Thus $\mathcal{P}(X)$ is a ring. 
\end{proof}

\item \textit{Prove that this ring is commutative, has an identity, and is a Boolean ring. }

\begin{proof}
Recall from part $(a)$ that multiplication is commutative because set intersections are commutative. Also, $X$ is our identity since $\forall A \in \mathcal{P}(X)$, $A \subset X \Rightarrow A \times X = A \cap X = A$. Now let $A \in \mathcal{P}(X)$, then $A^2 = A \times A = A \cap A = A$, and thus $\mathcal{P}(X)$ is a Boolean ring. 
\end{proof}

\end{enumerate}
\end{enumerate}


\begin{e}[\textbf{7.2.2}]
Let $p(x) = a_nx^n + a_{n - 1}x^{n - 1} + \cdots + a_1x + a_0$ be an element of the polynomial ring $R[x]$. Prove that $p(x)$ is a zero divisor in $R[x]$ if and only if there is a nonzero $b \in R$ s.t. $bp(x) = 0$. 
\end{e}
\begin{proof}
If there exists such a $b$, then $p(x)$ must be a zero divisor, so we need only prove that if $p(x)$ is a zero divisor, then there exists such a $b$. Let $g(x) = b_mx^m + \cdots b_1x + b_0$ be a nonzero polynomial of minimal degree such that $g(x)p(x) = 0$. Then suppose $b_ma_n \neq 0$. Then we must have that $b_m \neq 0 \neq a_n$. But $g(x)p(x)$ contains the term $b_ma_nx^{n + m}$, so we must have that $g(x)p(x) \neq 0$, which is contradiction, so $b_ma_n = 0$. Thus $a_ng(x)$ is a polynomial of degree less than $m$ such that $a_ng(x)p(x) = 0$. But we said $g(x)$ was the nonzero polynomial of minimal degree such that $g(x)p(x) = 0$, so we must have that $a_ng(x) = 0$. Now we show that $a_{n - i}g(x) = 0$ for $i = 0,1,...,n$. Let $i = 0$. We already know that $a_ng(x) = 0$, so the base case holds. Suppose $a_{n - i}g(x) = 0$. We wish to prove that $a_{n - (i + 1)}g(x) = 0$. We claim $b_ma_{n - (i + 1)} = 0$. Suppose $b_ma_{n - (i + 1)} \neq 0$. Then $b_m \neq 0 \neq a_{n - (i + 1)}$. But again, since $g(x)p(x)$ contains a term $kb_ma_{n - (i + 1)}x^{m + (n - (i + 1))}$ for some nonzero $k \in R$, we must have that $g(x)p(x) \neq 0$, which again is a contradiction, so $b_ma_{n - (i  +1)} = 0$. Then suppose $a_{n - (i + 1)}g(x) \neq 0$. Then this polynomial has degree less than $m$, since $b_ma_{n - (i + 1)} = 0$ and this contradicts the definition of $g(x)$, so we must have that $a_{n - (i + 1)}g(x) = 0$. By induction, $a_{n - i}g(x) = 0$ for all $i \leq n$, and thus $b_ma_i = 0$ for all $i$, since otherwise $g(x)p(x) \neq 0$, so then we know $b_m$ is our desired element $b$. 
\end{proof}


\begin{e}[\textbf{7.2.10}]
Consider the following elements of the integral group ring $\mathbb{Z}S_3$:
\[\alpha = 3(1 \text{ }2) - 5(2 \text{ }3) + 14(1\text{ } 2\text{ } 3) \text{ }\text{ }\text{ }\text{     and     }\text{ }\text{ }\text{ } \beta = 6(1) + 2(2\text{ } 3) - 7 (1 \text{ }3\text{ } 2),\]
where $1$ is the identity of $S_3$. Compute the following elements:
\end{e}

\begin{enumerate}
\item[]
\begin{enumerate}
\item $\alpha + \beta = 6(1) - 3(2 \text{ } 3) + 14(1 \text{ }2 \text{ }3) - 7(1 \text{ }2 \text{ }3) + 3(1 \text{ } 2).$
\item \begin{equation}
\begin{aligned}
\alpha\beta &= 18(1 \text{ }2) + 6(1 \text{ }2 \text{ }3) - 21(1 \text{ }3) - 30(2 \text{ }3) - 10(1) + 35(1 \text{ }2) + 84(1 \text{ }2 \text{ }3) + 28(1 \text{ }2) - 98(1)\\
&= 81(1\text{ }2) + 90(1\text{ }2\text{ }3) - 21(1\text{ }3) - 30(2\text{ }3) - 108(1)
\end{aligned}
\end{equation}
\end{enumerate}



\end{enumerate}




\end{document}



