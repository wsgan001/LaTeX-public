

%	options include 12pt or 11pt or 10pt
%	classes include article, report, book, letter, thesis

\title{}



\author{Brendan Whitaker}


\documentclass[10pt,oneside,reqno]{amsart}

%-------------------------------------
%--------PREAMBLE---------------------

%    Include referenced packages here.
\usepackage{}
\usepackage[margin=1in]{geometry}
\usepackage{graphicx}
\usepackage[margin=1in]{geometry}
\usepackage{amsmath}
\usepackage{amssymb}
\usepackage{amsthm}
\usepackage{bbm}
\usepackage{cancel}
\usepackage{verbatim}
\usepackage{amsrefs}
\usepackage{enumitem}
\usepackage{hyperref}
\usepackage{tikz-cd}
%\usepackage[pdf]{pstricks}
\usepackage{braket}
\usetikzlibrary{cd}
\hypersetup{
     colorlinks   = true,
     citecolor    = red
}
%\usepackage{adjustbox}
\usepackage[ruled,linesnumbered]{algorithm2e}
\usepackage{adjustbox}
\usepackage{changepage}


\let\oldemptyset\emptyset
\let\emptyset\varnothing

\theoremstyle{plain}
\newtheorem{Thm}{Theorem}
\newtheorem{Prob}[Thm]{Problem}
%\theoremstyle{definition}
\newtheorem{Remark}[Thm]{Remark}
\newtheorem{Tech}[Thm]{Technical Remark}
\newtheorem*{Claim}{Claim}
%----------------------------------------
%CHAPTER STUFF
\newtheorem{theorem}{Theorem}%[chapter]
%\numberwithin{section}%{chapter}
%\numberwithin{equation}%{chapter}
%CHAPTER STUFF
%----------------------------------------
\newtheorem{lem}[theorem]{Lemma}
%\newtheorem{Q}[theorem]{Question}
\newtheorem{Prop}[theorem]{Proposition}
\newtheorem{Cor}[theorem]{Corollary}

\theoremstyle{definition}
\newtheorem{e}{Exercise}
\newtheorem{Def}[theorem]{Definition}
\newtheorem{Ex}[theorem]{Example}
\newtheorem{xca}[theorem]{Exercise}



\theoremstyle{remark}
\newtheorem{rem}[theorem]{Remark}


\newcommand{\Mod}[1]{\ (\mathrm{mod}\ #1)}
\newcommand{\norm}{\trianglelefteq}
\newcommand{\propnorm}{\triangleleft}
\newcommand{\semi}{\rtimes}
\newcommand{\sub}{\subseteq}
\newcommand{\fa}{\forall}
\newcommand{\R}{\mathbb{R}}
\newcommand{\z}{\mathbb{Z}}
\newcommand{\n}{\mathbb{N}}
\newcommand{\Q}{\mathbb{Q}}
\renewcommand{\c}{\mathbb{C}}
\newcommand{\bb}{\vspace{3mm}}

\newcommand{\bee}{\begin{equation}\begin{aligned}}
\newcommand{\eee}{\end{aligned}\end{equation}}
\newcommand{\nequiv}{\not\equiv}
\newcommand{\lc}[2]{#1_1 + \cdots + #1_{#2}}
\newcommand{\lcc}[3]{#1_1 #2_1 + \cdots + #1_{#3} #2_{#3}}
\newcommand{\ten}{\otimes} %tensor product
\newcommand{\fracc}{\frac}
\newcommand{\tens}{\otimes}
\newcommand{\lpar}{\left(}
\newcommand{\rpar}{\right)}
\newcommand{\floor}{\lfloor}

\renewcommand{\rm}{\normalshape}%text inside math
\renewcommand{\Re}{\operatorname{Re}}%real part
\renewcommand{\Im}{\operatorname{Im}}%imaginary part
\renewcommand{\bar}{\overline}%bar (wide version often looks better)
\renewcommand{\phi}{\varphi}

\makeatletter
\newenvironment{restoretext}%
    {\@parboxrestore%
     \begin{adjustwidth}{}{\leftmargin}%
    }{\end{adjustwidth}
     }
\makeatother

%---------END-OF-PREAMBLE---------
%---------------------------------







\begin{document}

\title{CSE 6331 Homework 5}

\date{SP18}

\author[Brendan Whitaker]{Brendan Whitaker}

\maketitle



\begin{enumerate}[label=\arabic*.]
\item \textit{Consider the single-pair shortest path problem.  We solved the problem using the
forward approach.  Now, solve it using the backward approach.  Your answer must
include:  the definition of $f(x)$, a recurrence, boundary conditions, and the goal.}

Recall that the problem is to find the shortest path from node $u$ to node $v$ in a directed acyclic graph. The graph $G = (V,E)$ is represented by a matrix: 
$$
d(i,j) = \begin{cases}
length(i,j) & (i,j) \in E\\
0 & i = j\\
\infty & otherwise
\end{cases}.
$$
Define $f(x)$ to be the shortest distance from node $u$ to node $x$. Then $f(x)$ is given by the recurrence: 
$$
f(x) = \min\Set{d(y,x) + f(y):indegree(y) \neq 0,y \neq u}. 
$$
Our boundary conditions are: 
$$
f(x) = \begin{cases}
\infty & x \neq u, indegree(x) = 0\\
0 & x = u
\end{cases}.
$$
And the goal is to compute $f(v)$. 
\bb

\item \textit{Implement the
third approach
of dynamic programming to the longest common
subsequence problem.  Your algorithm needs to print the actual longest common
subsequence.  Specifically, write two procedures:  (1) a
non-recursive
procedure
to  compute $L(i,j),1 \leq i,j \neq n$,  and  (2)  a
recursive
procedure  Longest$(i,j)$
such that Longest$(1,1)$ will print the longest common subsequence.}\\

\begin{restoretext}
\begin{algorithm}[H]\label{alg1}
\KwData{Two sequences:$A = (a_1,...,a_n),B = (b_1,...,b_n)$. }
\KwResult{The array $L(i,j)$ computed for $1 \leq i,j \leq n + 1$. }
\Begin{
\textbf{global array} $L[1..n + 1,1..n + 1],\phi[1..n,1..n]$\;
\textbf{initialize} $L[i, n + 1] \longleftarrow,L[n + 1,j] \longleftarrow 0$ for $1 \leq i,j \leq n + 1$;

\For{$i \leftarrow n$ \KwTo $1$}{
	\For{$j \leftarrow n$ \KwTo $1$}{
		\uIf{$a_i \neq b_j$}{
			$L[i,j] \longleftarrow \max\Set{L[i + 1,j],L[i,j + 1]}$;
		} \Else {
			$L[i,j] \longleftarrow 1 + L[i + 1,j+ 1]$;
		
		
		}
		
	
	}
}
}
\caption{Compute-Array-$L$}
\end{algorithm}
\end{restoretext}

\begin{restoretext}
\begin{algorithm}[H]\label{alg2}
\KwData{The computed array $L[1..n+1,1..n+1]$.}
\KwResult{Prints the longest common subsequence of $A$ and $B$. }
\Begin{
\tcc{Assume $L[1..n + 1,1..n + 1]$ has already been computed. }
	\uIf{$L[i,j] = 1 + L[i + 1,j + 1]$}{
		\textbf{print} $a_i$;
		
		Longest$(i + 1,j + 1)$;
		
	} \uElseIf{$L[i,j] = L[i + 1,j]$}{
		Longest$(i + 1,j)$;
	} \Else{
		Longest$(i,j + 1)$;
	}
}
\caption{Longest$(i,j)$}
\end{algorithm}
\end{restoretext}

\bb

\item \textit{Consider the all-pair shortest paths problem. Suppose
global
arrays $D^k[1..n,1..n]$ and $P^k[1..n,1..n]$ $1 \leq k \leq n$,  have  been  computed  as  in  the  straightforward
implementation.   Write  a
recursive
procedure  Path$(k,i,j)$  such  that  a  call  to Path$(n,i,j)$ will print the shortest path from $i$ to $j$.
  Note:  a path is a sequence
of vertices.  You may print a vertex more than once (e.g., it is OK to print a path
$(a,b,c,d)$ as $(a,a,b,c,c,c,d)$.)}

\bb

\begin{restoretext}
\begin{algorithm}[H]\label{alg3}
\KwData{The pre-computed arrays $D[1..n,1..n]$ and $P[1..n,1..n]$.}
\KwResult{Prints the shortest path from $i$ to $j$. }
\Begin{
\tcc{Assume $D[1..n,1..n]$ and $P[1..n,1..n]$ have already been computed. }
	\uIf{$P[i,j] = 0$} {
		\Return;
	}\Else{
		\textbf{print} $i$;
		
		\uIf{$i = j$} {
			\Return;
		}\Else {
			Path$(P[i,j],j)$;
		}
	}
}
\caption{Path$(i,j)$}
\end{algorithm}
\end{restoretext}







\end{enumerate}












\end{document}


