
%	options include 12pt or 11pt or 10pt
%	classes include article, report, book, letter, thesis

\title{\Huge Brendan Whitaker}



\author{\huge CSE 2221 Homework 3 \linebreak \\\\ \Large Professor Bucci}

\date{1/18/2017}
\documentclass[10pt]{article}
\usepackage{tikz}
\usepackage{graphicx}
\usepackage{cancel}
\usepackage{amsmath}
\usepackage{listings}


\begin{document}
\maketitle


\paragraph{1. }
\begin{list}{} {}
\item {a. }
\begin{lstlisting}
	int i = 0;
	while ((i*i) < n) {
		out.print(i*i + " ");
		i++;
	}
\end{lstlisting}
\item {b. }
\begin{lstlisting}
	int i = 1;
	while ((10*i) < n) {
		out.print(10*i + " ");
		i++;
	}
\end{lstlisting}
\item {c. }
\begin{lstlisting}
	//we assume non-negative integer powers of n. 
	int i = 0;
	while ((Math.pow(2, i)) < n) {
		out.print(Math.pow(2, i) + " ");
		i++;
	}
\end{lstlisting}
\end{list}
\hfill
\hfill
\hfill
\hfill
\hfill
\hfill
\hfill
\hfill
\hfill
\hfill
\hfill\\
\hfill\\



\paragraph{3. } { }
We rewrite the given for loop into a while loop: \\
\begin{lstlisting}
	int i = 1, s = 0;
	while (i <= 10) {
		s += i;
		i++;
	}
\end{lstlisting}

\paragraph{4. } { }
The following snippet of code sums the first n terms in the Gregory-Leibniz series: \\
\begin{lstlisting}
	int i = 0;
	while (i <= n) {
		pi += Math.pow(-1, i)/((2*n) + 1);
		i++;
	}
	pi *= 4;
\end{lstlisting}

\paragraph{5. } { }
The following snippet of code assigns to sum the result of adding up all integers of the form $n^2 + m^2$ where:
\begin{itemize}
\item both n and m are at least 1
\item $n^2 <$  areaBound
\item $m^2 <$  areaBound
\end{itemize}

\begin{lstlisting}
	int n = 1;
	int m;
	while (n*n < areaBound) {
		m = 1;
		while (m*m < areaBound) {
			sum += (n*n) + (m*m);
			m++;
		}
		n++;
	}
\end{lstlisting}

\hfill\\
\hfill\\

\paragraph{6. } { }
The following snippet of code keeps adding terms until the difference between two consecutive estimates is less than some predefined tolerance, say double epsilon = 0.0001: \\

\begin{lstlisting}
	int i = 0;
	double currentTerm = 1.0;
	while (currentTerm > epsilon) {
		currentTerm = Math.pow(-1, i)/((2*n) + 1);
		pi += currentTerm;
	}
	pi *= 4;
\end{lstlisting}










\end{document}


